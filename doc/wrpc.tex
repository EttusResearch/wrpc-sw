\documentclass[a4paper, 12pt]{article}
\usepackage[english]{babel}
\usepackage{fullpage}
\usepackage[table,x11names,dvipsnames,table]{xcolor}
\usepackage{pgf}
\usepackage{tikz}
\usepackage[pdftex]{hyperref}  % makes cross references and URLs clickable
\usepackage[overload]{textcase}
\usepackage{listings}
\usepackage{color}
\usepackage{textcomp}
\usepackage{longtable} % table over many pages
\usepackage[document]{ragged2e} %texta djustment
\usepackage{mdwlist} % to have tight itemization
\usepackage[T1]{fontenc}
\usepackage{lmodern}
%%%%%%%%%%%%%%%%%%%%%%%%%%%%%%%%%%%%
\usepackage{graphicx}
\usepackage{colortbl}
\usepackage{array}
\usepackage{multirow}

\newcommand{\newparagraph}[1]{\paragraph{#1}\mbox{}\\}

\definecolor{wrlblue}{RGB}{165,195,210}
\definecolor{wrlgray}{RGB}{209,211,212}
\definecolor{light-gray}{gray}{0.95}

\hypersetup{
    colorlinks,
    linkcolor={red!50!black},
    citecolor={blue!50!black},
    urlcolor={blue!80!black}
}

% set listings as in other WR-doc(s)
\lstset{columns=flexible, upquote=true, frame=single,
basicstyle=\footnotesize\ttfamily, backgroundcolor=\color{light-gray}, label=lst:init_src}

\newcommand{\multirowpar}[2]{
  \multirow{#1}{\hsize}{\parbox{\hsize}{\strut\raggedright#2\strut}}
}

\newcommand{\hdltablesection}[1]{
  \multicolumn{4}{|c|}{\bf\small#1}
}

\newcolumntype{L}[1]{>{\raggedright\let\newline\\\arraybackslash\hspace{0pt}}m{#1}}
\newcolumntype{M}[1]{>{\raggedright\let\newline\\\arraybackslash\hspace{0pt}\ttsmall}m{#1}}
\newcolumntype{C}[1]{>{\centering\let\newline\\\arraybackslash\hspace{0pt}}m{#1}}
\newcolumntype{D}[1]{>{\centering\let\newline\\\arraybackslash\hspace{0pt}\ttsmall}m{#1}}

\let\underscore\_
\renewcommand{\_}{\underscore\allowbreak}

\newenvironment{hdlparamtable}{
  \setlength{\extrarowheight}{1pt}
  \begin{center}
    \begin{longtable}{|M{.2\textwidth}|C{.09\textwidth}|D{.11\textwidth}|L{.5\textwidth}|}
      \firsthline
      \rowcolor{wrlblue}
      \bf{name} & \bf{type} & \bf{default} & \bf{description}\\
      \hline
      \endhead
}{
  \lasthline
    \end{longtable}
  \end{center}
}

\newenvironment{hdlporttable}{
  \setlength{\extrarowheight}{1pt}
  \begin{center}
    \begin{longtable}{|M{.25\textwidth}|C{.05\textwidth}|D{.05\textwidth}|L{.55\textwidth}|}
      \firsthline
      \rowcolor{wrlblue}
      \bf{name} & \bf{dir} & \bf{size} & \bf{description}\\
      \hline
      \endhead
}{
  \lasthline
    \end{longtable}
  \end{center}
}

\def \wrpcrelease {wrpc-v4.1}

\newcommand{\code}[1]{\texttt{#1}}

\newcommand{\tts}[1]{
  \texttt{\small{#1}}}

% same as \tts{}, without argument
\newcommand{\ttsmall}{\ttfamily\small}

\newcommand{\hrefwrpc}[1]{
  \tts{\href{http://www.ohwr.org/projects/wr-cores/repository/entry/#1?rev=\wrpcrelease}{#1}}}

\title{White Rabbit PTP Core User's Manual}
\author{Grzegorz Daniluk \\Dimitrios Lampridis \\Adam Wujek \hfill CERN (BE-CO-HT)}

\begin{document}
\input{version.tex}

\makeatletter
\hypersetup{pdftitle={\@title},pdfauthor={\@author}}
\raggedright
{\LARGE\bf\@title}\\[0.2 cm]
\hrule height 4pt \vspace{0.1cm}
{\large\gitrevinfo\hfill\today}\\
\vspace*{\fill}
{\large\@author}\\
\hrule height 2pt
\justify
\makeatother

\newpage

\tableofcontents

\newpage

\section{Introduction}

This is the user manual for the White Rabbit PTP Core (WRPC), part of the White
Rabbit project. It describes the building and running process, and it provides a
guide to instantiating the WRPC in your own HDL design.

If you don't want to get your hands dirty and prefer to start with the demo binaries
available at \url{http://www.ohwr.org/projects/wr-cores/files} for officially
supported boards, please skip Section~\ref{Building the Core} and move forward
directly to Section~\ref{Programming FPGA}. For help with instantiating the WRPC in
your own HDL design, see Section~\ref{sec:wrpc_hdl}.


% ##########################################################################
\subsection{Software and hardware requirements}
\label{Software and hardware requirements}

% ==========================================================================
\subsubsection{Repositories and Releases}
\label{Repositories and Releases}

This manual is about the official \gitrevinfo{} stable release of the White
Rabbit PTP Core (WRPC).  The code and documentation for the project is
distributed in the following places:

\begin{itemize*}

\item \url{http://www.ohwr.org/projects/wr-cores/documents}

	hosts the pdf documentation for every official release.

\item \url{http://www.ohwr.org/projects/wr-cores/files}

	place where you can find already synthesized demo bitstreams, ready to be
  downloaded to one of the officially supported boards

\item \url{git://ohwr.org/hdl-core-lib/wr-cores.git}

	repository with HDL sources of the WRPC

\item \url{git://ohwr.org/hdl-core-lib/wr-cores/wrpc-sw.git}

  repository with the LM32 software running inside the WRPC

\end{itemize*}
Other tools useful for building and running the WRPC can be downloaded from the
following locations:

\begin{itemize*}

\item \url{git://ohwr.org/misc/hdl-make.git}

  \textit{hdlmake} is used in the HDL synthesis process to create a Makefile and
  Xilinx ISE / Altera Quartus project file

\item \url{http://www.ohwr.org/attachments/download/1133/lm32.tar.xz}

  LM32 toolchain used to compile the LM32 software running inside the WRPC.
  This is a 32-bit version of the toolchain. If you encounter problems running
  this toolchain on modern 64bit machines, try 64 version described below.

\item \url{http://www.ohwr.org/attachments/download/3868/lm32_host_64bit.tar.xz}

  LM32 toolchain used to compile the LM32 software running inside the WRPC
  (64-bit version of the toolchain).

  \end{itemize*}
Repositories containing the WRPC gateware and software (\textit{wr-cores},
\textit{wrpc-sw}) are tagged with \texttt{wrpc-v4.1} tag. Other tools used to
build the core and load it into the FPGA should be used in their newest stable
releases, unless otherwise stated.

% ==========================================================================
\subsubsection{Required hardware}
\label{Required hardware}

The minimum hardware set required to run the WR PTP Core reference firmware
depends on the hardware platform you want to use. One of the following setups
can be chosen:
\begin{itemize}
\item \href{http://www.ohwr.org/projects/spec}{SPEC PCIe board} +
  \href{http://www.ohwr.org/projects/fmc-dio-5chttla}{FMC DIO card} + PC
  computer running Linux
\item \href{http://www.ohwr.org/projects/svec}{SVEC VME board} + VME crate with
  a single board computer running Linux\footnote{\label{note_a20}In our test setup
    we used MEN A20 board}
\item \href{http://www.ohwr.org/projects/vfc-hd}{VFC-HD VME board} +
  \href{http://www.ohwr.org/projects/fmc-dio-5chttla}{FMC DIO card} +
  VME crate with a single board computer running Linux\footref{note_a20}
\item \href{http://www.ohwr.org/projects/fasec}{FASEC board} +
  \href{http://www.ohwr.org/projects/fmc-dio-5chttla}{FMC DIO card}
\end{itemize}

To be able to test White Rabbit synchronization you would also need
additional components regardless of the reference platform chosen from the list
above:
\begin{itemize}
  \item another WR node (e.g. one of the reference boards listed above) or a
    White Rabbit Switch
  \item a pair of WR-supported SFP transceivers\footnote{The list of supported
    SFPs can be found on our wiki page
    \url{http://www.ohwr.org/projects/white-rabbit/wiki/SFP}}
  \item a roll of G652, single mode fiber to connect your WR devices
\end{itemize}

% ##########################################################################
\newpage
\section{Building the Core}
\label{Building the Core}

\textbf{Note:} you can skip this chapter if you want to use the release binaries
available from \textit{ohwr.org}.

\vspace{1em}
Depending on your needs, building the WRPC can be a one- or two-step process.
In most of the cases you only need to synthesize the FPGA firmware (section
\ref{HDL synthesis}). This way, you get a working WRPC with the default/release
LM32 software running inside the core. If, for some reasons, you need to modify
the LM32 software, please check also section \ref{LM32 software compilation}
which contains a description of the software compilation process.

% ==========================================================================
\subsection{HDL synthesis}
\label{HDL synthesis}

Before running the synthesis process you have to make sure your environment is
set up correctly. You will need a synthesis software from your FPGA vendor.
Depending on whether you want to run the WRPC on Xilinx (e.g. SPEC, SVEC boards) or
Altera/Intel (e.g. VFC-HD) FPGA, you should install either Xilinx ISE or Quartus
Prime software.

\subsubsection{Setting up Xilinx ISE on Linux}
\label{Setting up Xilinx ISE}
To synthesize the FPGA firmware containing the WRPC, Xilinx ISE with free of
charge WebPack license is enough. ISE provides a set of scripts:
\texttt{settings32.sh}, \texttt{settings32.csh}, \texttt{settings64.sh} and
\texttt{settings64.csh} that configure all the system variables to let you
easily run the software. Depending on the shell you use and whether your Linux is
32 or 64-bits you should execute one of them before the other tools are used.
For 64-bit system and BASH shell you should call (assuming that ISE is installed
in the default \textit{/opt} directory):
\begin{lstlisting}
$ /opt/Xilinx/<version>/ISE_DS/settings64.sh
\end{lstlisting}

You can check if the shell is configured correctly by verifying if the
\texttt{\$XILINX} variable contains path to your ISE installation directory.

\subsubsection{Setting up Quartus Prime on Linux}
\label{Setting up Quartus Prime}
To be able to synthesize the WRPC for Arria V FPGA (which is used on the VFC-HD
board) you need at least a license for the Quartus Prime Standard Edition with
the support of Arria V family. To set up the software after it is installed, you
should add the location of its binaries to your \texttt{\$PATH} environment
variable. Assuming you have installed the software in \textit{/opt/altera}, the
following command should be executed:
\begin{lstlisting}
$ export PATH=/opt/altera/16.0/quartus/bin:$PATH
\end{lstlisting}

\subsubsection{Downloading the sources}
Thanks to the \textit{hdlmake} tool, the synthesis process for the reference
designs does not differ between Xilinx and Altera/Intel based boards. The tool creates
synthesis Makefile as well as ISE/Quartus project file based on a set of
Manifest.py files that you will find in the \textit{wr-cores} repository.\\

First, please clone the \textit{hdlmake} repository from its location given in
section \ref{Repositories and Releases}:
\begin{lstlisting}
$ git clone git://ohwr.org/misc/hdl-make.git <your_location>/hdl-make
$ cd <your_location>/hdl-make
$ git checkout c4789c4
\end{lstlisting}
This checks out \textit{hdlmake} version 2.1 patched with the Arria V FPGA
support (VFC-HD board).

Then, using your favorite editor, you should create an \textit{hdlmake} script in
/usr/bin to be able to call it from any directory. The script should have the
following content:
\begin{lstlisting}
#!/usr/bin/env bash
python2.7 <path_to_hdlmake_sources>/hdl-make/hdlmake/__main__.py #@
\end{lstlisting}

Please, make your script executable:
\begin{lstlisting}
$ chmod a+x /usr/bin/hdlmake
\end{lstlisting}

Having all the tools in place, you can now clone the main WR PTP Core git
repository for the v4.1 release. The set of commands below clones the WR PTP Core
repository, checks out the release tag, and downloads other HDL repositories
(submodules) needed to synthesize the core:
\begin{lstlisting}
$ git clone git://ohwr.org/hdl-core-lib/wr-cores.git <your_location>/wr-cores
$ cd <your_location>/wr-cores
$ git checkout wrpc-v4.1
$ git submodule init
$ git submodule update
\end{lstlisting}

The local copies of the submodules are stored to:

\texttt{<your\_location>/wr-cores/ip\_cores}

\vspace{1em}
\textbf{Note:} If you use the WRPC within another project (like
\textit{wr-nic}), you may need to checkout another release tag for this
repository. Please refer to the project's documentation to find out which
version of this package you need to build.

\subsubsection{Running the synthesis (ISE, Quartus)}
The subdirectory you should enter to run the synthesis depends on the hardware
platform you use:
\begin{itemize*}
  \item \textbf{SPEC}: \texttt{<your\_location>/wr-cores/syn/spec\_ref\_design}
  \item \textbf{SVEC}: \texttt{<your\_location>/wr-cores/syn/svec\_ref\_design}
  \item \textbf{VFC-HD}: \texttt{<your\_location>/wr-cores/syn/vfchd\_ref\_design}
\end{itemize*}

After selecting a proper location from the list above, please call
\textit{hdlmake} without any arguments to create the Makefile and project file:
\begin{lstlisting}
$ hdlmake
\end{lstlisting}

After that, the actual synthesis is just the matter of executing:
\begin{lstlisting}
$ make
\end{lstlisting}

This takes (depending on your computer) about 10 minutes and should generate
bitstream files in various formats depending on your selected reference
hardware:
\begin{itemize*}
  \item \textbf{SPEC}: \texttt{spec\_wr\_ref\_top.bin}, \texttt{spec\_wr\_ref\_top.bit}
  \item \textbf{SVEC}: \texttt{svec\_wr\_ref\_top.bin}, \texttt{svec\_wr\_ref\_top.bit}
  \item \textbf{VFC-HD}: \texttt{vfchd\_wr\_ref.sof}
\end{itemize*}

You can select the bitstream format to be downloaded to FPGA depending on the
programming method:
\begin{itemize}
  \item \textbf{*.bin} files to program the Xilinx FPGA on SPEC or SVEC board
    using the official software support package (\textit{spec-sw},
    \textit{svec-sw}). See section \ref{Programming FPGA} for more
    information.
  \item \textbf{*.bit} files to program the Xilinx FPGA with Xilinx USB Platform
    Cable (using e.g. Xilinx Impact tool)
  \item \textbf{*.sof} file to program the Intel FPGA (VFC-HD board) using using
    the Altera / Intel JTAG cable
\end{itemize}

If, you would like to clean-up the repository to start building everything from
scratch you can use the following commands:
\begin{itemize*}
\item \texttt{\$ make clean} - removes all synthesis reports and log files;
\item \texttt{\$ make mrproper} - removes \texttt{*.bin}, \texttt{*.bit} and
  \texttt{*.sof} files;
\end{itemize*}

\subsubsection{Running the synthesis (Vivado)}
The workflow in Xilinx Vivado is different than for ISE and Quartus. It is
heavily based on packed IP cores and schematic entry. Therefore, if you would
like to synthesize the reference design for Zynq (FASEC board), you
need to go through two Vivado projects:
\begin{itemize*}
  \item \texttt{<your\_location>/wr-cores/syn/wrc\_board\_fasec\_ip} - is a
    project created with \textit{hdlmake} from all HDL files necessary to
    synthesize the WR PTP Core with all its peripherals for Zynq (Kintex-7 FPGA).
    The project contains also IP-XACT module description to generate Vivado IP
    Core for further synthesis with a complete FASEC reference design.
  \item \texttt{<your\_location>/wr-cores/syn/fasec\_ref\_design} - is a main
    reference design project for FASEC board. It is made with the Vivado Block
    Design instantiating Processing System, reset circuits, AXI interconnects
    and WR PTP Core IP generated from \texttt{wrc\_board\_fasec\_ip} project.
\end{itemize*}

First, please execute the \texttt{build.tcl} script in Vivado batch mode
\footnote{You can also use Tools->Run Tcl Script... from Vivado gui} to create
WR PTP Core IP:
\begin{lstlisting}
$ cd <your_location>/wr-cores/syn/wrc_board_fasec_ip
$ vivado -mode batch -source build.tcl
\end{lstlisting}

After this step, you can generate the main reference design Vivado project,
using Tcl scripts in the repository:
\begin{lstlisting}
$ cd <your_location>/wr-cores/syn/fasec_ref_design
$ vivado -mode batch -source build.tcl
\end{lstlisting}

Open the project file \texttt{fasec\_ref\_design.xpr} and continue with regular
Vivado flow to synthesize the bitstream.

% ==========================================================================
\subsection{LM32 software compilation}
\label{LM32 software compilation}

\textbf{Note:} By default, the LM32 software for a stable release is embedded
inside the FPGA bitstream you've downloaded from \textit{ohwr.org} or
synthesized in section \ref{HDL synthesis}. You can skip this section, unless
you need to make some custom changes to the LM32 software and compile it
yourself.\\

\textbf{Note:} To compile also the host tools that come with this software
package you will need a \emph{readline-dev} library. In some Linux distributions
you would have to install it manually. E.g. in Ubuntu, please install
\emph{libreadline-dev} package.\\

First, you need to download and unpack the LM32 toolchain from the location
mentioned in section \ref{Repositories and Releases}. The following example
uses 32bit version of a toolchain. If you encounter problems running it, please
use 64bit version.
\begin{lstlisting}
$ wget http://www.ohwr.org/attachments/download/1133/lm32.tar.xz
$ tar xJf lm32.tar.xz -C <your_location>
\end{lstlisting}

Then you need to set a \texttt{CROSS\_COMPILE} environment variable in order
to compile the software for the LM32 processor:
\begin{lstlisting}
$ export CROSS_COMPILE="<your_location>/lm32/bin/lm32-elf-"
\end{lstlisting}

To get the sources of the WRPC software, please clone the \textit{wrpc-sw} git
repository tagged with \texttt{wrpc-v4.1} tag. The commands in the listing below
clone the \textit{wrpc-sw} repository together with submodules needed for this software.\\
\begin{lstlisting}
$ git clone git://ohwr.org/hdl-core-lib/wr-cores/wrpc-sw.git \
<your_location>/wrpc-sw
$ cd <your_location>/wrpc-sw
$ git checkout wrpc-v4.1
\end{lstlisting}

\textbf{Note:} If you use WRPC within another project, you may need to checkout
a different tag or a specific commit. If this applies, please refer to a
documentation for this project.\\

Before you can compile \textit{wrpc-sw}, you can make some configuration
choices. The package uses \textit{Kconfig} as a configuration engine, so you may
run one of the following commands (the first two are text-mode, the third uses
a KDE GUI and the fourth uses a Gnome GUI):
\begin{lstlisting}
$ make menuconfig
$ make config
$ make xconfig
$ make gconfig
\end{lstlisting}

Other \code{Kconfig} target applies, like \code{config}, \code{oldconfig}
and so on. A few default known-good configurations are found in
\texttt{./configs} and you choose one by \code{make}ing it by name. For all
three boards mentioned in this manual \code{spec\_defconfig} can be used.
\begin{lstlisting}
$ make spec_defconfig
\end{lstlisting}

After the package is configured, just run \code{make} without parameters to
build your binary file:
\begin{lstlisting}
$ make
\end{lstlisting}

The first time you build, the \textit{Makefile} automatically downloads
the \textit{git submodules} of this package, unless you already did that
by hand. The second and later build won't download anything
from the network.

The compilation process produces the binary in 3 different formats:
\begin{itemize*}
  \item \textit{wrc.bin} can be then used with the loader from \textit{spec-sw}
    or \textit{svec-sw} software package to program the LM32 inside the White
    Rabbit PTP Core (see section \ref{Programming FPGA} for details).
  \item \textit{wrc.bram} you can use to initialize WRPC internal RAM with LM32
    software during the synthesis for Xilinx FPGAs.
  \item \textit{wrc.mif} you can use to initialize WRPC internal RAM with LM32
    software during the synthesis for Altera FPGAs.
\end{itemize*}
The location of \textit{wrc.bram}/\textit{wrc.mif} files should be passed to the
WR PTP Core HDL using the \texttt{g\_dpram\_initf} generic.

% ##########################################################################
\newpage
\section{Programming FPGA with WRPC firmware}
\label{Programming FPGA}

% ==========================================================================
\subsection{Programming FPGA on SPEC}
\label{Programming FPGA on SPEC}

\textbf{Note:} If you use a more recent version of the \texttt{spec-sw} than the
one described in this manual, the up-to-date documentation can always be found
in \textit{doc/} subdirectory of the \texttt{spec-sw} git repository.\\

First you need to clone the SPEC board software support package
(\textit{spec-sw}) from \textit{ohwr.org}. It is a set of Linux kernel drivers and
userspace tools, that interacts with a SPEC board plugged into a PCI-Express
slot.\\

\begin{lstlisting}
$ git clone git://ohwr.org/fmc-projects/spec/spec-sw.git <your_location>/spec-sw
$ cd <your_location>/spec-sw
$ git checkout 0745464
$ make
\end{lstlisting}

You have to download also the "golden" firmware for SPEC card. It is used by
the drivers to recognize the hardware:
\begin{lstlisting}
$ wget http://www.ohwr.org/attachments/download/4057/spec-init.bin-2015-09-18
$ sudo mv spec-init.bin-2015-09-18 /lib/firmware/fmc/spec-init.bin
\end{lstlisting}

Now you can load the drivers necessary to access the SPEC board from your
system:
\begin{lstlisting}
$ sudo insmod fmc-bus/kernel/fmc.ko
$ sudo insmod kernel/spec.ko
\end{lstlisting}

You can use the \textit{dmesg} Linux command to verify if this step succeeded.
Among plenty of messages you should be able to find something very similar to:
\begin{lstlisting}[basicstyle=\scriptsize\ttfamily]
[1275526.738895] spec 0000:20:00.0:  probe for device 0020:0000
[1275526.738906] spec 0000:20:00.0: PCI INT A -> GSI 16 (level, low) -> IRQ 16
[1275526.738913] spec 0000:20:00.0: setting latency timer to 64
[1275526.743102] spec 0000:20:00.0: got file "fmc/spec-init.bin", 1485236 (0x16a9b4) bytes
[1275526.934710] spec 0000:20:00.0: FPGA programming successful
[1275527.296754] spec 0000:20:00.0: mezzanine 0
[1275527.296756]       Manufacturer: CERN
[1275527.296757]       Product name: FmcDio5cha
\end{lstlisting}

Now, you are ready to program the FPGA with your synthesized bitstream from
\texttt{<your\_location>/wr-cores/syn/spec\_ref\_design/spec\_wr\_ref\_top.bin}
\begin{lstlisting}
$ sudo tools/spec-fwloader \
<your_location>/wr-cores/syn/spec_ref_design/spec_wr_ref_top.bin
\end{lstlisting}

If everything went right up to this moment you have your board running the FPGA 
bitstream with the WR PTP Core. If you need to load your own \texttt{wrc.bin}
compiled in section \ref{LM32 software compilation}, you can use the
\texttt{spec-cl} tool:
\begin{lstlisting}
$ sudo tools/spec-cl <your_location>/wrpc-sw/wrc.bin
\end{lstlisting}

After all these steps, you should be able to start a Virtual-UART tool (also
part of the \textit{spec-sw} package) that will be used to interact with the
WRPC shell:
\begin{lstlisting}
$ sudo tools/spec-vuart
\end{lstlisting}
If you are able to see the shell prompt \textit{wrc\#} this means the Core
is up and running on your SPEC card. Congratulations !\\

Starting from version 3.0, WR PTP Core uses a flash memory chip on the carrier
as a default place for storing calibration parameters and an init script.
The storage format of this information is organized in SDBFS filesystem.
Therefore, starting from v3.0 you have to write the empty
SDBFS filesystem image to the flash before running the WRPC. The simplest way of
doing this is by calling a WR PTP Core shell command:
\begin{lstlisting}
wrc# sdb fs 0
\end{lstlisting}
You should see the output similar to:
\begin{lstlisting}[basicstyle=\scriptsize\ttfamily]
filename: .                 ; first: 2e0000; last: 32007f
filename: wr-init           ; first: 2f0000; last: 2f00ff
filename: calibration       ; first: 300000; last: 30007f
filename: mac-address       ; first: 310000; last: 310005
filename: sfp-database      ; first: 320000; last: 32007f
Formatting SDBFS in Flash(0x2e0000)...
\end{lstlisting}

The other two methods: through the PCIe bus and using a Xilinx JTAG cable are
described in appendix \ref{appendix:writing_sdbfs}.


% ==========================================================================
\subsection{Programming FPGA on other boards}

Currently the driver packages for SVEC and VFC-HD boards depend on other CERN
drivers and front-end infrastructure. We are working to make them exportable. If
you are a CERN user of SVEC, VFC-HD board or SPEC in a Front-end computer,
please check
\url{https://wikis.cern.ch/display/HT/WR+Nodes} (accessible \textbf{only} from
CERN network) for more instructions. If you are outside CERN and would like to
use one of these boards, please contact us.

% ==========================================================================
\newpage
\section{Using WRPC shell}

\subsection{Writing configuration}
\label{Writing configuration}

First, you should perform a few configuration steps through the WRPC shell
before using the core.

\noindent\textbf{Note:} the examples below describe only a subset of the WRPC
Shell commands. The full list of supported commands can be found in Appendix
\ref{WRPC Shell commands}.\\

Before making any configuration changes, it is recommended to stop the
PTP daemon.  Then, the messages from the daemon will not show up to the
console while you interact with the shell.

\begin{lstlisting}
wrc# ptp stop
\end{lstlisting}

\noindent First you should make sure your board has a proper MAC address
assigned:
\begin{lstlisting}
wrc# mac get
\end{lstlisting}
If the result of above command is \texttt{MAC-address: 22:33:ww:xx:yy:zz} (where
ww, xx, yy, zz are hex values), this
means MAC was not yet configured and stored in the Flash/EEPROM. The value is
based on thermometer serial number as is unique among SPEC devices. This is
globally accepted as ``locally assigned'', but you might want to assign your own
address. A value \texttt{22:33:44:55:66:77} is the final fallback if no
thermometer is found.

You should get the MAC for your board from its manufacturer. To configure the
address and store it into the Flash/EEPROM (so that it's automatically loaded
every time the WRPC starts) you should type two commands in the shell:
\begin{lstlisting}
wrc# mac set xx:xx:xx:xx:xx:xx
wrc# mac setp xx:xx:xx:xx:xx:xx
\end{lstlisting}
where \texttt{xx:xx:xx:xx:xx:xx} is the MAC address of your board.\\

Next, you should input calibration fixed delays values and alpha parameters. The
example below clears any existing entries and adds two Axcen transceivers with
$\Delta_{TX}$, $\Delta_{RX}$ and $\alpha$ parameters associated with them.

\begin{lstlisting}
wrc# sfp erase
wrc# sfp add AXGE-1254-0531 180750 148326 72169888
wrc# sfp add AXGE-3454-0531 180750 148326 -73685416
\end{lstlisting}

To check the content of the SFP database you can execute the \textit{sfp show}
shell command.\\

\noindent\textbf{Note:} The $\Delta_{TX}$ and $\Delta_{RX}$ parameters above are
the defaults for wrpc-v4.1 release bitstream available on \textit{ohwr.org},
running on the SPEC v4 board and calibrated to port 1 of a~WR Switch
v3.3. These values as well as the default parameters for other boards are
available on the calibration wiki page
(\url{http://www.ohwr.org/projects/white-rabbit/wiki/Calibration}). If you
re-synthesize the firmware or want to use the most accurate estimation of
the fixed delays and alpha for your fiber, you should read and perform the WR
Calibration procedure (\url{http://www.ohwr.org/documents/213}).\\

The WRPC  mode of operation (GrandMaster/Master/Slave) can be set
using the \textit{mode} command:

\begin{lstlisting}
wrc# mode gm       # for GrandMaster mode
wrc# mode master   # for Master mode
wrc# mode slave    # for Slave mode
\end{lstlisting}

This stops the PTP daemon, changes the mode of operation, but does not start it
automatically. Therefore, after calling it, you need to start the daemon
manually:

\begin{lstlisting}
wrc# ptp start
\end{lstlisting}

\noindent\textbf{Note:} For running the GrandMaster mode, you need to have the
\texttt{g\_with\_external\_clock\_input} generic enabled in your FPGA firmware.
You also must provide 1-PPS and 10MHz signal from an external source (e.g. GPS receiver or
Cesium clock). Depending on your board you should connect:
\begin{itemize*}
  \item \textbf{SPEC}: DIO mezzanine LEMO No.4 for 1-PPS, LEMO No.5 for 10MHz
  \item \textbf{SVEC}: SVEC LEMO No.4 for 1-PPS, LEMO No.3 for 10MHz
  \item \textbf{VFC-HD}: DIO mezzanine LEMO No.4 for 1-PPS, LEMO No.5 for 10MHz
\end{itemize*}

\vspace{1em}
WRPC has a possibility to load user-defined intialization script on every
startup. If you run WRPC release v4.1 or higher, the default LM32 software
comes with a built-in init script. The default script disables VLAN support,
loads calibration values from the SFP database (provided that it was written
earlier by the user) and starts PTP in slave mode. Here are the actual WRPC
shell commands executed from the built-in init script:
\begin{lstlisting}
vlan off
ptp stop
sfp match
mode slave
ptp start
\end{lstlisting}

You can still define your own init script that will be saved in the
Flash/EEPROM memory attached to the WRPC. In that case, user-defined set of
commands is executed after the built-in script. For example, if you would
like to expand the default configuration by your own script that assigns an IP
address, you need to to execute the following WRPC shell commands:
\begin{lstlisting}
wrc# init erase
wrc# init add ip set 192.168.1.5
\end{lstlisting}

Another typical situation would be configuring the WRPC to run in GrandMaster or
Master mode. In this case, assuming you also want to configure the IP address
(like in the previous example) you would need to run the following WRPC shell
commands:
\begin{lstlisting}
wrc# init erase
wrc# init add mode master # for Free-running Master mode or
wrc# init add mode gm     #for GrandMaster mode
wrc# init add ptp start
wrc# init add ip set 192.168.1.5
\end{lstlisting}

You can always check the content of the built-in and user-defined init scripts
by calling \texttt{init show} command:
\begin{lstlisting}
wrc# init show
-- built-in script --
vlan off
ptp stop
sfp match
mode slave
ptp start
-- user-defined script --
ip set 192.168.1.5
\end{lstlisting}

\noindent\textbf{Note:} This simple configuration disables VLANs configuration
in the WR PTP Core. If, for your network configuration you need to configure
VLANs, please check the instructions in section \ref{VLAN Support}.

% ==========================================================================
\subsection{Running the Core}
\label{Running the Core}

Having the SFP calibration database, and eventually a user-defined init script
created in \ref{Writing configuration} you can now restart the WRPC by typing
the shell command:

\begin{lstlisting}
wrc# init boot
\end{lstlisting}

You should see log messages that confirm the execution of the initialization
script:
\begin{minipage}{\textwidth}
\begin{lstlisting}
executing: vlan off
current vlan: 0 (0x0)
executing:  ptp stop
PTP stop
executing:  sfp match
AXGE-3454-0531  
SFP matched, dTx=180750 dRx=148326 alpha=-73685416
executing:  mode slave
PTP stop
Locking PLL
executing:  ptp start
PTP start
Slave Only, clock class set to 255
executing: ip set 192.168.1.5
IP-address: 192.168.1.5 (static assignment)
\end{lstlisting}
\end{minipage}

Now, your device runs as a WR Slave.\\

\textbf{Important!!!} WRPC needs to make a calibration of t24p phase transition
value. It has to be done only once for a new bitstream and is performed
automatically when WRPC runs in the Slave mode. That is why it is very
important, even if WRPC is meant to run in the Master mode, to configure it to
Slave for a moment and connect to any WR Master. This has to be repeated every
time a new bitstream (gateware) is deployed. The measured value is automatically
stored to Flash/EEPROM and used later in the Master or GrandMaster mode.\\

The Shell also contains a monitoring function which you can use to check the
WR synchronization status:

\begin{lstlisting}
wrc# gui
\end{lstlisting}

The information is presented in a clear, auto-refreshing screen (see the figure blow). 
The information is refreshed at every WR iteration or periodically if
nothing else happens (so you see an up-to-date timestamp). The period
defaults to 1 second and can be changed using the \texttt{refresh} command. To
exit from this console mode press <Esc>. A full description of the information
reporter by \textit{gui} is provided in Appendix \ref{WRPC GUI elements}.

\noindent\textbf{Note:} the \textit{Synchronization status} and \textit{Timing
parameters} in \texttt{gui} are available only in the WR Slave mode. When
running as WR Master, you can see only the current date and time,
link status, Tx and Rx packet counters, IP, lock and calibration status.

\vspace{1em}
\includegraphics[width=12cm]{wrpc_mon.png}
\vspace{1em}

If you want to log statistics from the WRPC operation, it is better to use a
\texttt{stat} shell command. It reports the same information as \texttt{gui},
but in a single line, which is easier to parse and analyze:

\begin{lstlisting}
wrc# stat
lnk:1 rx:172338 tx:151811 lock:1 ptp:slave sv:1 ss:'TRACK_PHASE' aux0:1 sec:6047 \
nsec:828412744 mu:836453 dms:398530 dtxm:224455 drxm:232479 dtxs:180667 drxs:149251 \
asym:39393 crtt:49643 cko:1 setp:5082 ucnt:270 hd:31734 md:46228 ad:0 temp: 52.6875 C
lnk:1 rx:172392 tx:151860 lock:1 ptp:slave sv:1 ss:'TRACK_PHASE' aux0:1 sec:6049 \
nsec:399776360 mu:836452 dms:398530 dtxm:224455 drxm:232479 dtxs:180667 drxs:149251 \
asym:39392 crtt:49642 cko:2 setp:5082 ucnt:271 hd:31730 md:46211 ad:0 temp: 52.6875 C
(...)
\end{lstlisting}

\vspace{1em}
Unlike \texttt{gui}, the \texttt{stat} command runs asynchronously: you can still
issue shell commands while stats are running. You can stop statistics by running
\texttt{stat} again. As an alternative to the toggling action of \texttt{stat}
alone, you can use ``\texttt{stat on}'' or ``\texttt{stat off}''.

Statistics are printed every time the WR servo runs; thus no statistics
are reported when the node is running in master and GrandMaster mode, nor when your node
is running as slave and the master disappeared.\\

You can verify the synchronization performance by observing the offset between
1-PPS signals from the WR Master and your WRPC running in the Slave mode. Please
remember to use oscilloscope cables of the same length and type (with the same
delay), or take their delay difference into account in your measurements.
Depending on your board, 1-PPS output is produced on:
\begin{itemize*}
  \item \textbf{SPEC}: DIO mezzanine LEMO No.1
  \item \textbf{SVEC}: SVEC LEMO No.1
  \item \textbf{VFC-HD}: VFC-HD LEMO L3
\end{itemize*}

% ==========================================================================
\newpage
\section{WRPC diagnostics}
\subsection{SNMP}
\label{Diagnostics via SNMP}

Up to the version 4.0 of WRPC the only way to perform diagnostics
of the \texttt{wrpc-sw} was to use serial console with tools like \textit{gui}, \textit{stat},
etc. For some set-ups, like standalone node, it is very inconvenient to use
external console for diagnostics.

Starting with version 4.0 of WRPC, it is possible to include the \textit{Mini
SNMP responder}, which allows to perform remote diagnostics using \textit{SNMP} via
a port connected to the \textit{Write Rabbit} network.

The configuration file of WRPC contains the following
SNMP-related options:
\begin{itemize*}
\item \texttt{CONFIG\_SNMP} -- include the \textit{Mini SNMP responder} into WRPC
\item \texttt{CONFIG\_SNMP\_SET} -- enable the support of SNMP \textit{SET} packets
\item \texttt{CONFIG\_SNMP\_VERBOSE} -- enable verbose output from the \textit{Mini SNMP
      responder} on the WRPC's console
\end{itemize*}

The MIB file describing WRPC's OIDs can be found in the \texttt{lib} directory
of the \texttt{wrpc-sw} repo.
So far, the \textit{Mini SNMP responder} supports version 1 and a subset of version
2c of the SNMP protocol.
The following types of requests are supported:
\begin{itemize*}
   \item GET -- get value of a given OID
   \item GETNEXT -- get value of a next OID after the given OID (this is used
         for \texttt{snmpwalk}s)
   \item SET -- change the value of a given OID (so far used only for adding
         SFP's to the database and PTP restarts)
\end{itemize*}
The \textit{Mini SNMP responder} does not support:
\begin{itemize*}
   \item bulk requests packets (GETBULK)
   \item more than one OID in the request packet
   \item \texttt{trap} and \texttt{inform} packets
   \item encryption
   \item authentication
   \item SNMPv2c return error types; all returned error types follows SNMPv1
\end{itemize*}
To make examples more readable, listings below use \texttt{SNMP\_OPT} environment
variable. Make sure you set it properly in your shell.
\begin{lstlisting}
 $ SNMP_OPT="-c public -v 2c -m WR-WRPC-MIB -M +/var/lib/mibs/ietf:lib 192.168.1.20"
\end{lstlisting}
where:
\begin{sloppypar} % to prevent \texttt{} from going to the margine
\begin{itemize*}
   \item \texttt{-c public} -- sets SNMP community as "\textit{public}"
   \item \texttt{-v 2c} -- specifies SNMP version
   \item \texttt{-m WR-WRPC-MIB} -- specifies MIBs to be loaded
   \item \texttt{-M +/var/lib/mibs/ietf:lib} -- contains path to MIBs in the host
         system (\texttt{/var/lib/mibs/ietf}) and path to \texttt{WR-WRPC-MIB} (\texttt{lib});
         on Debian-like systems default MIBs can be downloaded using
         \texttt{download-mibs} command (package \texttt{snmp-mibs-downloader}); on
         CentOS and RedHat MIBs are included in the \texttt{libsmi} package
   \item \texttt{192.168.1.20} -- the IP address of the target board
\end{itemize*}\end{sloppypar}\noindent
For example, to get the system uptime please execute the \texttt{snmpget} command:
\begin{lstlisting}
 $ snmpget $SNMP_OPT wrpcTimeSystemUptime.0
\end{lstlisting}
To get a dump of all available OIDs please execute the \texttt{snmpwalk}
command:
\begin{lstlisting}
 $ snmpwalk $SNMP_OPT wrpcCore
\end{lstlisting}
Part of the \texttt{snmpwalk}'s output:
\begin{lstlisting}
 WR-WRPC-MIB::wrpcVersionHwType.0 = STRING: spec
 WR-WRPC-MIB::wrpcVersionSwVersion.0 = STRING: wrpc-v3.0-251-g14e952e
 WR-WRPC-MIB::wrpcVersionSwBuildBy.0 = STRING: Adam Wujek
 WR-WRPC-MIB::wrpcVersionSwBuildDate.0 = STRING: Jun  7 2016 18:12:24
 WR-WRPC-MIB::wrpcTimeTAI.0 = Counter64: 1465375022
 WR-WRPC-MIB::wrpcTimeTAIString.0 = STRING: 2016-06-08-08:37:02
 WR-WRPC-MIB::wrpcTimeSystemUptime.0 = Timeticks: (18186) 0:03:01.86
 WR-WRPC-MIB::wrpcTemperatureName.1 = STRING: pcb
 WR-WRPC-MIB::wrpcTemperatureValue.1 = STRING: 38.5625
 WR-WRPC-MIB::wrpcSpllMode.0 = INTEGER: slave(3)
 WR-WRPC-MIB::wrpcSpllIrqCnt.0 = Counter32: 1259605
 [...]
 WR-WRPC-MIB::wrpcPortSfpInDB.0 = INTEGER: inDataBase(2)
 WR-WRPC-MIB::wrpcPortInternalTx.0 = Counter32: 452
 WR-WRPC-MIB::wrpcPortInternalRx.0 = Counter32: 869
 WR-WRPC-MIB::wrpcSfpPn.1 = STRING: AXGE-1254-0531
 WR-WRPC-MIB::wrpcSfpDeltaTx.1 = INTEGER: 180750
 WR-WRPC-MIB::wrpcSfpDeltaRx.1 = INTEGER: 148326
 WR-WRPC-MIB::wrpcSfpAlpha.1 = INTEGER: 72169888
 End of MIB
\end{lstlisting}

It is recommended to use SNMP v2c for communication with a WRPC.
Please note that when the version 1 of SNMP is used, 64 bit counters are not
supported. This makes impossible to read some WRPC's objects with
SNMPv1.

% --------------------------------------------------------------------------
\subsubsection{Managing SFP database via SNMP}
\label{Managing SFP database via SNMP}

The SFPs database can be displayed using the \texttt{sfp show} command from
the WRPC's console:
\begin{lstlisting}
 wrc# sfp show
 1: PN:AXGE-1254-0531   dTx:   180750 dRx:   148326 alpha: 72169888
 2: PN:AXGE-3454-0531   dTx:   180750 dRx:   148326 alpha: -73685416
\end{lstlisting}
The same data is exported by the \textit{Mini SNMP responder} via the table
\texttt{wrpcSfpTable}:

\begin{lstlisting}
 $ snmpwalk $SNMP_OPT wrpcSfpTable
 WR-WRPC-MIB::wrpcSfpPn.1 = STRING: AXGE-1254-0531
 WR-WRPC-MIB::wrpcSfpPn.2 = STRING: AXGE-3454-0531
 WR-WRPC-MIB::wrpcSfpDeltaTx.1 = INTEGER: 180750
 WR-WRPC-MIB::wrpcSfpDeltaTx.2 = INTEGER: 180750
 WR-WRPC-MIB::wrpcSfpDeltaRx.1 = INTEGER: 148326
 WR-WRPC-MIB::wrpcSfpDeltaRx.2 = INTEGER: 148326
 WR-WRPC-MIB::wrpcSfpAlpha.1 = INTEGER: 72169888
 WR-WRPC-MIB::wrpcSfpAlpha.2 = INTEGER: -73685416
 End of MIB
\end{lstlisting}
When the SET support is compiled into the \textit{Mini SNMP responder}, it is
possible to erase or add/replace SFP entires to the SFPs database via SNMP.\\

\begin{sloppypar} % to prevent \texttt{} from going to the margine
Addition (or modification) of one SFP to the database can be done by a row of
SNMP SETs. First, please set the delta Tx (\texttt{wrpcPtpConfigDeltaTx.0}), the
delta Rx (\texttt{wrpcPtpConfigDeltaRx.0}) and the alpha (\texttt{wrpcPtpConfigAlpha.0})
with new values. Then, to commit the change to the SFP database, perform the SNMP SET on
the \texttt{wrpcPtpConfigApply.0} with the value \texttt{writeToFlashCurrentSfp}. It will
add/update values for the currently plugged SFP.
\end{sloppypar}

To add or update entries for other SFPs, you shoud set deltas and alpha like
above, set PN of an SFP to the \texttt{wrpcPtpConfigSfpPn.0} and commit
the change by setting \texttt{writeToFlashGivenSfp} to the
\texttt{wrpcPtpConfigApply.0}.\\

It is also possible to update parameters of the currently used SFP without
storing them to the Flash/EEPROM. For that, please set delta Tx, delta Rx and
alpha as described above, then set \texttt{writeToMemoryCurrentSfp} to the
\texttt{wrpcPtpConfigApply.0}. Please remember that these changes are made only
in RAM and will be lost after a power cycle of a board, soft reset of WRPC or
unplug/plug of a fiber/SFP.\\

If a database entry or values in RAM of the currently used SFP are updated, it is
necessary to perform a restart of the PTP daemon
(set \texttt{wrpcPtpConfigRestart.0} with the value \texttt{restartPtp}). Such
restart is necessary because currently PTP does not support on-the-fly changes
of deltas nor alpha. It is expected that this behavior will change in the
future.\\

Each SNMP SET of \texttt{wrpcPtpConfigApply.0} or \texttt{wrpcPtpConfigRestart.0} returns
the status of a performed action. For details please check \texttt{WR-WRPC-MIB}
file.

Commands below add an SFP with PN as "\texttt{NEW-SFP}", delta Tx "\texttt{1111}",
delta Rx "\texttt{2222}" and alpha "\texttt{3333}".
\begin{lstlisting}
 $ snmpset $SNMP_OPT wrpcPtpConfigDeltaTx.0 = 1111
 WR-WRPC-MIB::wrpcPtpConfigDeltaTx.0 = INTEGER: 1111
 $ snmpset $SNMP_OPT wrpcPtpConfigDeltaRx.0 = 2222
 WR-WRPC-MIB::wrpcPtpConfigDeltaRx.0 = INTEGER: 2222
 $ snmpset $SNMP_OPT wrpcPtpConfigAlpha.0 = 3333
 WR-WRPC-MIB::wrpcPtpConfigAlpha.0 = INTEGER: 3333
 $ snmpset $SNMP_OPT wrpcPtpConfigSfpPn.0 = NEW-SFP
 WR-WRPC-MIB::wrpcPtpConfigSfpPn.0 = STRING: "NEW-SFP"
 $ snmpset $SNMP_OPT wrpcPtpConfigApply.0 = writeToFlashGivenSfp
 WR-WRPC-MIB::wrpcPtpConfigApply.0 = INTEGER: applySuccessful(100)
\end{lstlisting}

In case when the SFP database does not contain the currently plugged SFP, the last
\texttt{snmpset} command will return \texttt{applySuccessfulMatchFailed(101)}.

Optionally restart the PTP:
\begin{lstlisting}
 $ snmpset $SNMP_OPT wrpcPtpConfigRestart.0 = restartPtp
 WR-WRPC-MIB::wrpcPtpConfigRestart.0 = INTEGER: restartPtpSuccessful(100)
\end{lstlisting}

Simple verification of performed actions:\\
\begin{minipage}{\textwidth}
\begin{lstlisting}
 wrc# sfp show
 1: PN:AXGE-1254-0531   dTx:   180750 dRx:   148326 alpha: 72169888
 2: PN:AXGE-3454-0531   dTx:   180750 dRx:   148326 alpha: -73685416
 3: PN:NEW-SFP          dTx:     1111 dRx:     2222 alpha:     3333
\end{lstlisting}
\end{minipage}

The same add can also be achieved by performing \texttt{sfp add} command in
the WRPC's console:
\begin{lstlisting}
 wrc# sfp add NEW-SFP 1111 2222 3333
 Update existing SFP entry
 3 SFPs in DB
\end{lstlisting}

Verify the result via SNMP:
\begin{lstlisting}
 $ snmpwalk $SNMP_OPT wrpcSfpTable
 WR-WRPC-MIB::wrpcSfpPn.1 = STRING: AXGE-1254-0531
 WR-WRPC-MIB::wrpcSfpPn.2 = STRING: AXGE-3454-0531
 WR-WRPC-MIB::wrpcSfpPn.3 = STRING: NEW-SFP
 WR-WRPC-MIB::wrpcSfpDeltaTx.1 = INTEGER: 180750
 WR-WRPC-MIB::wrpcSfpDeltaTx.2 = INTEGER: 180750
 WR-WRPC-MIB::wrpcSfpDeltaTx.3 = INTEGER: 1111
 WR-WRPC-MIB::wrpcSfpDeltaRx.1 = INTEGER: 148326
 WR-WRPC-MIB::wrpcSfpDeltaRx.2 = INTEGER: 148326
 WR-WRPC-MIB::wrpcSfpDeltaRx.3 = INTEGER: 2222
 WR-WRPC-MIB::wrpcSfpAlpha.1 = INTEGER: 72169888
 WR-WRPC-MIB::wrpcSfpAlpha.2 = INTEGER: -73685416
 WR-WRPC-MIB::wrpcSfpAlpha.3 = INTEGER: 3333
 End of MIB
\end{lstlisting}

It is also possible to erase the SFPs database via SNMP (equivalent of
the \texttt{sfp erase} command):
\begin{lstlisting}
 $ snmpset $SNMP_OPT wrpcPtpConfigApply.0 = eraseFlash
 WR-WRPC-MIB::wrpcPtpConfigApply.0 = INTEGER: applySuccessful(100)
\end{lstlisting}

To verify that database is empty:
\begin{lstlisting}
 wrc# sfp show
 SFP database empty
\end{lstlisting}

% ==========================================================================
\newpage
\subsection{Syslog}
\label{Syslog}

\begin{sloppypar} % to prevent \texttt{} from going to the margine
The node can act as a \textit{syslog} client, though only on the UDP protocol.
To activate it, you must build with \texttt{CONFIG\_SYSLOG} and pass proper
parameters at run time.

To configure \textit{syslog} you can run the \texttt{syslog} shell command, which
receives two parameters (\texttt{ipaddr} and \texttt{macaddr}), or a
single \texttt{off} subcommand.

When deploying a network of nodes, you can choose to put the \texttt{syslog
<ip> <mac>} command in the build-time init command. To do so, you
must activate \texttt{CONFIG\_BUILD\_INIT} and then pass your command string
as \texttt{CONFIG\_INIT\_COMMAND}.  In that context, you can use ``\texttt{;}'' as
a command separator as no newlines are permitted in \texttt{Kconfig}
strings.

The strings that a WR node sends to the \textit{syslog} server are always
using the format:  ``\texttt{<}\textit{level}\texttt{>} \textit{Jan 01 00:00:00 192.168.1.1 msg}''
where \textit{level} is usually 14 (type ``user'', priority ``info'')
and \textit{msg} is a free-format message strings.
The \textit{syslog} client sends strings to the server in the following
situations:\\

\begin{longtable}{  p{6.5cm}  p{9cm} }

\textit{ Boot time } &

	The node sends ``\texttt{(ma:ca:dd:rr:ee:ss) Node up since} \textit{X}
        \texttt{seconds}'' as soon as the network link is up and the \textit{syslog}
        server is configured with the shell command or init script.
        The message is re-sent, with an updated uptime value, if you change
        the \textit{syslog} server parameters. \\
& \\
\textit{ Link up after link down } &

	The message is ``\texttt{"Link up after} \textit{2.345} \texttt{s}''. The
        time printed is the duration of the link-down interval that has just
        passed -- no lost-by-design message is sent at link-down time. The
        message is not sent the first time the link goes up, because
        the boot message is already there.\\
& \\
\textit{ Synchronization, first time } &

	When the node reaches WR synchronization (i.e. ``track phase''
        state), it sends ``\texttt{Tracking after} \textit{5..678} \texttt{s}''.
        The reported time is the lapse since power-on.\\
& \\
\textit{ Synchronization lost } &

	Whenever WR looses \textit{track-phase} status, the node reports
        \texttt{Lost track}.\\
& \\
\textit{ Synchronization recovered } &

	When the WR servo is in \textit{track phase} state after loosing
        synchronization, the node sends ``\textit{45}\texttt{-th re-track after}
        \textit{23.456} \texttt{s}''. The time reported is the amount of time
        during which the node has not been synchronized since previous synchronization. 
        The seconth and
        thirth re-sync are reported as \texttt{2-th} and \texttt{3-th}, to make you
        smile. At the \texttt{4-th} you should stop smiling and be concerned.\\
& \\
\textit{ Temperature over threshold } &

	The node monitors various thermometers every few seconds.
        If \texttt{CONFIG\_TEMP\_POLL\_INTERVAL} and related parameters are
        set, any over-temperature event is reported to \textit{syslog}.
        If any temperature in the collected set is over threshold,
        the message is ``\texttt{Temperature high:}'' followed by the list of
        all collected temperatures.  The message is repeated every
        few seconds (\texttt{CONFIG\_TEMP\_HIGH\_RAPPEL}, default 60)
        until all temperatures are under-threshold. When temperature
        is recovered the node sends ``\texttt{Temperature ok:}'' followed by
        the current list of temperatures.\\
\end{longtable}
\end{sloppypar}
% --------------------------------------------------------------------------

% FIXME: syslog examples

% ==========================================================================
\newpage
\subsection{Diagnostic Tools}
\label{Diagnostic Tools}
This section describes various diagnostics tools that come with the WR PTP Core.
These tools are foreseen for the hosted environments as they access various WR
PTP Core registers over the External Wishbone interface\footnote{Please check
section \ref{sec:wrpc_hdl} for information about the WR PTP Core interfaces.}
(through PCIe or VME bridge).

% --------------------------------------------------------------------------
\subsubsection{wrpc-diags}
\label{wrpc-diags}

A direct access to synchronization status of the WR PTP Core is possible via WB registers.
Such access can be used by the application-specific logic or by any software running on the 
host machine, provided the PCIe/VME bridge is instantiated in your design and
lets you access the WR PTP Core external Wishbone Slave interface (see
\ref{sec:hdl_wrpc}. Every reference design contains appropriate bridge, so you
can use one of them to access conveniently the Wishbone registers for WRPC
diagnostics using the \texttt{wrpc-diags} tool.

You will find \texttt{wrpc-diags} tool in the \textit{wrpc-sw} repository. If
you have not yet cloned it, please see section \ref{LM32 software compilation}
for more instructions how this should be done, as well as how to compile it.

The tool is self documented in the sense that a help \texttt{h} command lists
all the currently supported commands, and to get detailed information for a
given command, one can type \texttt{h cmd}.

To use the \texttt{wrpc-diags} tool interactively, call it with appropriate
parameters, depending on your hardware type. For example for a SPEC board at
address 01:00.0:
\begin{lstlisting}
$ sudo <your_wrpc-sw_location>/tools/wrpc-diags -o 0x20800 \
   -f /sys/bus/pci/devices/0000:01:00.0/resource0
\end{lstlisting}

The prompt \texttt{wrcdiag[00] >} will appear and you will be able to input commands.
All the available commands are listed with \texttt{h}:
\begin{lstlisting}[basicstyle=\scriptsize\ttfamily]
cfv-774-cbt:wrcdiag[00] > h
Valid COMMANDS:
Idx  Name    Params           Description
# 1: q       [           ] -> Quit test program
# 2: h       [ o c       ] -> Help on commands
# 3: a       [           ] -> Atom list commands
# 4: his     [           ] -> History
# 5: s       [ Seconds   ] -> Sleep seconds
# 6: ms      [ MilliSecs ] -> Sleep milliseconds
# 7: sh      [ Unix Cmd  ] -> Shell command
# 8: diags   [           ] -> show all wrc diags
# 9: sstat   [           ] -> get servo status
#10: ptpstat [           ] -> get PTP state
#11: pstat   [           ] -> get port status
#12: astat   [           ] -> get auxiliare state
#13: txfcnt  [           ] -> get TX PTP frame count
#14: rxfcnt  [           ] -> get RX frame count
#15: ltime   [           ] -> Local Time expressed in sec since epoch
#16: rttime  [           ] -> Round trip time in picoseconds
#17: msdelay [           ] -> Master slave link delay in picoseconds
#18: asym    [           ] -> Total link asymmetry in picoseonds
#19: cko     [           ] -> Clock offset in picoseonds
#20: setp    [           ] -> Current slave's clock phase shift value
#21: ucnt    [           ] -> Update counter
#22: temp    [           ] -> get board temperature

Type "h name" to get complete command help
\end{lstlisting}
In order to see all the wrc diagnostics, the \texttt{diags} command inside the prompt should
be executed as follows:
\begin{lstlisting}
cfc-774-cbt:wrcdiag[00] > diags
servo status:		Track phase
Port status:		Link up, PLL locked, 
PTP state:		PPS slave
Aux state:		ch0:enabled
TX frame count:		1593970
RX frame count:		6883447
TAI time:		Wed May 31 15:31:56 2017
Round trip time:	848002 ps
Master slave delay:	417483 ps
Total Link asymmetry:	13036 ps
Clock offset:		2 ps
Phase setpoint:		7883 ps
Update counter:		812977
temp:			47.3750 C
\end{lstlisting}
Some advanced commands can be also executed. For example, one can call
\texttt{temp} command every 1 second for 5 times:
\begin{lstlisting}
cfc-774-cbt:wrcdiag[00] > 5(temp, s 1)
temp:			47.3750 C
temp:			47.3750 C
temp:			47.3750 C
temp:			47.3750 C
temp:			47.3750 C
\end{lstlisting}

The commands that are available from the interactive prompt can be also executed directly
from the host's shell prompt and via ssh. Both cases can be useful when writing shell 
scripts using the tool. To call one of the commands from the host shell prompt,
use \texttt{echo cmd | wrpc-diags} format. For example, to execute
\texttt{diags} command for SPEC:
\begin{lstlisting}
echo 'diags' | sudo <your_wrpc-sw_location>/tools/wrpc-diags -o 0x20800 \
    -f /sys/bus/pci/devices/0000:01:00.0/resource0
\end{lstlisting}
In order to  call \texttt{temp} command every 1 second for 5 times:
\begin{lstlisting}
echo '5(temp, 1s)' | sudo <your_wrpc-sw_location>/tools/wrpc-diags -o 0x20800 \
    -f /sys/bus/pci/devices/0000:01:00.0/resource0
\end{lstlisting}

% --------------------------------------------------------------------------

\subsubsection{wr-streamers}
\label{sec:wr-streamers-tool}

{\bf Note:} This tool can be used only if you've selected \tts{STREAMERS} mode
for the external WRPC fabric interface or if you have manually instantiated
WR Streamers module in your HDL design. See section \ref{sec:wrpc_hdl} for more
details on WR PTP Core and Board Support Packages interfaces.\\

The WR Streamers module provides information about transmitted and received
frames, which can be useful for diagnostics and debugging. One way of accessing
this information is by using the \texttt{diag} command of the WR PTP Core Shell.
An alternative is to use the \texttt{wr-streamers} tool described in this
section. You will find it in the \textit{wrpc-sw} repository. If
you have not yet cloned it, please see section \ref{LM32 software compilation}
for more instructions how this should be done, as well as how to compile it.\\

The available statistical information is identical to the one that can be
accessed via the \texttt{diag} command of WR PTP Core's Shell. It includes
statistics collected since the most recent reset: max/min latency, number of
transmitted/received frames, number of lost frames/blocks, number of latency
values accumulated, accumulated latency, and the time of the last reset. The
wishbone registers can also be used to override the default network
configuration of the WR Streamers.\\

The \texttt{wr-streamers} tool is self documented in the sense that a help
\texttt{h} command lists you all the currently supported commands, and to get
detailed information for a given command, one can type \texttt{h cmd}. 

To use the \texttt{wr-streamers} tool interactively, call the tool with appropriate
parameters, depending on your hardware type. For example for a SPEC board at
address 01:00.0:
\begin{lstlisting}
$ sudo <your_wrpc-sw_location>/tools/wr-streamers -o 0x20700 \
   -f /sys/bus/pci/devices/0000:01:00.0/resource0
\end{lstlisting}

The prompt \texttt{wrstm[00] >} will appear and you will be able to input commands.
All the available commands are listed with \texttt{h}:
\begin{lstlisting}[basicstyle=\scriptsize\ttfamily]
cfv-774-cbt:wrstm[00] > h
Valid COMMANDS:
Idx  Name       Params                   Description
# 1: q          [                   ] -> Quit test program
# 2: h          [ o c               ] -> Help on commands
# 3: a          [                   ] -> Atom list commands
# 4: his        [                   ] -> History
# 5: s          [ Seconds           ] -> Sleep seconds
# 6: ms         [ MilliSecs         ] -> Sleep milliseconds
# 7: sh         [ Unix Cmd          ] -> Shell command
# 8: stats      [ [0/1/2/3/4/5/6/7] ] -> show streamers statistics
# 9: reset      [                   ] -> show time of the latest reset / time elapsed since then
#10: resetcnt   [                   ] -> reset tx/rx/lost counters and avg/min/max latency values
#11: resetseqid [                   ] -> reset sequence ID of the tx streamer
#12: lat        [ [latency]         ] -> get/set config of fixed latency in integer [us] (-1 to disable)
#13: qtagf      [ [0/1]             ] -> QTags flag on off
#14: qtagvp     [ [VID,prio]        ] -> QTags Get/Set VLAN ID and priority
#15: qtagor     [ [0/1]             ] -> get/set overriding of default qtag config with WB config (set
                                         using qtagf, qtagvp)
#16: ls         [ [leapseconds]     ] -> get/set leap seconds

Type "h name" to get complete command help
\end{lstlisting}
In order to see all the WR Streamers statistics, the \texttt{stats} command
inside the prompt should be executed as follows:
\begin{lstlisting}
cfc-774-cbt:wrstm[01] > stats
Latency [us]    : min=     3.736 max=     8.216 avg =   3.88176 
Frames  [number]: tx =0 rx =61897834620 lost=0 (lost blocks =0)
\end{lstlisting}

The commands that are available from the interactive prompt can be also executed directly
from the host's shell prompt and via ssh. Both cases can be useful when writing shell 
scripts using the tool. To call one of the commands from the host shell prompt, use
\texttt{echo cmd | wr-streamers}. For example, to execute \texttt{stats} command for SPEC:
\begin{lstlisting}
echo 'stats' | sudo <your_wrpc-sw_location>/tools/wr-streamers -o 0x20700 \
    -f /sys/bus/pci/devices/0000:01:00.0/resource0
\end{lstlisting}
In order see the number of received frames every 1 second for 5 times, call:
\begin{lstlisting}
echo '5(stats 1, 1s)' | sudo <your_wrpc-sw_location/tools/wr-streamers -o 0x20700 \
    -f /sys/bus/pci/devices/0000:01:00.0/resource0
\end{lstlisting}

% ==========================================================================
\newpage
\subsection{Other Diagnostic Methods}
\label{Other Diagnostic Tools}

% --------------------------------------------------------------------------
\subsubsection{Latency Test}
\label{Latency Test}

\begin{sloppypar} % to prevent \texttt{} from going to the margine
The configuration choice \texttt{CONFIG\_LATENCY\_PROBE} activates a
mechanism for \textit{wr} nodes to measure network latency, base on the same
hardware timestamps that are used for synchronization -- the mechanism
assumes the nodes are synchronized.

\textit{ltest} frames can be used to verify whether the network is
overloaded and/or has strange inconsistencies in node-to-node delays.

\textit{ltest} frames use a special ethernet type, \texttt{CONFIG\_LATENCY\_ETHTYPE},
which defaults to 0x0123 (291). If \textit{vlans} are active, these frames
are sent and received in the same \textit{vlans} as other CPU frames.

The \texttt{ltest} sender periodically sends three frames, with a sequence
number. One frames is at priority 7, one is at priority 6, and one is
at priority 0. The last frame reports the departure timestamps of the
previous frames.  The \textit{ltest} receiver uses ingress timestamps to
measure latency and report lost frames.

Every node that is built with \texttt{CONFIG\_LATENCY\_PROBE} listens for frames
belonging to \texttt{CONFIG\_LATENCY\_ETHTYPE}.  A single node in the
network is expected to send \textit{ltest} frames; use the \texttt{ltest}
shell command to select how often to send the \textit{ltest} tuple of three
frames. To enable sending every second use ``\texttt{ltest 1}'', to enable
sending every 100ms use ``\texttt{ltest 0 100}'', to stop sending use
\texttt{ltest 0}.

In the sender node, a reminder is sent to the console very 10s
reporting that the node is currently sending \textit{ltest} frames.  All
non-sending nodes report every minute to \textit{syslog}. The
report message includes the number of samples received as well
as the minimum, average and maximum latency, in nanoseconds.
Any lost frames are reported both to the console and to \textit{syslog}.

You can use \textit{ltest} without \texttt{CONFIG\_SYSLOG}. In that case the
receiver nodes print the exact latency (picosecond resolution) for
every received event.
\end{sloppypar}

% --------------------------------------------------------------------------
\subsubsection{wrpc-dump}
\label{wrpc-dump}

When trying to diagnose software issues, especially lockup situations,
it may be useful to look at the current values or critical variables
within \textit{wrpc-sw}/\textit{ppsi}.

To this aim, you can pass a memory image (dump of RAM content) of \textit{wrpc} to 
\texttt{tools/wrpc-dump}.
The tool will print information for softpll, ppsi data structures,
ptp data sets and version information.

For example, for the \textit{spec} board, you can use the resource file in
\textit{sysfs} to look at a live system, or copy the file for off-line
analysis. The following command line show both uses:

\begin{lstlisting}
   # Look for "resource0" in /sys/devices/pci, choose your bus number.
   FILE=/sys/devices/pci0000:00/0000:00:04.0/0000:04:00.0/resource0

   sudo ./tools/wrpc-dump $FILE

   sudo tools/mapper $FILE 0 0x20000 > wrpc-memory-image
   ./tools/wrpc-dump wrpc-memory-image
\end{lstlisting}

The \textit{mapper} tool used above, that is part of \textit{wrpc-sw}, reads a file
using \textit{mmap()}. The kernel doesn't allow plain \textit{read()} from a
resource file. If you encounter problem using \texttt{wrpc\_dump} directly,
please use \texttt{mapper} then \texttt{wrpc\_dump}.

\textbf{Note:} Data read by \texttt{mapper} may look as if it has wrong endianness compared
to the file used for programming LM32 by \texttt{spec-cl}, but \texttt{spec-cl} is the
one which change the endianness of the binary data during the programming.

With \textit{etherbone}, you can get a snapshot for \textit{wrpc} memory using
\texttt{eb-get} and the proper address (the size is always 128kB)

\begin{lstlisting}
   eb-get dev/wbm2 0x4040000/0x20000 wrpc-memory-image
   eb-get dev/ttyUSB2 0x4040000/0x20000 wrpc-memory-image
\end{lstlisting}

We won't show the 180 output lines here, to save some paper, but they
are readable dumps of the data structures, including the PTP timestamps.

% --------------------------------------------------------------------------
\subsubsection{Softpll Timing}
\label{spll Softpll Timing}

To help understanding the CPU time spent in the \textit{softpll}, you can
set \texttt{CONFIG\_SPLL\_FIFO\_LOG} in the configuration. The option
depends on \texttt{CONFIG\_DEVELOPER} and is disabled by default.

When the configuration option is set, \texttt{wrpc-dump} will also show
information about the last 16 \textit{softpll} iterations. Both the \texttt{tstamp}
and \texttt{duration} fields come from reading the \texttt{PPSG} nanosecond counter.

\begin{lstlisting}
   fifo log at 0x17258
        trr:                           0x0126d921
        tstamp:                        0x06b58d6a
        duration:                      0x00000c7e
        irq_count:                     2305
        tag_count:                     2304
        [... repeats for 5 more events ...]
\end{lstlisting}

% --------------------------------------------------------------------------
\subsubsection{Uptime Counter}
\label{Uptime Counter}

\textit{Wrpc-sw} now maintains an ``uptime'' counter, in seconds. It lives
at binary address 0xa0. It can be queried by \textit{etherbone}, or
seen in memory maps.

In addition to knowing how much the node has been up, it can be used to
know roughly when a node got stuck, and whether software is still
running when a node is not active on the network. Neither of this events
happens in production, but the tool is useful during development.

% --------------------------------------------------------------------------
\subsubsection{Profiling}
\label{Profiling}

The node now has a \textit{ps} command, that shows the number of iterations
and time spent in each \textit{task}. Each task reports when it did
something (as opposed to just polling the clock or network socket and
seeing that nothing is there to do); the \textit{iterations} count shows how many
times the task did something. The \textit{max\_ms} show the longest execution
time of a particular task.

\begin{lstlisting}
wrc# ps
 iterations     seconds.micros    max_ms name
     145560          29.007144        75 idle
          0           0.000000         0 spll-bh
          6           0.001630         1 shell+gui
        642           2.540416         8 ptp
         31           0.000361         1 uptime
          1           0.050424        51 check-link
         32           0.011172         9 diags
          0           0.000000         0 stats
        235           0.023562         1 net-bh
         32           0.029157         2 ipv4
          0           0.000000         0 arp
          0           0.000000         0 snmp
          6           0.054566        12 temperature
\end{lstlisting}

By using ``\texttt{ps reset}'' you can zero all counters to start a new
test run.

It is possible to configure \texttt{ps} in such way that it prints information
when any task runs longer than any run before since reset
(or \texttt{ps reset}) and when it runs longer than a specified value
in miliseconds.
For this please use command ``\texttt{ps max <msecs>}'', where
\texttt{<msecs>} is a number of miliseconds triggering printouts.
\begin{lstlisting}
wrc# ps max 10
task temperature, run for 11 ms
task temperature, run for 12 ms
task temperature, run for 11 ms
wrc# ps
[...]
New max run time for a task shell+gui, old 1, new 75
task shell+gui, run for 75 ms
\end{lstlisting}

% --------------------------------------------------------------------------
\subsubsection{Pfilter rules}
\label{Pfilter rules}

By setting \texttt{CONFIG\_PFILTER\_VERBOSE}, which depends on
\texttt{CONFIG\_DEVELOPER}, you can get a dump of packet-filter rules
whenever they are activated.  This happens at initialization time and
whenever you change the MAC address or \textit{vlan} choice.

% ==========================================================================
\newpage
\section{Network Services}
\label{Network Services}

If built with \texttt{CONFIG\_IP=y}, \textit{wrpc-sw} implements the following
udp-based network services, in addition to \textit{ptp}:

\begin{longtable}{  p{6.5cm}  p{9cm} }

\textit{ bootp } &

	The node will get an IPV4 address by making \textit{bootp} queries
        once per second, until it gets a reply from a server. As an
        alternative, the address can be set using the shell command,
        and in this case the node won't send \textit{bootp} queries, or will
        stop sending them.\\
& \\
\textit{ rdate } &

	The node can be queried using \texttt{rdate -u}, once it has an IP
        address.\\
& \\
\textit{ syslog } &

	If \texttt{CONFIG\_SYSLOG} is set, the node is a syslog client.
        See section \ref{Syslog} for details.\\
& \\
\textit{ snmp } &

	If \texttt{CONFIG\_SNMP} is set, the node is a snmp agent.
        See section \ref{Diagnostics via SNMP} for details.\\

\end{longtable}

% ==========================================================================
\newpage
\section{VLAN Support}
\label{VLAN Support}

If you are using the White Rabbit PTP Core version 4.1 or later, it comes by
default with VLAN support enabled. You can always disable them in run time using
a WRPC shell command:
\begin{lstlisting}
wrc# vlan off
\end{lstlisting}

Current implementation allows to configure up to 4 VLAN IDs (VIDs) at build
time. The following Kconfig options are available in the LM32 software:
\begin{longtable}{  p{6.5cm}  p{9cm} }

\small{\textit{CONFIG\_VLAN}} &

	This is the top-level choice. It can be enabled or disabled.
        If VLANs are disabled, all incoming tagged frames will be discarded.\\
& \\
\small{\textit{CONFIG\_VLAN\_NR}} &

  The default \textit{vlan} number for the CPU (class 0).  All network traffic
        directed to (and originating from) the LM32 processor will
        belong to this VLAN.  The~value can be changed at run time
        using the \texttt{vlan set} shell command.\\
& \\
\small{\textit{CONFIG\_VLAN\_1\_FOR\_CLASS7}} & \\
\small{\textit{CONFIG\_VLAN\_2\_FOR\_CLASS7}} &

	The packet-filter rules are setup to deliver frames belonging
        to two specific VLANs to class 7, usually Etherbone.
        To deliver one VLAN only, set the two options to the same
        number. These VIDs can be modified only at build time.\\
& \\
\small{\textit{CONFIG\_VLAN\_FOR\_CLASS6}} &

	The packet-filter rules are setup to deliver frames belonging
        to one specific VLAN to class 6, usually routed to the
        Streamer module. This VID can be modified only at build time.\\

\end{longtable}

Frame classes are assigned inside the WRPC and are used to distinguish received
frames between these that should be processed by the WR PTP Core (e.g. PTP
frames, SNMP requests) and frames that should be passed to the external WR
Fabric interface. Currently, class 0 is used for all frames that should be
processed by the WRPC, class 6 is used for Streamers traffic, class 7 is used
for Etherbone traffic (see HDL documentation for boards HDL modules and
selection between Streamers, Etherbone and Plain modes).

The default WR PTP Core v4.1 release firmware comes with the following VIDs:
\begin{itemize}
  \item \textbf{VID 1} - for class 0, traffic processed inside the WR PTP Core
  \item \textbf{VID 10} and \textbf{VID 11} - for class 7, Etherbone traffic
  \item \textbf{VID 20} - for class 6, Streamers traffic
\end{itemize}
This of course means that all the traffic from VIDs 10, 11, 20 is forwarded and
available on the WR PTP Core external fabric interface. The user is free to use
it for any other HDL module, not necessary Etherbone or Streamers.

Currently, only VID for the traffic processed by the WRPC (class 0) can be
modified in the run time. You can use the WRPC shell command
\texttt{vlan set <vid>} to modify it. If, for example, your WR node is supposed
to synchronize and respond to SNMP requests in VLAN 5, you should add this to
the user-defined init script:
\begin{lstlisting}
wrc# init add vlan set 5
\end{lstlisting}

% ##########################################################################
\newpage

\section{Introduction}

This document provides information about the diagnostics of White Rabbit
PTP Core (WRPC) - an HDL module present in every White Rabbit node. It is a
complementary documentation to the official \emph{White Rabbit PTP Core User's
Manual} published with every stable release. Please refer to this user manual
for the information about the WRPC, its interfaces and building instructions for
the official reference designs.\\

White Rabbit PTP Core starting from \emph{v4.0} provides diagnostic mechanisms
in the form of SNMP objects and optional Syslog messages (depending on the build
time LM32 software configuration). The implementation of the SNMP agent in the
WRPC is very basic comparing to the diagnostics offered by the White Rabbit
Switch. Since we are very constraint on the code size running inside the WR PTP
Core, almost all of the logic to detect and report errors has to be implemented
on the SNMP Manager's side.

%This document is
%organized in two parts. It starts with a description of the SNMP objects and
%procedures to be followed if various errors are reported (section
%\ref{sec:snmp_exports}). This first part is meant for the operators and people
%integrating a WR switch into a control system, without the deep knowledge about
%the White Rabbit internals. These people usually have to perform a quick
%diagnostics and decide on actions to restore a WR network.
%Second part of the document tries to list all the possible failures
%that may disturb synchronization and Ethernet switching (section
%\ref{sec:failures}). It is meant for the WR experts to help them with in-depth
%diagnosis of the problems reported by SNMP.

This document has many internal hyperlinks that associate general SNMP status
objects and expert SNMP objects with related problems' description and the other
way round. These links can be easily used when reading the document on a
computer.

\section{WR Core}
\label{sec:hdl_wrpc}
This section describes the input and output ports of the WRPC IP-core and VHDL generic parameters
that can be used to personalize the core.

The top-level VHDL module is located under:\\\hrefwrpc{modules/wrc\_core/wr\_core.vhd}

A wrapper for the top-level VHDL module which makes use of VHDL records to reduce the number of
ports can be found under:\\\hrefwrpc{modules/wrc\_core/xwr\_core.vhd}

\begin{figure}
  \begin{center}
    \includegraphics[width=.9\textheight, angle=270]{fig/basic_top.pdf}
    \label{intro:fig:wrpc_top}
    \caption{Simple top design with WRPC}
  \end{center}
\end{figure}

Figure \ref{intro:fig:wrpc_top} is an example on how to instantiate the WRPC component inside a
Xilinx Spartan6-based project. It contains few additional modules besides the WRPC:
\begin{itemize}
  \item \emph{wr\_gtp\_phy\_spartan6}: module wrapping Xilinx GTP SerDes to improve its determinism
  \item \emph{PLL\_BASE}: Xilinx Spartan6 PLL primitive \cite{pll_base}, used to produce 62.5 MHz
    system clock from 125 MHz local reference clock and to produce the DMTD offset clock from a
    local 20 MHz oscillator
  \item \emph{spec\_serial\_dac\_arb}: converts DACs tuning values to serial interface and
    arbitrates access to two DACs used for reference and DMTD clock tuning.
\end{itemize}

A very similar example can be found in the WRPC reference design for PCI-Express SPEC board (see
Section~\ref{sec:hdl_board_spec}).


\subsubsection{Generic parameters}
\label{sec:wrc_generics}

\begin{hdlparamtable}
  g\_simulation & integer & 0 & setting to '1' speeds up the simulation,
  must be set to '0' for synthesis\\
  \hline
  g\_with\_external\_clock\_input & boolean & false &
  enable external clock and 1-PPS inputs. The PLL inside WRPC will lock to
  external 10 MHz and 1-PPS signal when operating in GrandMaster mode\\
  \hline
  g\_board\_name & string & "NA  " & board name, exported by WRPC software as
  SNMP object for diagnostics\\
  \hline
  g\_flash\_secsz\_kb & integer & 256 & Flash memory sector size in kilobytes.
  Available through a Wishbone register, used by WRPC software to read/write
  SDBFS image\\
  \hline
  g\_flash\_sdbfs\_baddr & integer & 0x600000 & Default base address in Flash
  memory where \code{sdb fs} command should store an empty SDBFS image\\
  \hline
  g\_phys\_uart & boolean & true & enable physical UART interface\\
  \hline
  g\_virtual\_uart & boolean & false & enable virtual UART interface\\
  \hline
  g\_aux\_clks & integer & 0 & number of aux clocks syntonized by WRPC to WR timebase\\
  \hline
  g\_rx\_buffer\_size & integer & 1024 & size of Rx buffer in WRPC MAC module,
  default value is 1024 and should not be changed\\
  \hline
  g\_tx\_runt\_padding & boolean & true & when set to true, all user frames
  transmitted from the external fabric interface are padded if shorter than
  minimal Ethernet frame size (60B with header)\\
  \hline
  g\_dpram\_initf & string & "" & filename of compiled WRPC software, to be
  stored in WRPC memory during the synthesis (default is \emph{wrc.ram}
  created by compiling WRPC software from \emph{wrpc-sw} git repository)\\
  \hline
  g\_dpram\_size & integer & 32768 & size of RAM used by WRPC software (in 32-bit
  words), default value is 22528 and should not be changed\\
  \hline
  g\_interface\_mode & enum& PIPELINED & external Wishbone Slave interface mode
  \tts{[PIPELINED/CLASSIC]}\\
  \hline
  g\_address\_granularity & enum & BYTE & granularity of address bus in external
  Wishbone Slave interface \tts{[BYTE/WORD]}\\
  \hline
  g\_aux\_sdb & rec & c\_wrc\_periph3\_sdb & structure providing an SDB descriptor
  for the peripheral attached to the WRPC auxiliary WB interface. This parameter is optional
  and can be left unassigned. The default value corresponds to an undocumented device with an
  address space of 256 bytes\\
  \hline
  g\_softpll\_enable\_debugger & boolean & false & when set to true, additional
  FIFO is instantiated in the SoftPLL for collecting DMTD tags. It can be read
  out by the host and analyzed for SoftPLL debugging.\\
  \hline
  g\_vuart\_fifo\_size & integer & 1024 & size (in bytes) for the virtual UART FIFO\\
  \hline
  g\_pcs\_16bit & boolean & false & when set to \tts{true}, make use of 16-bit PCS, otherwise use 8-bit PCS\\
  \hline
  g\_records\_for\_phy & boolean & false & when set to \tts{true}, all the PHY-related
  signals will be grouped in the \tts{phy8/phy16} VHDL records, otherwise the individual standard
  logic signals will be used\\
  \hline
  g\_diag\_id  & integer & 0 & auxiliary diagnostics module ID\\
  \hline
  g\_diag\_ver & integer  & 0 & auxiliary diagnostics version for a given module ID\\
  \hline
  g\_diag\_ro\_size & integer & 0 & number of read-only registers fed to auxiliary diagnostics\\
  \hline
  g\_diag\_rw\_size & integer & 0 & number of read-write registers fed to
  auxiliary diagnostics\\  
\end{hdlparamtable}

\subsubsection{Ports}
\label{sec:wrc_ports}

\begin{hdlporttable}
  \hdltablesection{Clocks and resets}\\
  \hline
  clk\_sys\_i & in & 1 & main system clock, can be any frequency $\leq f_{clk\_ref\_i}$
  e.g. 62.5~MHz\\
  \hline
  clk\_dmtd\_i & in & 1 & DMTD offset clock (close to 62.5 MHz, e.g. 62.49 MHz)\\
  \hline
  clk\_ref\_i & in & 1 & 125 MHz reference clock\\
  \hline
  clk\_aux\_i & in & var & [optional] vector of auxiliary
  clocks that will be disciplined to WR timebase. Size is equal to \tts{g\_aux\_clks}\\
  \hline
  clk\_ext\_mul\_i & in & 1 & 125 MHz clock, derived from \tts{clk\_ext\_i}\\
  \hline
  clk\_ext\_mul\_locked\_i & in & 1 & PLL locked indicator for \tts{clk\_ext\_mul\_i}\\
  \hline
  clk\_ext\_stopped\_i & in & 1 & PLL stopped indicator for \tts{clk\_ext\_mul\_i}\\
  \hline
  clk\_ext\_rst\_o & out & 1 & Reset output to be used for \tts{clk\_ext\_mul\_i}\\
  \hline  
  clk\_ext\_i & in & 1 & [optional] external 10 MHz reference clock input for
  GrandMaster mode\\
  \hline
  pps\_ext\_i & in & 1 & [optional] external 1-PPS input used in GrandMaster mode\\
  \hline
  rst\_n\_i & in & 1 & main reset input, active-low (hold for at least 5
  \tts{clk\_sys\_i} cycles)\\
  \hline\pagebreak
  \hdltablesection{Timing system}\\
  \hline
  dac\_hpll\_load\_p1\_o & out & 1 & validates DAC value on data port \\
  \hline
  dac\_hpll\_data\_o & out & 16 & DAC value for tuning helper (DMTD) VCXO\\
  \hline
  dac\_dpll\_load\_p1\_o & out & 1 & validates DAC value on data port \\
  \hline
  dac\_dpll\_data\_o & out & 16 & DAC value for tuning main (ref) VCXO\\
  \hline
  \hdltablesection{PHY inteface (when \tts{g\_records\_for\_phy = false})}\\
  \hline
  phy\_ref\_clk\_i & in & 1 & TX clock\\
  \hline
  phy\_tx\_data\_o & out & var & TX data. If \tts{g\_pcs\_16bit = true}, then \tts{size = 16}, else \tts{size=8}\\
  \hline
  phy\_tx\_k\_o & out & var & \tts{1} when \tts{phy\_tx\_data\_o} contains a control code, \tts{0} when it's a data byte. If \tts{g\_pcs\_16bit = true}, then \tts{size = 2}, else \tts{size=1}\\
  \hline
  phy\_tx\_disparity\_i & in  & 1 & disparity of the currently transmitted 8b10b code (\tts{1} for positive, \tts{0} for negative)\\
  \hline
  phy\_tx\_enc\_err\_i & in  & 1 & TX encoding error indication\\
  \hline
  phy\_rx\_data\_i & in & var & RX data. If \tts{g\_pcs\_16bit = true}, then \tts{size = 16}, else \tts{size=8}\\
  \hline
  phy\_rx\_rbclk\_i & in & 1 & RX recovered clock\\
  \hline
  phy\_rx\_k\_i & in & var & \tts{1} when \tts{phy\_rx\_data\_i} contains a control code, \tts{0} when it's a data byte. If \tts{g\_pcs\_16bit = true}, then \tts{size = 2}, else \tts{size=1}\\
  \hline
  phy\_rx\_enc\_err\_i & in & 1 & RX encoding error indication\\
  \hline
  phy\_rx\_bitslide\_i & in & var & RX bitslide indication. If \tts{g\_pcs\_16bit = true}, then \tts{size = 5}, else \tts{size=4}\\
  \hline
  phy\_rst\_o & out & 1 & PHY reset, active high\\
  \hline
  phy\_rdy\_i & in & 1 & PHY is ready: locked and aligned\\
  \hline
  phy\_loopen\_o & out & 1 & \multirowpar{2}{local loopback enable (TX$\rightarrow$RX), active high}\\
  \cline{1-3}
  phy\_loopen\_vec\_o & out & 3 &\\
  \hline
  phy\_tx\_prbs\_sel\_o & out & 3 & PRBS select (see Xilinx UG386 Table 3-15; "000" = Standard operation, pattern generator off)\\
  \hline
  phy\_sfp\_tx\_fault\_i & in & 1 & SFP TX fault indicator\\
  \hline
  phy\_sfp\_los\_i & in & 1 & SFP Loss Of Signal indicator\\
  \hline
  phy\_sfp\_tx\_disable\_o & out & 1 & SFP TX disable control\\
  \hline
  \hdltablesection{PHY inteface (when \tts{g\_records\_for\_phy = true})}\\
  \hline
  phy8\_o & out & rec & \multirowpar{2}{input/output records for PHY signals
    when \tts{g\_pcs\_16bit = false}}\\
  \cline{1-3}
  phy8\_i & in & rec & \\
  \hline
  phy16\_o & out & rec & \multirowpar{2}{input/output records for PHY signals
    when \tts{g\_pcs\_16bit = true}}\\
  \cline{1-3}
  phy16\_i & in & rec & \\
  \hline\pagebreak
  \hdltablesection{GPIO}\\
  \hline
  led\_act\_o & out & 1 & signal for driving Ethernet activity LED\\
  \hline
  led\_link\_o & out & 1 & signal for driving Ethernet link LED\\
  \hline
  sda\_i & in  & 1 & \multirowpar{4}{I2C interface for EEPROM memory storing calibration}\\
  \cline{1-3}
  sda\_o & out & 1 & \\
  \cline{1-3}
  scl\_i & in  & 1 & \\
  \cline{1-3}
  scl\_o & out & 1 & \\
  \hline
  sfp\_sda\_i & in  & 1 & \multirowpar{4}{I2C interface for EEPROM inside SFP module}\\
  \cline{1-3}
  sfp\_sda\_o & out & 1 & \\
  \cline{1-3}
  sfp\_scl\_i & in  & 1 & \\
  \cline{1-3}
  sfp\_scl\_o & out & 1 & \\
  \hline
  sfp\_det\_i & in & 1 & SFP presence indicator\\
  \hline
  btn1\_i & in & 1 & \multirowpar{2}{two microswitch inputs, active low, currently not
    used in official WRPC software}\\
  \cline{1-3}
  btn2\_i & in & 1 & \\
  \hline
  spi\_sclk\_o & out & 1 & Flash SPI SCLK\\
  \hline
  spi\_ncs\_o  & out & 1 & Flash SPI $\overline{\mbox{SS}}$\\
  \hline
  spi\_mosi\_o & out & 1 & Flash SPI MOSI\\
  \hline
  spi\_miso\_i & in  & 1 & Flash SPI MISO\\
  \hline
  \hdltablesection{UART}\\
  \hline
  uart\_rxd\_i & in  & 1 & \multirowpar{2}{[optional] serial UART interface for
    interaction with WRPC software}\\
  \cline{1-3}
  uart\_txd\_o & out & 1 & \\
  \hline
  \hdltablesection{OneWire}\\
  \hline
  owr\_pwren\_o & out & 1 & \multirowpar{3}{[optional] 1-Wire interface used to read the
    temperature of hardware board from digital thermometer (e.g. Dallas DS18B20)}\\
  \cline{1-3}
  owr\_en\_o & out & 1 & \\
  \cline{1-3}
  owr\_i & in & 1 & \\
  \hline
  \hdltablesection{External WB interface}\\
  \hline
  wb\_adr\_i   & in & 32 & \multirowpar{11}{Wishbone slave interface that operates in
    Pipelined or Classic mode (selected with \tts{g\_interface\_mode}), with the address
    bus granularity controlled with \tts{g\_address\_granularity}}\\
  \cline{1-3}
  wb\_dat\_i   & in & 32 &\\
  \cline{1-3}
  wb\_dat\_o   & out & 32 &\\
  \cline{1-3}
  wb\_sel\_i   & in & 4 & \\
  \cline{1-3}
  wb\_we\_i    & in & 1 & \\
  \cline{1-3}
  wb\_cyc\_i   & in & 1 & \\
  \cline{1-3}
  wb\_stb\_i   & in & 1 & \\
  \cline{1-3}
  wb\_ack\_o   & out & 1 & \\
  \cline{1-3}
  wb\_err\_o   & out & 1 & \\
  \cline{1-3}
  wb\_rty\_o   & out & 1 & \\
  \cline{1-3}
  wb\_stall\_o & out & 1 & \\
  \hline
  wb\_slave\_o & out & rec & \multirowpar{2}{Alternative record-based ports
    for the WB slave interface (available in \tts{xwr\_core.vhd})}\\
  \cline{1-3}
  wb\_slave\_i & in & rec & \\
  \hline\pagebreak
  \hdltablesection{Auxiliary WB master}\\
  \hline
  aux\_adr\_i   & in & 32 & \multirowpar{11}{Auxilirary Wishbone pipelined
    master interface}\\
  \cline{1-3}
  aux\_dat\_o   & out & 32 &\\
  \cline{1-3}
  aux\_dat\_i   & in  & 32 &\\
  \cline{1-3}
  aux\_sel\_o   & out & 4 & \\
  \cline{1-3}
  aux\_we\_o    & out & 1 & \\
  \cline{1-3}
  aux\_cyc\_o   & out & 1 & \\
  \cline{1-3}
  aux\_stb\_o   & out & 1 & \\
  \cline{1-3}
  aux\_ack\_i   & in  & 1 & \\
  \cline{1-3}
  aux\_stall\_i & in  & 1 & \\
  \hline
  aux\_master\_o & out & rec & \multirowpar{2}{Alternative record-based
    ports for the aux WB master interface (available in \tts{xwr\_core.vhd})}\\
  \cline{1-3}
  aux\_master\_i & in & rec & \\
  \hline
  \hdltablesection{External fabric interface}\\
  \hline
  ext\_snk\_adr\_i & in & 2 & \multirowpar{9}{External fabric Wishbone
    pipelined interface, direction Sink$\rightarrow$Source}\\
  \cline{1-3}
  ext\_snk\_dat\_i & in & 16 & \\
  \cline{1-3}
  ext\_snk\_sel\_i & in & 2 & \\
  \cline{1-3}
  ext\_snk\_cyc\_i & in & 1 & \\
  \cline{1-3}
  ext\_snk\_stb\_i & in & 1 & \\
  \cline{1-3}
  ext\_snk\_we\_i  & in & 1 & \\
  \cline{1-3}
  ext\_snk\_ack\_o & out & 1 & \\
  \cline{1-3}
  ext\_snk\_err\_o & out & 1 & \\
  \cline{1-3}
  ext\_snk\_stall\_o & out & 1 & \\
  \hline
  ext\_src\_adr\_o & out & 2 & \multirowpar{9}{External fabric Wishbone
    pipelined interface, direction Source$\rightarrow$Sink}\\
  \cline{1-3}
  ext\_src\_dat\_o & out & 16 & \\
  \cline{1-3}
  ext\_src\_sel\_o & out & 2 & \\
  \cline{1-3}
  ext\_src\_cyc\_o & out & 1 & \\
  \cline{1-3}
  ext\_src\_stb\_o & out & 1 & \\
  \cline{1-3}
  ext\_src\_we\_o  & out & 1 & \\
  \cline{1-3}
  ext\_src\_ack\_i & in & 1 & \\
  \cline{1-3}
  ext\_src\_err\_i & in & 1 & \\
  \cline{1-3}
  ext\_src\_stall\_i & in & 1 & \\
  \hline
  wrf\_src\_o & out & rec & \multirowpar{4}{Alternative record-based
    ports for the fabric interface (available in \tts{xwr\_core.vhd})}\\
  \cline{1-3}
  wrf\_src\_i & in &  rec & \\
  \cline{1-3}
  wrf\_snk\_o & out & rec & \\
  \cline{1-3}
  wrf\_snk\_i & in &  rec & \\
  \hline\pagebreak
  \hdltablesection{External TX timestamp interface}\\
  \hline
  txtsu\_port\_id\_o & out & 5 & physical port ID from which the timestamp
  was originated. WRPC has only one physical port, so this value is always
  \tts{0}.\\
  \hline
  txtsu\_frame\_id\_o & out & 16 & frame ID for which the timestamp is
  available\\
  \hline
  txtsu\_ts\_value\_o & out & 32 & Tx timestamp value\\
  \hline
  txtsu\_ts\_incorrect\_o & out & 1 & Tx timestamp is not reliable since it
  was generated while PPS generator inside WRPC was being adjusted\\
  \hline
  txtsu\_stb\_o & out & 1 & strobe signal that validates the rest of signals
  described above\\
  \hline
  timestamps\_o & out & rec & Alternative record-based output ports for
  the TX timestamp interface (available in \tts{xwr\_core.vhd})\\
  \hline
  txtsu\_ack\_i & in & 1 & acknowledge, indicating that user-defined module
  has received the timestamp\\
  \hline
  \hdltablesection{Pause frame control}\\
  \hline
  fc\_tx\_pause\_req\_i   & in  &  1 & Ethernet flow control, request sending
  Pause frame\\
  \hline
  fc\_tx\_pause\_delay\_i & in  & 16 & Pause quanta\\
  \hline
  fc\_tx\_pause\_ready\_o & out &  1 & Pause acknowledge - active after the
  current pause send request has been completed\\
  \hline
  \hdltablesection{Timecode/Servo control}\\
  \hline
  tm\_link\_up\_o & out & 1 & state of Ethernet link (up/down), \tts{1}
  means Ethernet link is up\\
  \hline
  tm\_dac\_value\_o & out & 24 & DAC value for tuning auxiliary clock
  (\tts{clk\_aux\_i})\\
  \hline
  tm\_dac\_wr\_o & out & var & validates auxiliary DAC value. Size is equal
  to \tts{g\_aux\_clks}\\
  \hline
  tm\_clk\_aux\_lock\_en\_i & in & var & enable locking auxiliary clock to
  internal WR clock. Size is equal to \tts{g\_aux\_clks}\\
  \hline
  tm\_clk\_aux\_locked\_o & out & var & auxiliary clock locked to internal WR
  clock. Size is equal to \tts{g\_aux\_clks}\\
  \hline
  tm\_time\_valid\_o & out & 1 & if \tts{1}, the timecode generated by the
  WRPC is valid\\
  \hline
  tm\_tai\_o & out & 40 & TAI part of the timecode (full seconds)\\
  \hline
  tm\_cycles\_o & out & 28 & fractional part of each second represented by
  the state of counter clocked with the frequency 125 MHz (values from 0 to
  124999999, each count is 8 ns)\\
  \hline
  pps\_p\_o & out & 1 & 1-PPS signal generated in \tts{clk\_ref\_i} clock
  domain and aligned to WR time, pulse generated when the cycle counter is 0
  (beginning of each full TAI second)\\
  \hline
  pps\_led\_o & out & 1 & 1-PPS signal with extended pulse width to drive a LED\\
  \hline
  rst\_aux\_n\_o & out & 1 & Auxiliary reset output, active low\\  
  \hline
  link\_ok\_o & out & 1 & Link status indicator\\
  \hline
  \hdltablesection{Auxiliary diagnostics to/from external modules}\\
  \hline
  \linebreak aux\_diag\_i\linebreak & in & var & \multirowpar{2}{Arrays of
    32-bit vectors, to be accessed from WRPC via SNMP or uart console. Input array
    contains \tts{g\_diag\_ro\_size} elements, while output array contains
    \tts{g\_diag\_rw\_size} elements}\\
  \cline{1-3}
  \linebreak aux\_diag\_o\linebreak & out & var & \\
\end{hdlporttable}

\subsection{PHY interface}

\begin{figure}[ht]
  \begin{center}
    \includegraphics[width=.7\textwidth]{fig/wrpc_phyif.pdf}
    \caption{PHY interface of WRPC}
  \end{center}
\end{figure}

The interface connects WRPC with the Ethernet PHY layer IP-core. The interface is
generic, but currently two Gigabit Ethernet PHYs are tested and supported: Xilinx
8-bit GTP and 16-bit GTX SerDes. The signals' naming convention is the same as
in the GTP/GTX component definition.\\

{\bf Important !} If a WRPC user wants to use one of the supported PHYs (GTP,
GTX), they have to be taken from the White Rabbit HDL package instead of generating
them with the Xilinx Coregen tool. That is because WR developers have attached
additional logic to Xilinx GTP/GTX to improve its determinism.\\

\newpage
\subsection{GPIO/UART/I2C/1-Wire interfaces}

\begin{figure}[ht]
  \begin{center}
    \includegraphics[width=.9\textwidth]{fig/basic_wrpc_gpio.pdf}
    \caption{Other interfaces of WRPC}
  \end{center}
\end{figure}

Other hardware peripherals can be connected to the White Rabbit PTP
Core. It has a UART, 1-Wire and two $I^2C$ interfaces implemented
inside. The $I^2C$ connection to the SFP module
is used to read its Part ID, while the external EEPROM stores calibration
values for each supported SFP together with an initialization script. That script
is executed every time the WRPC is powered on and can contain instructions to
automatically match the SFP's Part ID with the EEPROM content and load appropriate
calibration values. The SFP presence indicator is mandatory and has to be connected.
Otherwise, the WRPC won't be able to operate properly (without knowing whether
the SFP transceiver is actually inserted).

Furthermore, a 1-Wire digital thermometer provides
on-board temperature, but also its unique ID is used to calculate default MAC
address of the physical Ethernet interface. The UART interface provides a user shell
that can be used to interact with the White Rabbit PTP Core. More detailed
description of the WRPC shell can be found in the \emph{White Rabbit PTP Core User's
Manual - Building and Running} \cite{wrpc_man}.

\begin{center}
  \begin{tabular}{|p{3cm}|l|p{11cm}|}
    \hline
    {\bf Signal name} & {\bf size} & {\bf description} \\
    \hline \hline
    \texttt{pps\_ext\_i} & 1 & [optional] external 1-PPS input used in
    GrandMaster mode\\
    \hline
    \texttt{scl\_i} \linebreak \texttt{scl\_o} \linebreak \texttt{sda\_i} \linebreak
    \texttt{sda\_o} & 4 & I2C interface for EEPROM memory storing calibration
    parameters and WRPC init script\\
    \hline
    \texttt{sfp\_scl\_i} \linebreak \texttt{sfp\_scl\_o} \linebreak
    \texttt{sfp\_sda\_i} \linebreak \texttt{sfp\_sda\_o} & 4 & I2C interface for
    EEPROM inside SFP module\\
    \hline
    \texttt{sfp\_det\_i} & 1 & SFP presence indicator\\
    \hline
    \texttt{uart\_rxd\_i} \linebreak \texttt{uart\_txd\_o} & 2 & [optional] serial
    UART interface for interaction with WRPC software\\
    \hline
    \texttt{owr\_pwren\_o} \linebreak \texttt{owr\_en\_o} \linebreak \texttt{owr\_i} &
    3 & [optional] 1-Wire interface used to read the temperature of hardware board from
    digital thermometer (e.g. Dallas DS18B20)\\
    \hline
    \texttt{pps\_led\_o} & 1 & 1-PPS signal with extended pulse width to drive a
    LED\\
    \hline
    \texttt{btn1\_i} \linebreak \texttt{btn2\_i} & 2 & two microswitch inputs,
    active low, currently not used in official WRPC software\\
    \hline
    \texttt{led\_link\_o} & 1 & signal for driving Ethernet Link LED\\
    \hline
    \texttt{led\_act\_o} & 1 & signal for driving Ethernet Act LED\\
    \hline
  \end{tabular}
\end{center}

\subsection{External Wishbone Slave/Master interface}

\begin{figure}[ht]
  \begin{center}
    \includegraphics[width=.7\textwidth]{fig/wrpc_wb.pdf}
    \caption{External Wishbone interfaces of WRPC}
  \end{center}
\end{figure}

{\bf Aux WB Master} is a Pipelined Wishbone Master interface. It is connected
through the Wishbone Crossbar inside the White Rabbit PTP Core to the LM32 soft-core
processor (instantiated inside the WRPC). It can optionally be used to control
any user-defined module having a Pipelined Wishbone Slave interface. In that case, the WRPC software
has to be modified to control additional modules connected to the \emph{Aux WB
Master} interface.\\

{\bf Ext WB Slave} is a Wishbone Slave interface that operates in Pipelined or
Classic mode (selected with \emph{g\_interface\_mode} generic), with the address bus
granularity set with \linebreak \emph{g\_address\_granularity} generic. It gives the access
to control all the WRPC internals. In the WRPC reference design it is connected to
the Gennum GN4124 IP-core and used to upload WRPC software to its internal memory.\\

HDL modules accessible through \emph{Ext WB Slave} interface:
\begin{center}
  \begin{tabular}{|l|l|}
    \hline {\bf module name} & {\bf offset (bytes)}\\
    \hline
    WRPC internal memory & 0x00000\\
                Mini NIC & 0x20000\\
                Endpoint & 0x20100\\
                Soft PLL & 0x20200\\
           PPS generator & 0x20300\\
                  Syscon & 0x20400\\
                    UART & 0x20500\\
           1-Wire Master & 0x20600\\
           Aux WB Master & 0x20700\\
    \hline
  \end{tabular}
\end{center}

\subsubsection{Fabric interface}
\label{sec:wrpc_fabric}

The Fabric interface is used for sending and receiving Ethernet frames. It consists 
of two pipelined Wishbone interfaces operating independently: 

\begin{itemize}
  \item \emph{WRF Source}: pipelined Wishbone Master, passes all the Ethernet frames
    received from a physical link to WRF Sink interface implemented in a
    user-defined module.
  \item \emph{WRF Sink}: pipelined Wishbone Slave, receives Ethernet frames from
    the WRF Source implemented in the user-defined module, and sends them to a
    physical link.
\end{itemize}

{\bf Address bus} can have one of the following values:

\begin{center}
\begin{tabular}{|c|l|}
  \hline {\bf decimal value} & {\bf meaning of data word on data bus}\\
  \hline
  \emph{0} & regular data (packet header and payload)\\
  \emph{1} & OOB (Out-of-band) data\\
  \emph{2} & status word\\
  \emph{3} & currently not used\\
  \hline
\end{tabular}
\end{center}

{\bf Status word} (sent when the value of address bus is \emph{2}) contains
various information about Ethernet frame's structure and type:
%\begin{figure}[ht]
  \begin{center}
    \includegraphics[width=.6\textwidth]{fig/status.pdf}
    %\caption{Status word format}
  \end{center}
%\end{figure}

\begin{itemize}
  \item[] \emph{isHP} - if \emph{1}, the frame is high priority
  \item[] \emph{err} - if \emph{1}, the frame contains an error
  \item[] \emph{vSMAC} - the frame contains a source MAC address (otherwise
    it will be assigned from WRPC configuration)
  \item[] \emph{vCRC} - the frame contains a valid CRC checksum
  \item[] \emph{packet class} - the packet class assigned by the classifier
    inside WRPC MAC module
\end{itemize}

OOB data is used for passing the timestamp-related information for the incoming and 
outgoing Ethernet frames. Each frame received from a physical link is
timestamped inside the WRPC and this value is passed as Rx OOB
data. On the other hand, for each transmitted frame the Tx timestamp can be read
from the Tx Timestamping Interface (section \ref{sec:txts}) together with a unique
frame number assigned in Tx OOB. Therefore, the format of OOB differs between Rx
and Tx frames.\\

{\bf Tx OOB format}:

%\begin{figure}[ht]
  \begin{center}
    \includegraphics[width=.7\textwidth]{fig/oob_tx.pdf}
    %\caption{Tx OOB data format}
    %\label{fig:fabric_adv:tx_oob}
  \end{center}
%\end{figure}

\begin{itemize}
  \item[] \emph{OOB type}: "0001" means Tx OOB
  \item[] \emph{frame ID}: ID of the frame being sent. It is later output
    through the \emph{Tx Timestamping interface} to associate Tx timestamp with
    appropriate frame. Frame ID = 0 is reserved for PTP packets inside WRPC
    and cannot be used by user-defined modules.
\end{itemize}

{\bf Rx OOB format}:
%\begin{figure}[ht]
  \begin{center}
    \includegraphics[width=.7\textwidth]{fig/oob_rx.pdf}
    %\caption{Rx OOB data format}
    %\label{fig:fabric_adv:rx_oob}
  \end{center}
%\end{figure}

\begin{itemize}
  \item[] \emph{OOB type}: "0000" means Rx OOB
  \item[] \emph{Tiv}: timestamp invalid. When this bit is set to '1', the PPS
    generator inside WRPC is being adjusted which means the Rx timestamp is not
    reliable.
  \item[] \emph{port ID}: the ID of a physical port on which the packet was
    received. In case of WRPC, this field is always 0, because there is only one
    physical port available.
  \item[] \emph{CNTR\_f}: least significant bits of the Rx timestamp generated on
    the falling edge of the reference clock.
  \item[] \emph{CNTR\_r}: Rx timestamp generated on the rising edge of the reference
    clock.
\end{itemize}

\begin{figure}
  \begin{center}
    \includegraphics[width=.6\textwidth]{fig/basic_wrf_data.pdf}
    \caption{Data words that make the Ethernet frame}
    \label{fig:fabric:simple_data}
  \end{center}
\end{figure}

Figure \ref{fig:fabric:simple_data} presents data words fed to the WRF
data bus by the sender and the information got at the receiving side. Please
note that the CRC checksum is calculated and inserted automatically inside the
WRPC and user-defined module doesn't care about it. The Ethernet frame received
from the WR Fabric interface may contain additional OOB data suffixed. It has to
be received (acknowledged) by the user-defined module, but can be simply discarded.

\newparagraph{Examples}
Figure \ref{fig:fabric:simple_tx} shows a very simple WR Fabric cycle. The WRF
Source of user-defined module sends there an Ethernet frame containing even
number of bytes.

\begin{figure}
  \begin{center}
    \includegraphics[width=\textwidth]{fig/basic_wrf_cycle_simple.pdf}
    \caption{Simple WR Fabric cycle - user-defined module sending packet}
    \label{fig:fabric:simple_tx}
  \end{center}
\end{figure}

\begin{enumerate}
  \item The WRF Source in user-defined module starts the cycle by asserting
    \emph{cyc\_o}, \emph{stb\_o} and putting a status word to the data bus.
    However, since WRF Sink set \emph{stall} signal to active state, Source has
    to wait until Sink is ready to receive data.
  \item After the last word is transmitted, the WRF Source sets \emph{stb\_o} back
    to \emph{0}, but waits until Sink acknowledges all the words transmitted in
    the cycle (\emph{ack\_i} line). The cycle ends when \emph{cyc\_o} goes back
    to the low state.
\end{enumerate}

Figure \ref{fig:fabric:sel} shows again a very simple WR Fabric cycle where
user-defined WRF Source sends an Ethernet frame to the WRPC. This time though,
the frame contains odd number of bytes, therefore the \emph{sel} line is used to
signal this fact to WRF Sink inside the WRPC (1).

\begin{figure}
  \begin{center}
    \includegraphics[width=\textwidth]{fig/basic_wrf_cycle_sel.pdf}
    \caption{Simple WR Fabric cycle - user-defined module sending packet(odd
    number of bytes in the payload)}
    \label{fig:fabric:sel}
  \end{center}
\end{figure}

Figure \ref{fig:fabric:cyc} presents more complicated Fabric cycle where an
Ethernet frame is received from WRF Source in the WRPC (output signals in the
diagram are driven by WRF Source on the WRPC side): 

\begin{figure}
  \begin{center}
    \includegraphics[width=\textwidth]{fig/basic_wrf_cycle.pdf}
    \caption{WR Fabric cycle}
    \label{fig:fabric:cyc}
  \end{center}
\end{figure}
\begin{enumerate}
  \item The WRF Source starts the cycle by asserting \emph{cyc\_o}, \emph{stb\_o}
    and putting a status word to the data bus. However, since WRF Sink set 
    \emph{stall} signal to active state, Source has to wait until Sink is ready
    to receive data.
  \item While the payload of the Ethernet frame is being transmitted, Sink
    stalls the cycle. The WRF Source pauses the transmission until Sink becomes
    ready to process the rest of the data. During that time \emph{stb\_o} has to
    remain in a high state.
  \item The Ethernet frame contains an odd number of bytes, so only half of last
    word of payload carries a valid data. \emph{Sel\_o} is used to signal this
    fact to WRF Sink.
  \item After the whole payload is transmitted, Source may additionally sent Rx
    OOB data. It contains some internal WRPC data that should be acknowledged
    by Sink, but discarded in the user's module.
  \item After the last word is transmitted, the WRF Source sets \emph{stb\_o} back
    to \emph{0}, but waits until Sink acknowledges all the words transmitted in
    the cycle (\emph{ack\_i} line). The cycle ends when \emph{cyc\_o} goes back
    to the low state.
\end{enumerate}

WRF Sink can use the \emph{stall} line to pause the frame transmission if it cannot
process the flow of data coming from WRF Source. However, if some more serious
problem appears on the receiving side, the \emph{err} line can be used to
immediately break the cycle. This situation is presented in figure
\ref{fig:fabric:cycerr}:

\begin{figure}
  \begin{center}
    \includegraphics[width=\textwidth]{fig/basic_wrf_cycle_err.pdf}
    \caption{WR Fabric cycle interrupted with an error line}
    \label{fig:fabric:cycerr}
  \end{center}
\end{figure}

\begin{enumerate}
  \item WRF Sink wants to break a bus cycle, so it drives \emph{err\_i} high.
  \item WRF Source breaks the cycle immediately after receiving an error indicator
    from the WRF Sink.
\end{enumerate}

\newparagraph{SystemVerilog model}
The SystemVerilog simulation model of the WR Fabric interface (both WRF Source and 
WRF Sink) can be found in the \emph{wr-cores} git repository
(git://ohwr.org/hdl-core-lib/wr-cores.git) and consists of the files:
\begin{itemize}
  \item \emph{sim/if\_wb\_master.svh}
  \item \emph{sim/if\_wb\_slave.svh}
  \item \emph{sim/wb\_packet\_source.svh}
  \item \emph{sim/wb\_packet\_sink.svh}
\end{itemize}

The testbench example using the simulation model of WR Fabric interface can
be found in the zip archive attached to this documentation.


\subsubsection{Tx Timestamping interface}
\label{sec:txts}

The Tx Timestamping interface provides the timestamps generated inside WRPC for each
Ethernet frame transmitted from user-defined module through the WRF Sink interface.\\


\subsection{Timecode interface}
\label{sec:wrpc_timecode}

%\begin{figure}[ht]
%  \begin{center}
%    \includegraphics[width=.5\textwidth]{fig/basic_wrpc_tm.pdf}
%    \caption{Timecode output interface of WRPC}
%  \end{center}
%\end{figure}

Timecode interface provides current time to the other HDL modules in a form that
can be easily used. It consists of: a 1-PPS and a UTC timecode
aligned to the time of WR Master.


\subsubsection{Auxiliary diagnostics interface}
\label{sec:aux_diag}

Auxiliary diagnostics interface can be used if a user would like to benefit from
the WR PTP Core diagnostics capabilities to export some registers from his/her
IP core. The interface consists of two 32-bit \texttt{std\_logic\_vector}
arrays. User-defined registers that are to be read from the WRPC SNMP agent (SNMP
GET requests), should be connected to the \texttt{aux\_diag\_i} vector.
User-defined values that are written from the WRPC SNMP agent (SNMP SET
requests) will be available in the \texttt{aux\_diag\_o} vector.

Two VHDL generics \texttt{g\_diag\_id} and \texttt{g\_diag\_ver} are used to let
the user uniquely identify given application (user-defined set of registers)
to match it with appropriate, custom SNMP MIB file.


\subsection{Platform Support Packages}
\label{sec:hdl_platform}

The White Rabbit (WR) PTP core project provides platform support packages (PSPs) for Altera and
Xilinx FPGAs.

By using these modules, users gain the benefit of instantiating all the platform-specific support
components for the WR PTP core (PHY, PLLs, etc.) in one go, without having to delve into the
implementation details, using a setup that has been tested and is known to work well on the
supported FPGAs.

\subsubsection{Common}
\label{sec:hdl_platform_common}

This section describes the generic parameters and ports which are common to all provided PSPs.

\newparagraph{Generic parameters}

\begin{hdlparamtable}
  g\_with\_external\_clock\_input & boolean & false & Select whether to
  include the external 10MHz reference clock input (used in WR Grandmaster mode)\\
  \hline
  g\_use\_default\_plls & boolean & true & Set to FALSE if you want to
  instantiate your own PLLs\\
\end{hdlparamtable}

Each PSP provides two generic parameters of boolean type , which allow the users to configure the
PLLs in their designs. As such, four different PLL setups can be achieved by changing the values of
these parameters.

\begin{description}
\item[PLL setup 1:] Use default PLLs, no external reference clock. In this setup, the PSP expects
  one 20MHz and one 125MHz clock, and it will instantiate all the required PLLs internally. This is
  the default mode.
\item[PLL setup 2:] Use default PLLs, with external reference clock. This is the same as PLL setup
  1, with the addition of the external 10MHz reference clock input, which will be multiplied
  internally by the PSP to 125MHz.
\item[PLL setup 3:] Use custom PLLs, no external reference clock. In this setup, the PSP will not
  instantiate any PLLs internally. It is up to the user to provide the 62.5MHz system clock, the
  125MHz reference clock and the 62.5MHz DMTD clock.
\item[PLL setup 4:] Use custom PLLs, with external reference clock. This is the same as PLL setup 3,
  with the addition of the external reference clock input, which should be provided as is (10MHz)
  and also multiplied to 125MHz.
\end{description}

\newparagraph{Ports}

\begin{hdlporttable}
  areset\_n\_i & in & 1 & asynchronous reset (active low)\\
  \hline
  clk\_10m\_ext\_i & in & 1 & 10MHz external reference clock input
  (used when \tts{g\_with\_external\_clock\_input = true})\\
  \hline\pagebreak
  \hdltablesection{Clock inputs for default PLLs (used when
    \tts{g\_use\_default\_plls = true})}\\
  \hline
  clk\_20m\_vcxo\_i & in & 1 & 20MHz VCXO clock\\
  \hline
  clk\_125m\_pllref\_i & in & 1 & 125MHz PLL reference\\
  \hline
  \hdltablesection{Interface with custom PLLs (used when
    \tts{g\_use\_default\_plls = false})}\\
  \hline
  clk\_62m5\_dmtd\_i & in & 1 &  \multirowpar{2}{62.5MHz DMTD offset
    clock and lock status}\\
  \cline{1-3}
  clk\_dmtd\_locked\_i & in & 1 & \\
  \hline
  clk\_62m5\_sys\_i & in & 1 & \multirowpar{2}{62.5MHz Main system
    clock and lock status}\\
  \cline{1-3}
  clk\_sys\_locked\_i & in & 1 & \\
  \hline
  clk\_125m\_ref\_i & in & 1 & 125MHz Reference clock\\
  \hline
  clk\_125m\_ext\_i & in & 1 & \multirowpar{4}{125MHz derived
      from 10MHz external reference, locked/stopped status inputs and reset output (used when
      \tts{g\_with\_external\_clock\_input = true})}\\
  \cline{1-3}
  clk\_ext\_locked\_i & in & 1 & \\
  \cline{1-3}
  clk\_ext\_stopped\_i & in & 1 & \\
  \cline{1-3}
  clk\_ext\_rst\_o & out & 1 &\\
  \hline
  \hdltablesection{Interface with SFP}\\
  \hline
  sfp\_tx\_fault\_i & in & 1 & TX fault indicator\\
  \hline
  sfp\_los\_i & in & 1 & Loss Of Signal indicator\\
  \hline
  sfp\_tx\_disable\_o & out & 1 & TX disable control\\
  \hline
  \hdltablesection{Interface with WRPC}\\
  \hline
  clk\_62m5\_sys\_o & out & 1 & 62.5MHz system clock output\\
  \hline
  clk\_125m\_ref\_o & out & 1 & 125MHz reference clock output\\
  \hline
  clk\_62m5\_dmtd\_o & out & 1 & 62.5MHz DMTD clock output\\
  \hline
  pll\_locked\_o & out & 1 & logic AND of system and DMTD PLL lock\\
  \hline
  clk\_10m\_ext\_o & out & 1 & 10MHz external reference clock output\\
  \hline
  phy8\_o & out & rec & \multirowpar{2}{input/output records for PHY signals
    when \tts{g\_pcs\_16bit = false}}\\
  \cline{1-3}
  phy8\_i & in & rec & \\
  \hline
  phy16\_o & out & rec & \multirowpar{2}{input/output records for PHY signals
    when \tts{g\_pcs\_16bit = true}}\\
  \cline{1-3}
  phy16\_i & in & rec & \\
  \hline
  ext\_ref\_mul\_o & out & 1 & \multirowpar{4}{125MHz derived from
    10MHz external reference, locked/stopped status outputs and reset input}\\
  \cline{1-3}
  ext\_ref\_mul\_locked\_o & out & 1 & \\
  \cline{1-3}
  ext\_ref\_mul\_stopped\_o & out & 1 & \\
  \cline{1-3}
  ext\_ref\_rst\_i & in & 1 & \\
\end{hdlporttable}

\subsubsection{Altera}
\label{sec:hdl_platform_altera}

The Altera PSP currently supports the Arria V family of FPGAs.

The top-level VHDL module is located under:\\\hrefwrpc{platform/altera/xwrc\_platform\_altera.vhd}

A VHDL package with the definition of the module can be found
under:\\\hrefwrpc{platform/wr\_altera\_pkg.vhd}

An example (VHDL) instantiation of this module can be found in the VFC-HD board support package (see
also Section~\ref{sec:hdl_board_vfchd}):\\\hrefwrpc{board/vfchd/xwrc\_board\_vfchd.vhd}

This section describes the generic parameters and ports which are specific to the Altera
PSP. Parameters and ports common to all PSPs are described in Section~\ref{sec:hdl_platform_common}.

\newparagraph{Generic parameters}

\begin{hdlparamtable}
  g\_fpga\_family & string & arria5 & Defines the family/model of Altera
  FPGA. Recognized values are "arria5" (more will be added)\\
  \hline
  g\_pcs16\_bit & boolean & false & Some FPGA families provide the possibility
  to configure the PCS of the PHY as either 8bit or 16bit. The default is to use the 8bit PCS,
  but this generic can be used to override it\\
\end{hdlparamtable}

\newparagraph{Ports}

\begin{hdlporttable}
  \hdltablesection{Interface with SFP}\\
  \hline
  \linebreak sfp\_tx\_o\linebreak & out & 1 & \multirowpar{2}{PHY TX and RX. These
    are single ended and should be mapped to the positive half of each differential signal.
    Altera tools will infer both the negative half and the differential receiver}\\
  \cline{1-3}
  \linebreak sfp\_rx\_i\linebreak & in & 1 &\\
\end{hdlporttable}

\subsubsection{Xilinx}
\label{sec:hdl_platform_xilinx}

The Xilinx PSP currently supports the Spartan 6 and Kintex 7 (also inside Zynq) family of FPGAs.

The top-level VHDL module is located under:\\ \hrefwrpc{platform/xilinx/xwrc\_platform\_xilinx.vhd}

A VHDL package with the definition of the module can be found
under:\\ \hrefwrpc{platform/wr\_xilinx\_pkg.vhd}

Examples of (VHDL) instantiation of this module can be found in the SPEC, SVEC
and FASEC board support packages (see also Sections~\ref{sec:hdl_board_spec}
and~\ref{sec:hdl_board_svec}):\\
\hrefwrpc{board/spec/xwrc\_board\_spec.vhd}\\
\hrefwrpc{board/svec/xwrc\_board\_svec.vhd}\\
\hrefwrpc{board/fasec/xwrc\_board\_fasec.vhd}\\

This section describes the generic parameters and ports which are
specific to the Xilinx PSP. Parameters and ports common to all PSPs
are described in Section~\ref{sec:hdl_platform_common}.

\newparagraph{Generic parameters}

\begin{hdlparamtable}
  g\_fpga\_family & string & spartan6 & Defines the family/model of Xilinx
  FPGA. Recognized values are "spartan6" (more will be added)\\
  \hline
  g\_simulation & integer & 0 & setting to '1' speeds up the simulation, must
  be set to '0' for synthesis\\
\end{hdlparamtable}

\newparagraph{Ports}

\begin{hdlporttable}
  clk\_125m\_gtp\_p\_i & in & 1 & \multirowpar{2}{125MHz GTP
    reference differential clock input}\\
  \cline{1-3}
  clk\_125m\_gtp\_n\_i & in & 1 & \\
  \hline
  \hdltablesection{Interface with SFP}\\
  \hline
  sfp\_txn\_o & out & 1 & \multirowpar{2}{differential pair for PHY TX}\\
  \cline{1-3}
  sfp\_txp\_o & out & 1 & \\
  \hline
  sfp\_rxn\_i & in & 1 & \multirowpar{2}{differential pair for PHY RX}\\
  \cline{1-3}
  sfp\_rxp\_i & in & 1 & \\
\end{hdlporttable}

\subsection{Board Support Packages}
\label{sec:hdl_board}

The White Rabbit (WR) PTP core project provides board support packages (BSPs) for the following
boards:
\begin{itemize}
\item \href{http://www.ohwr.org/projects/spec}{SPEC}, a PCIe single FMC carrier board based on a
  Xilinx Spartan 6 FPGA.
\item \href{http://www.ohwr.org/projects/svec}{SVEC}, a VME dual FMC carrier board based on a Xilinx
  Spartan 6 FPGA.
\item \href{http://www.ohwr.org/projects/vfc-hd}{VFC-HD}, a VME single FMC carrier board based on an
  Altera Arria V FPGA.
\end{itemize}

By using these modules, users gain the benefit of instantiating all the necessary components of the
WR PTP core (including the core itself, the PHY, PLLs, etc.) in one go, without having to delve into
the implementation details, using a setup that has been tested and is known to work well on the
supported boards.

Each BSP is split in two modules: the common module, which is shared across all BSPs, and the
board-specific module. The common module instantiates the WRPC itself, together with a selection of
interfaces for connecting the core the the user FPGA logic. The board-specific module instantiates
all the FPGA- and system-specific parts (related to WR), such as hard IP provided by the FPGA
vendor, interfaces to DACs, reset inputs, etc.

The BSPs make use internally of the appropriate FPGA family platform support packages (PSPs, see also
Section~\ref{sec:hdl_platform}). For users who need more control and flexibility over their designs,
it is suggested to use the BSP as a reference, and to consider instantiating directly the respective
PSP for their FPGA family.

\subsubsection{Common}
\label{sec:hdl_board_common}

Most of the generic parameters and ports of the board-common module map directly to those of the
WRPC. One notable exception to this rule is that of the parameters and ports related to the selected
interface for connecting the core to the user FPGA logic.

The board-common module provides the \tts{g\_fabric\_iface} generic parameter, an enumeration
type with three possible values:

\begin{description}
\item[PLAIN:] No additional module is instantiated and the ``raw'' WRPC fabric interface (see also
  Section~\ref{sec:wrpc_fabric}) is provided on the board's ports.
\item[STREAMERS:] A set of \href{http://www.ohwr.org/projects/wr-cores/wiki/WR_Streamers}{TX/RX
  streamers} is attached to the WRPC fabric interface.
\item[ETHERBONE:] An \href{http://www.ohwr.org/projects/etherbone-core/wiki}{Etherbone} slave node
  is attached to the WRPC fabric interface.
\end{description}

Sections~\ref{sec:hdl_board_common_param} and~\ref{sec:hdl_board_common_ports} list the generic
parameters and ports of the board-common module which are shared across the BSPs.

\textbf{Note:} the board-common module defines more parameters and ports than the ones mentioned in
the following sections. Those that are not exposed by any of the BSPs have been left out to keep the
tables short and to the point. Users interested in studying the board-common module and/or writing
their own BSP, can find the board-common module under:
\\\hrefwrpc{board/common/xwrc\_board\_common.vhd}

\newpage
\newparagraph{Generic parameters}
\label{sec:hdl_board_common_param}

\begin{hdlparamtable}
  g\_simulation & integer & 0 & \multirowpar{11}{These map directly to generic parameters with the
    same name in the WRPC module (see Section~\ref{sec:wrc_generics})}\\
  \cline{1-3}
  g\_with\_external\_clock\_input & boolean & true & \\
  \cline{1-3}
  g\_board\_name & string & "NA   " & \\
  \cline{1-3}
  g\_flash\_secsz\_kb & integer & 256 & \\
  \cline{1-3}
  g\_flash\_sdbfs\_baddr & integer & 0x600000 & \\
  \cline{1-3}
  g\_aux\_clks & integer & 0 & \\
  \cline{1-3}
  g\_dpram\_initf & string & "" & \\
  \cline{1-3}
  g\_diag\_id & integer & 0 & \\
  \cline{1-3}
  g\_diag\_ver & integer & 0 & \\
  \cline{1-3}
  g\_diag\_ro\_size & integer & 0 & \\
  \cline{1-3}
  g\_diag\_rw\_size & integer & 0 & \\
  \hline
  g\_streamers\_op\_mode & enum & TX\_AND\_RX & Selects whether both TX and RX
  streamer modules should be instantiated or only one of them when
  \tts{g\_fabric\_iface = STREAMERS} (otherwise ignored)\\
  \hline
  g\_tx\_streamer\_params & record & default record & \multirowpar{2}{various TX/RX
    streamers parameters when \tts{g\_fabric\_iface = STREAMERS} (otherwise
    ignored)\footnote{See Streamers wiki page for detailed description of the
    configuration records:
    \url{http://www.ohwr.org/projects/wr-cores/wiki/TxRx_Streamers}}}\\
  \cline{1-3}
  g\_rx\_streamers\_params & record & default record & \\
  \hline
  g\_fabric\_iface & enum & PLAIN & optional module to be attached to the
  fabric interface of WRPC \tts{[PLAIN/STREAMERS/ETHERBONE]}\\
\end{hdlparamtable}

\newparagraph{Ports}
\label{sec:hdl_board_common_ports}

\begin{hdlporttable}
  \hdltablesection{Clocks and resets}\\
  \hline
  clk\_aux\_i & in & var & [optional] vector of auxiliary
  clocks that will be disciplined to WR timebase. Size is equal to \tts{g\_aux\_clks}\\
  \hline
  clk\_10m\_ext\_i & in & 1 & 10MHz external reference clock input
  (used when \tts{g\_with\_external\_clock\_input = true})\\
  \hline
  pps\_ext\_i & in & 1 & external 1-PPS input (used when
  \tts{g\_with\_external\_clock\_input = true})\\
  \hline
  clk\_sys\_62m5\_o & out & 1 & 62.5MHz system clock output\\
  \hline
  clk\_ref\_125m\_o & out & 1 & 125MHz reference clock output\\
  \hline
  rst\_62m5\_n\_o & out & 1 & Active low reset output, synchronous to \tts{clk\_sys\_62m5\_o}\\
  \hline
  rst\_125m\_n\_o & out & 1 & Active low reset output, synchronous to \tts{clk\_ref\_125m\_o}\\
  \hline
  \hdltablesection{Interface with SFP}\\
  \hline
  sfp\_tx\_fault\_i & in & 1 & TX fault indicator\\
  \hline
  sfp\_los\_i & in & 1 & Loss Of Signal indicator\\
  \hline
  sfp\_tx\_disable\_o & out & 1 & TX disable control\\
  \hline
  \hdltablesection{I2C EEPROM interface}\\
  \hline
  eeprom\_sda\_i & in  & 1 & \multirowpar{2}{EEPROM I2C SDA}\\
  \cline{1-3}
  eeprom\_sda\_o & out & 1 & \\
  \hline
  eeprom\_scl\_i & in  & 1 & \multirowpar{2}{EEPROM I2C SCL}\\
  \cline{1-3}
  eeprom\_scl\_o & out & 1 & \\
  \hline
  \hdltablesection{Onewire interface (UID and temperature)}\\
  \hline
  onewire\_i & in  & 1 & OneWire data input\\
  \hline
  onewire\_oen\_o & out & 1 & OneWire data output enable (when asserted,
  OneWire tri-state output buffer should be enabled and driven to ground)\\
  \hline
  \hdltablesection{External WishBone interface}\\
  \hline
  wb\_slave\_o & out & rec & \multirowpar{2}{Mapped to WRPC external WB slave
    interface (see also Section~\ref{sec:wrpc_wb})}\\
  \cline{1-3}
  wb\_slave\_i & in & rec & \\
  \hline
  aux\_master\_o & out & rec & \multirowpar{2}{Mapped to WRPC auxiliary WB
  master interface (see also Section~\ref{sec:wrpc_wb})}\\
  \cline{1-3}
  aux\_master\_i & in & rec & \\
  \hline
  \hdltablesection{WR fabric interface (when \tts{g\_fabric\_iface = plain})}\\
  \hline
  wrf\_src\_o & out & rec & \multirowpar{4}{Mapped to WRPC fabric interface
    (see also Section~\ref{sec:wrpc_fabric})}\\
  \cline{1-3}
  wrf\_src\_i & in &  rec & \\
  \cline{1-3}
  wrf\_snk\_o & out & rec & \\
  \cline{1-3}
  wrf\_snk\_i & in &  rec & \\
  \hline
  \hdltablesection{WR streamers (when \tts{g\_fabric\_iface = streamers})}\\
  \hline
  wrs\_tx\_data\_i & in &  var & Data to be sent. Size is equal to \tts{g\_tx\_streamer\_width}\\
  \hline
  wrs\_tx\_valid\_i & in & 1 & Indicates whether \tts{wrs\_tx\_data\_i} contains valid data\\
  \hline
  wrs\_tx\_dreq\_o & out & 1 & When active, the user may send a data word in the
  following clock cycle\\
  \hline
  wrs\_tx\_last\_i & in &  1 & Can be used to indicate the last data word in a
  larger block of samples\\
  \hline
  wrs\_tx\_flush\_i & in &  1 & When asserted, the streamer will immediatly send
  out all the data that is stored in its TX buffer\\
  \hline
  wrs\_tx\_cfg\_i & in & rec & \multirowpar{2}{Networking configuration of Tx/Rx
  Streamers\footnote{See Streamers wiki page for detailed description of the
    network configuration:
    \url{http://www.ohwr.org/projects/wr-cores/wiki/TxRx_Streamers}}}\\
  \cline{1-3}
  wrs\_rx\_cfg\_i & in & rec & \\
  \hline
  wrs\_rx\_first\_o & out & 1 & Indicates the first word of the data block on \tts{wrs\_rx\_data\_o}\\
  \hline
  wrs\_rx\_last\_o & out & 1 & Indicates the last word of the data block on \tts{wrs\_rx\_data\_o}\\
  \hline
  wrs\_rx\_data\_o & out & var & Received data. Size is equal to \tts{g\_rx\_streamer\_width}\\
  \hline
  wrs\_rx\_valid\_o & out & 1 & Indicates that \tts{wrs\_rx\_data\_o} contains valid data\\
  \hline
  wrs\_rx\_dreq\_i & in &  1 & When asserted, the streamer may output another data word in the
  subsequent clock cycle\\
  \hline
  \hdltablesection{Etherbone WB master interface (when \tts{g\_fabric\_iface = etherbone})}\\
  \hline
  \linebreak wb\_eth\_master\_o\linebreak & out & rec & \multirowpar{2}{WB master interface for the
    Etherbone core. Normally this is attached to a slave port of the primary WB crossbar in the design,
    in order to provide access to all WB peripherals over Etherbone}\\
  \cline{1-3}
  \linebreak wb\_eth\_master\_i\linebreak & in & rec & \\
  \hline
  \hdltablesection{Generic diagnostics interface}\\
  \hline
  \linebreak aux\_diag\_i\linebreak & in & var & \multirowpar{2}{Arrays of 32~bit vectors, to be
    accessed from WRPC via SNMP or uart console. Input array contains \tts{g\_diag\_ro\_size},
    while output array contains \tts{g\_diag\_rw\_size} elements.}\\
  \cline{1-3}
  \linebreak aux\_diag\_o\linebreak & out & var & \\
  \hline
  \hdltablesection{Auxiliary clocks control}\\
  \hline
  tm\_dac\_value\_o & out & 24 & DAC value for tuning auxiliary clock
  (\emph{clk\_aux\_i})\\
  \hline
  tm\_dac\_wr\_o & out & var & validates auxiliary DAC value. Size is equal
  to \tts{g\_aux\_clks}\\
  \hline
  tm\_clk\_aux\_lock\_en\_i & in & var & enable locking auxiliary clock to
  internal WR clock. Size is equal to \tts{g\_aux\_clks}\\
  \hline
  tm\_clk\_aux\_locked\_o & out & var & auxiliary clock locked to internal WR
  clock. Size is equal to \tts{g\_aux\_clks}\\
  \hline \pagebreak 
  \hdltablesection{External TX timestamp interface}\\
  \hline
  timestamps\_o & out & rec & Record-based output ports for
  the TX timestamp interface (see also Section~\ref{sec:txts})\\
  \hline
  txtsu\_ack\_i & in & 1 & acknowledge, indicating that user-defined module
  has received the timestamp\\
  \hline 
  abscal\_txts\_o & out & 1 & \multirowpar{2}{[optional] Endpoint timestamping
  triggers used in the absolute calibration procedure} \\
  \cline{1-3}
  abscal\_rxts\_o & out & 1 & \\
  \hline
  \hdltablesection{Pause frame control}\\
  \hline
  fc\_tx\_pause\_req\_i   & in  &  1 & [optional] Ethernet flow control, request sending
  Pause frame\\
  \hline
  fc\_tx\_pause\_delay\_i & in  & 16 & [optional] Pause quanta\\
  \hline
  fc\_tx\_pause\_ready\_o & out &  1 & [optional] Pause acknowledge - active after the
  current pause send request has been completed\\
  \hline
  \hdltablesection{WRPC timecode interface}\\
  \hline
  tm\_link\_up\_o & out & 1 & state of Ethernet link (up/down), \emph{1}
  means Ethernet link is up\\
  \hline
  tm\_time\_valid\_o & out & 1 & if \emph{1}, the timecode generated by the
  WRPC is valid\\
  \hline
  tm\_tai\_o & out & 40 & TAI part of the timecode (full seconds)\\
  \hline
  tm\_cycles\_o & out & 28 & fractional part of each second represented by
  the state of counter clocked with the frequency 125 MHz (values from 0 to
  124999999, each count is 8 ns)\\
  \hline
  \hdltablesection{Buttons, LEDs and PPS output}\\
  \hline
  led\_act\_o & out & 1 & signal for driving Ethernet activity LED\\
  \hline
  led\_link\_o & out & 1 & signal for driving Ethernet link LED\\
  \hline
  btn1\_i & in & 1 & \multirowpar{2}{two microswitch inputs, active low, currently not
    used in official WRPC software}\\
  \cline{1-3}
  btn2\_i & in & 1 & \\
  \hline
  pps\_p\_o & out & 1 & 1-PPS signal generated in \tts{clk\_ref\_i} clock
  domain and aligned to WR time, pulse generated when the cycle counter is 0
  (beginning of each full TAI second)\\
  \hline
  pps\_led\_o & out & 1 & 1-PPS signal with extended pulse width to drive a LED\\
  \hline
  link\_ok\_o & out & 1 & Link status indicator\\
\end{hdlporttable}

\subsubsection{SPEC}
\label{sec:hdl_board_spec}

The SPEC BSP provides a ready-to-use WRPC wrapper for the
\href{http://www.ohwr.org/projects/spec}{SPEC carrier board}.

The top-level VHDL module is located under: \\\hrefwrpc{board/spec/xwrc\_board\_spec.vhd}

An alternative top-level VHDL module which only makes use of standard logic for ports and integers
and strings for generic parameters (ideal for instantiation in Verilog-based designs) can be found
under: \\\hrefwrpc{board/spec/wrc\_board\_spec.vhd}

A VHDL package with the definition of both modules can be found under:
\\\hrefwrpc{board/spec/wr\_spec\_pkg.vhd}

An example (VHDL) instantiation of this module can be found in the SPEC WRPC reference design:
\\\hrefwrpc{top/spec\_ref\_design/spec\_wr\_ref\_top.vhd}

This section describes the generic parameters and ports which are specific to the SPEC BSP.
Parameters and ports common to all BSPs are described in Section~\ref{sec:hdl_board_common}.

\newparagraph{Generic parameters}

No additional generic parameters are declared in the SPEC BSP. See
Section~\ref{sec:hdl_board_common_param} for a the list of common BSP parameters.


\newparagraph{Ports}

\begin{hdlporttable}
  \hdltablesection{Clocks and resets}\\
  \hline
  areset\_n\_i & in & 1 & Reset input (active low, can be async)\\
  \hline
  areset\_edge\_n\_i & in & 1 & [optional] Reset input edge sensitive (active
  rising-edge, can be async). Should be connected to PCIe reset if the board
  should be able to operate both in hosted and standalone configuration.\\
  \hline
  clk\_20m\_vcxo\_i & in & 1 & 20MHz clock input from board VCXO\\
  \hline
  clk\_125m\_pllref\_p\_i & in & 1 & \multirowpar{2}{125MHz PLL reference
    differential clock input from board}\\
  \cline{1-3}
  clk\_125m\_pllref\_n\_i & in & 1 & \\
  \hline
  clk\_125m\_gtp\_p\_i & in & 1 & \multirowpar{2}{125MHz GTP
    reference differential clock input from board}\\
  \cline{1-3}
  clk\_125m\_gtp\_n\_i & in & 1 & \\
  \hline
  \hdltablesection{SPI interface to DACs}\\
  \hline
  plldac\_sclk\_o & out & 1 & SPI SCLK, common to both DACs\\
  \hline
  plldac\_din\_o & out & 1 & SPI MOSI, common to both DACs\\
  \hline
  pll25dac\_cs\_n\_o & out & 1 & SPI $\overline{\mbox{SS}}$ for DAC controlling 25MHz oscillator\\
  \hline
  pll20dac\_cs\_n\_o & out & 1 & SPI $\overline{\mbox{SS}}$ for DAC controlling 20MHz oscillator\\
  \hline
  \hdltablesection{SFP interface}\\
  \hline
  sfp\_txn\_o & out & 1 & \multirowpar{2}{differential pair for PHY TX}\\
  \cline{1-3}
  sfp\_txp\_o & out & 1 & \\
  \hline
  sfp\_rxn\_i & in & 1 & \multirowpar{2}{differential pair for PHY RX}\\
  \cline{1-3}
  sfp\_rxp\_i & in & 1 & \\
  \hline
  sfp\_det\_i & in  & 1 & Active low, indicates presence of SFP (corresponds to SFP MOD-DEF0)\\
  \hline
  sfp\_sda\_i & in  & 1 & \multirowpar{2}{SFP I2C SDA}\\
  \cline{1-3}
  sfp\_sda\_o & out & 1 & \\
  \hline
  sfp\_scl\_i & in  & 1 & \multirowpar{2}{SFP I2C SCL}\\
  \cline{1-3}
  sfp\_scl\_o & out & 1 & \\
  \hline
  sfp\_rate\_select\_o & out & 1 & SFP rate select\\
  \hline
  \hdltablesection{Physical UART interface}\\
  \hline
  uart\_rxd\_i & in  & 1 & UART RXD (serial data to WRPC)\\
  \hline
  uart\_txd\_o & out & 1 & UART TXD (serial data from WRPC)\\
  \hline
  \hdltablesection{Flash memory SPI interface}\\
  \hline
  flash\_sclk\_o & out & 1 & Flash SPI SCLK\\
  \hline
  flash\_ncs\_o  & out & 1 & Flash SPI $\overline{\mbox{SS}}$\\
  \hline
  flash\_mosi\_o & out & 1 & Flash SPI MOSI\\
  \hline
  flash\_miso\_i & in  & 1 & Flash SPI MISO\\
\end{hdlporttable}

\subsubsection{SVEC}
\label{sec:hdl_board_svec}

The SVEC BSP provides a ready-to-use WRPC wrapper for the
\href{http://www.ohwr.org/projects/svec}{SVEC carrier board}.

The top-level VHDL module is located under: \\\hrefwrpc{board/svec/xwrc\_board\_svec.vhd}

An alternative top-level VHDL module which only makes use of standard logic for ports and integers
and strings for generic parameters (ideal for instantiation in Verilog-based designs) can be found
under: \\\hrefwrpc{board/svec/wrc\_board\_svec.vhd}

A VHDL package with the definition of both modules can be found under:
\\\hrefwrpc{board/svec/wr\_svec\_pkg.vhd}

An example (VHDL) instantiation of this module can be found in the SVEC WRPC reference design:
\\\hrefwrpc{top/svec\_ref\_design/svec\_wr\_ref\_top.vhd}

This section describes the generic parameters and ports which are specific to the SVEC BSP.
Parameters and ports common to all BSPs are described in Section~\ref{sec:hdl_board_common}.

\newparagraph{Generic parameters}

No additional generic parameters are declared in the SVEC BSP. See
Section~\ref{sec:hdl_board_common_param} for a the list of common BSP parameters.

\newparagraph{Ports}

\begin{hdlporttable}
  \hdltablesection{Clocks and resets}\\
  \hline
  areset\_n\_i & in & 1 & Reset input (active low, can be async)\\
  \hline
  areset\_edge\_n\_i & in & 1 & [optional] Reset input edge sensitive (active
  rising-edge, can be async).\\
  \hline
  clk\_20m\_vcxo\_i & in & 1 & 20MHz clock input from board VCXO\\
  \hline
  clk\_125m\_pllref\_p\_i & in & 1 & \multirowpar{2}{125MHz PLL reference
    differential clock input from board}\\
  \cline{1-3}
  clk\_125m\_pllref\_n\_i & in & 1 & \\
  \hline
  clk\_125m\_gtp\_p\_i & in & 1 & \multirowpar{2}{125MHz GTP
    reference differential clock input from board}\\
  \cline{1-3}
  clk\_125m\_gtp\_n\_i & in & 1 & \\
  \hline
  \hdltablesection{SPI interface to DACs}\\
  \hline
  pll20dac\_sclk\_o & out & 1 & SPI SCLKfor DAC controlling 20MHz oscillator\\
  \hline
  pll20dac\_din\_o & out & 1 & SPI MOSI for DAC controlling 20MHz oscillator\\
  \hline
  pll20dac\_cs\_n\_o & out & 1 & SPI $\overline{\mbox{SS}}$ for DAC controlling 20MHz oscillator\\
  \hline
  pll25dac\_sclk\_o & out & 1 & SPI SCLK for DAC controlling 25MHz oscillator\\
  \hline
  pll25dac\_din\_o & out & 1 & SPI MOSI for DAC controlling 25MHz oscillator\\
  \hline
  pll25dac\_cs\_n\_o & out & 1 & SPI $\overline{\mbox{SS}}$ for DAC controlling 25MHz oscillator\\
  \hline
  \hdltablesection{SFP interface}\\
  \hline
  sfp\_txn\_o & out & 1 & \multirowpar{2}{differential pair for PHY TX}\\
  \cline{1-3}
  sfp\_txp\_o & out & 1 & \\
  \hline
  sfp\_rxn\_i & in & 1 & \multirowpar{2}{differential pair for PHY RX}\\
  \cline{1-3}
  sfp\_rxp\_i & in & 1 & \\
  \hline
  sfp\_det\_i & in  & 1 & Active low, indicates presence of SFP (corresponds to SFP MOD-DEF0)\\
  \hline
  sfp\_sda\_i & in  & 1 & \multirowpar{2}{SFP I2C SDA}\\
  \cline{1-3}
  sfp\_sda\_o & out & 1 & \\
  \hline
  sfp\_scl\_i & in  & 1 & \multirowpar{2}{SFP I2C SCL}\\
  \cline{1-3}
  sfp\_scl\_o & out & 1 & \\
  \hline
  sfp\_rate\_select\_o & out & 1 & SFP rate select\\
  \hline
  \hdltablesection{Physical UART interface}\\
  \hline
  uart\_rxd\_i & in  & 1 & UART RXD (serial data to WRPC)\\
  \hline
  uart\_txd\_o & out & 1 & UART TXD (serial data from WRPC)\\
  \hline
  \hdltablesection{Flash memory SPI interface}\\
  \hline
  spi\_sclk\_o & out & 1 & Flash SPI SCLK\\
  \hline
  spi\_ncs\_o  & out & 1 & Flash SPI $\overline{\mbox{SS}}$\\
  \hline
  spi\_mosi\_o & out & 1 & Flash SPI MOSI\\
  \hline
  spi\_miso\_i & in  & 1 & Flash SPI MISO\\
\end{hdlporttable}

\subsubsection{VFC-HD}
\label{sec:hdl_board_vfchd}

The VFC-HD BSP provides a ready-to-use WRPC wrapper for the
\href{http://www.ohwr.org/projects/vfc-hd}{VFC-HD carrier board}.

The top-level VHDL module is located under: \\\hrefwrpc{board/vfchd/xwrc\_board\_vfchd.vhd}

An alternative top-level VHDL module which only makes use of standard logic for ports and integers
and strings for generic parameters (ideal for instantiation in Verilog-based designs) can be found
under: \\\hrefwrpc{board/vfchd/wrc\_board\_vfchd.vhd}

A VHDL package with the definition of both modules can be found under:
\\\hrefwrpc{board/vfchd/wr\_vfchd\_pkg.vhd}

An example (VHDL) instantiation of this module can be found in the VFC-HD WRPC reference design:
\\\hrefwrpc{top/vfchd\_ref\_design/vfchd\_wr\_ref\_top.vhd}

This section describes the generic parameters and ports which are specific to the VFC-HD BSP.
Parameters and ports common to all BSPs are described in Section~\ref{sec:hdl_board_common}.

\newparagraph{Generic parameters}

\begin{hdlparamtable}
  g\_pcs16\_bit & boolean & false & Altera Arria V FPGAs provide the possibility
  to configure the PCS of the PHY as either 8bit or 16bit. The default is to use
  the 8bit PCS. Currently, 16bit PCS is not supported for Arria V.\\
\end{hdlparamtable}

\newparagraph{Ports}

\begin{hdlporttable}
  \hdltablesection{Clocks and resets}\\
  \hline
  areset\_n\_i & in & 1 & Reset input (active low, can be async)\\
  \hline
  areset\_edge\_n\_i & in & 1 & [optional] Reset input edge sensitive (active
  rising-edge, can be async).\\
  \hline
  clk\_board\_20m\_i & in & 1 & 20MHz clock input from board\\
  \hline
  clk\_board\_125m\_i & in & 1 & 125MHz reference clock input from board\\
  \hline
  \hdltablesection{SPI interface to DACs}\\
  \hline
  dac\_sclk\_o & out & 1 & SPI SCLK, common to both DACs\\
  \hline
  dac\_din\_o & out & 1 & SPI MOSI, common to both DACs\\
  \hline
  dac\_ref\_sync\_n\_o & out & 1 & SPI $\overline{\mbox{SS}}$ for DAC controlling 125MHz oscillator\\
  \hline
  dac\_dmtd\_sync\_n\_o & out & 1 & SPI $\overline{\mbox{SS}}$ for DAC controlling 20MHz oscillator\\
  \hline
  \hdltablesection{SFP interface}\\
  \hline
  sfp\_tx\_o & out & 1 & PHY TX\\
  \hline
  sfp\_rx\_i & in & 1 & PHY RX\\
  \hline
  sfp\_det\_i & in  & 1 & Active high, asserted if all of the following are true:\linebreak
  * SFP is detected (plugged in)\linebreak
  * The part number has been successfully read\\
  \hline
  sfp\_data\_i & in  & 128 & 16 byte SFP vendor Part Number (ASCII encoded, first character byte
  in bits 127 downto 120)\\

\end{hdlporttable}


% ##########################################################################
\newpage
\section{Troubleshooting}
\label{Troubleshooting}

\textbf{My computer hangs on loading spec.ko or programming the FPGA.}

This will occur when you try to load the driver or program the FPGA while your
\textit{spec-vuart} is running and trying to get messages from Virtual-UART's
registers inside the WRPC. Please remember to quit \textit{spec-vuart} before
reloading the driver or programming the FPGA.\\

\noindent \textbf{I want to synthesize WRPC but hdlmake says I don't have the
tools.}

If you have installed the synthesis tool (ISE for SPEC/SVEC or Quartus for
VFC-HD), but you still see the message, please make sure you have correctly set
up your environment. See sections \ref{Setting up Xilinx ISE} and \ref{Setting
up Quartus Prime} for the instructions.\\

\noindent \textbf{WR PTP Core seems to work but the 1-PPS skew on the
oscilloscope between WR Master and WR Slave is more than 1ns.}

Check if the oscilloscope cables you use have the same delays. If not, you
should either change the cables or take the delay difference into account in your
measurements. If that's not the case, please make sure you have the correct
calibration values loaded to both of your devices (see section \ref{Writing
configuration}). If you made your own synthesis of the WRPC, then the default
calibration values are no longer valid and you need to perform the White Rabbit
Calibration procedure\footnote{Use the latest version of the document available
at: \url{http://www.ohwr.org/documents/213}}.

% ##########################################################################
\section{Questions, reporting bugs}
\label{Questions}

If you have found a bug, you have problems with the WR PTP Core or one of the
tools used to build and run it, you can write to our mailing list
\textbf{\textit{white-rabbit-dev@ohwr.org}}


% ##########################################################################


\appendix
\clearpage
\section{WRPC Shell Commands}
\label{WRPC Shell commands}

\footnotesize
\renewcommand\arraystretch{1.5}
\begin{longtable}{  p{7.5cm}  p{7cm} }

  \code{calibration} & (legacy) tries to read t2/4 phase transition value from
    the Flash/EEPROM (in WR Master or GrandMaster mode), or executes the t24p
    calibration procedure and stores its result to the Flash/EEPROM (in WR 
    Slave mode) \\

  \code{config} & prints the dot-config file used to build this instance of
    WRPC and ppsi. It is an optional command, enabled at build time by
    \texttt{CONFIG\_CMD\_CONFIG}. \\

  \code{delays}& \\
  \code{delays <tx> <rx>} & gets or sets the constant delays in frame
    transmission, only available if \texttt{CONFIG\_CMD\_LL}. Delays are
    expressed in picoseconds. \\

  \code{devmem <addr>} &  \\
  \code{devmem <addr> <value>} & reads or writes a 32-bit word from memory
    and/or FPGA registers. Only available if \texttt{CONFIG\_CMD\_LL} is
    set. \\

  \code{diag} & diagnostics for auxiliary user registers in HDL. Only available
    if \texttt{CONFIG\_AUX\_DIAG}.\\

  \code{diag ro <reg\_no>} & reads register <reg\_no> from read-only bank.\\

  \code{diag rw <reg\_no>} & reads register <reg\_no> from read-write bank.\\

  \code{diag w <reg\_no>} & writes register <reg\_no> to read-write bank. \\

  \code{faketemp} &  \\
  \code{faketemp <T1> <T2> <T3>} & reads or sets the three fake temperatures,
    used to test the temperature/syslog mechanism. Available if
    \texttt{CONFIG\_FAKE\_TEMPERATURES} is set. \\

  \code{gui} & starts GUI WRPC monitor \\

  \code{help} & lists available commands in this instance of the WRPC \\

  \code{init add <cmd>} & adds shell command at the end of the initialization
    script \\

  \code{init boot} & executes the script stored in Flash/EEPROM (the same
    action is done automatically when WRPC starts after resetting LM32) \\

  \code{init erase} & erases the initialization script in Flash/EEPROM  \\

  \code{init show} & prints all commands from the script stored in
    Flash/EEPROM \\

  \code{ip get} & \\
  \code{ip set <ip>} & reports or sets the IPv4 address of the WRPC (only
    available if \texttt{CONFIG\_IP} is set at build time \\

  \code{ltest} & \\
  \code{ltest fake <nsecs>} & fakes delay to trigger latency failures (for
    testing). \\

  \code{ltest <secs> <msecs>} & reads or sets the latency-test sending
    interval. See \ref{Latency Test}. Available if
    \texttt{CONFIG\_LATENCY\_PROBE} is set. \\

  \code{ltest quiet} & disables sending latency messages to Syslog. \\

  \code{ltest verbose} & enables sending latency messages to Syslog. \\

  \code{mac getp} & reads the persistent MAC address stored in Flash/EEPROM \\

  \code{mac get} & prints WRPC's MAC address \\

  \code{mac setp <mac>} & stores the persistent MAC address in Flash/EEPROM \\

  \code{mac set <mac>} & sets the MAC address of WRPC \\

  \code{mode gm|master|slave} & (legacy) sets WRPC to operate as Grandmaster
    clock (requires external 10MHz and 1-PPS reference), Master or Slave.
    After setting the mode, \texttt{ptp start} must be re-issued \\

  \code{pll cl <channel>} & checks if SoftPLL is locked for the channel \\

  \code{pll gdac <index>} & gets dac's value \\

  \code{pll gps <channel>} & gets current and target phase shift for the
    channel \\

  \code{pll init <mode> <ref\_channel> <align\_pps>} & manually runs
    \texttt{spll\_init()} function to initialize SoftPll  \\

  \code{pll sdac <index> <val>} & sets the dac \\

  \code{pll sps <channel> <picoseconds>} & sets phase shift for the channel \\

  \code{pll start <channel>} & starts SoftPLL for the channel \\

  \code{pll stop <channel>} & stops SoftPLL for the channel \\

  \code{ps} & prints the list of running tasks (processes) in the CPU. For
    each task you get the number of iterations, the maximum execution time
    (measured with the monotonic clock) and the CPU time consumed (using
    the RT clock) since boot or last reset of values \\

  \code{ps reset} & zeroes the profiling information reported by the \code{ps}
    command \\

  \code{ps max <msecs>} & starts printing all tasks executing longer than
    a given number of miliseconds. Additionally, it triggers printing messages
    if particular task runs longer than ever before. Passing ``\code{0}'' as
    a parameter stops the further printouts. \\

  \code{ptp <e2e|p2p>} & selects PTP delay mechanism: end-to-end or peer-to-peer.
    If configured, you can set \texttt{p2p} mode. Alternatively you can use also
    aliases: \texttt{delay} (instead of \texttt{e2e}) or \texttt{pdelay}
    (instead of \texttt{p2p}).\\

  \code{ptp gm|master|slave} & sets WRPC to operate as Grandmaster clock
    (requires external 10MHz and 1-PPS reference), Master or Slave. After
    setting the mode, \texttt{ptp start} must be re-issued \\

  \code{ptp start} & starts WR PTP daemon \\

  \code{ptp stop} & stops WR PTP daemon \\

  \code{ptp <cmd> <cmd> ...} & you can concatenate several of the above
    subcommands in a single \texttt{ptp} command. With no arguments,
    the command reports the current values.\\

  \code{refresh} & changes the update time period of the gui and stat commands.
    Default period is 1 second. If you set the period to 0, the log message is
    only generated one time. \\

  \code{sdb} & prints devices connected to the Wishbone bus inside WRPC \\

  \code{sdb fs <memtype> <baseadr> <param>} & creates SDBFS image under
    specified \code{<baseadr>} in selected storage depending on \code{<memtype>}
    ({\bf 0} - Flash, {\bf 1} - I2C EEPROM, {\bf 2} - 1-Wire EEPROM). The meaning
    of last parameter \code{<param>} depends on the type of selected storage. It
    is either the sector size in kilobytes (for Flash) or I2C chip address (for
    I2C EEPROM). Command \code{sdb} is available if \texttt{CONFIG\_GENSDBFS}
    is set.\\

  \code{sdb fs 0} & creates SDBFS image in Flash memory. Base address and sector
    size are taken from HDL Syscon registers for SPEC/SVEC boards. If you want
    to use it for custom board, base address and sector size must be specified
    as VHDL generic parameters of the WR PTP Core. Command \code{sdb} is
    available if \texttt{CONFIG\_GENSDBFS} is set.\\

  \code{sdb fse <memtype> <baseadr> <param>} & erases SDBFS image under
    specified \code{<baseadr>} from selected storage depending on
    \code{<memtype>} ({\bf 0} - Flash, {\bf 1} - I2C EEPROM, {\bf 2} - 1-Wire
    EEPROM). The meaning of last parameter \code{<param>} depends on the type
    of selected storage. It is either the sector size in kilobytes (for Flash)
    or I2C chip address (for I2C EEPROM). Command \code{sdb} is available
    if \texttt{CONFIG\_GENSDBFS} is set.\\

  \code{sdb fse 0} & erases SDBFS image from Flash memory. Base address and sector
    size are taken from HDL Syscon registers for SPEC/SVEC boards. If you want
    to use it for custom board, base address and sector size must be specified
    as VHDL generic parameters of the WR PTP Core. Command \code{sdb} is
    available if \texttt{CONFIG\_GENSDBFS} is set.\\

  \code{sfp add <PN> <deltaTx> <deltaRx> <alpha>} & stores calibration
    parameters for SFP to a file in Flash/EEPROM \\

  \code{sfp erase} & erases the SFP database stored in the Flash/EEPROM \\

  \code{sfp match} & prints the ID of a currently used SFP transceiver and
    tries to load the calibration parameters for it \\

  \code{sfp show} & prints all SFP transceivers stored in database \\

  \code{stat} & toggles reporting of loggable statistics. You can pass
    \texttt{on} or \texttt{off} as argument as an alternative to toggling \\

  \code{stat bts} & prints bitslide value for established WR Link, needed by
    the calibration procedure \\

  \code{syslog off} &  \\
  \code{syslog <ipaddr> <macaddr>} & disables or sets your \textit{syslog}
    server. See \ref{Syslog}. Available if \texttt{CONFIG\_SYSLOG} is set. \\

  \code{temp} & reports current temperatures. \\

  \code{time} & prints current time from WRPC \\

  \code{time raw} &  prints current time in a raw format (seconds, nanoseconds) \\

  \code{time setnsec <nsec>} & sets only nanoseconds of the WRPC time \\

  \code{time setsec <sec>} & sets only seconds of the WRPC time (useful for
    setting time in GrandMaster mode, when nanoseconds counter is aligned to
    external 1-PPS and 10 MHz) \\

  \code{time set <sec> <nsec>} & sets WRPC time \\

  \code{verbose <digits>} & sets PPSi verbosity. See the PPSi manual about the
    meaning of the digits (hint: \texttt{verbose 1111} is a good first bet to
    see how the PTP system is working)  \\

  \code{ver} & prints which version of wrpc is running  \\

  \code{vlan} &  \\
  \code{vlan set <n>} & reports or sets the active vlan used for
    sending/receiving network frames. The command exists only if
    \texttt{CONFIG\_VLAN} is set.\\

  \code{vlan off} & disables vlans. Available if \texttt{CONFIG\_VLAN} is
    set. \\

  \code{w1} & lists onewire devices on the bus \\

  \code{w1w <offset> <byte> [<byte> ...]}&  \\

  \code{w1r <offset> <len>} & if \texttt{CONFIG\_W1} is set and a OneWire
    EEPROM exists, write and read data. For writing, \texttt{byte} values are
    decimal \\

\end{longtable}
\renewcommand\arraystretch{1}



% ##########################################################################
\clearpage
\section{WRPC GUI elements}
\label{WRPC GUI elements}

\footnotesize
\renewcommand\arraystretch{1.5}
\begin{longtable}{  p{4.5cm}  p{10cm} }

  \code{TAI Time:} & Current state of device's local clock \\

  \code{RX:} / \code{TX:} & Rx/Tx packets counters\\

  \code{IPv4:} & IP address; also whether it is statically configured or
    acquired via BOOTP (and the status of BOOTP) \\

  \code{mode:} & Operation mode of the WR PTP Core - \code{<WR Master,
    WR Slave>}\\

  \code{<Locked, NoLock>} & SoftPLL lock state\\

  \code{<Calibrated, Uncalibrated>} & Status of PHY calibration; not used
    anymore\\

  \code{PTP status:} & Current state of PTP state machine\\

  \code{Servo state:} & Current state of WR servo state machine -
    \code{<Uninitialized, SYNC\_SEC, SYNC\_NSEC, SYNC\_PHASE, TRACK\_PHASE>}\\

  \code{Phase tracking:} & Is phase tracking enabled when WR Slave is
    synchronized to WR Master - \code{<ON, OFF>}\\

  \code{Aux clock <N> status:} & Statuses of AUX clocks; one status line per
    available AUX clock; can contain <enabled> and <locked> \\

%   \code{Synchronization source:} & network interface name from which WR
% daemon gets synchronization - \code{<wru1>}\\

  \code{Round-trip time (mu):} & Round-trip delay in picoseconds
    ($delay_{MM}$)($delay_{MM}$)\\

  \code{Master-slave delay:} & Estimated one-way (master to slave) link
    delay ($delay_{MS}$)\\

  \code{Master PHY delays:} & Transmission/reception delays of WR 
    Master's hardware ($\Delta_{TXM}, \Delta_{RXM}$)\\

  \code{Slave PHY delays:} & Transmission/reception delays of WR Slave's
    hardware ($\Delta_{TXS}, \Delta_{RXS}$)\\

  \code{Total link asymmetry:} & WR link asymmetry calculated as
    $delay_{MM} - 2 \cdot delay_{MS}$\\

  \code{Cable rtt delay:} & Round-trip fiber latency\\

  \code{Clock offset:} & Slave to Master offset calculated by PTP daemon
  ($ offset_{MS} $)\\

  \code{Phase setpoint:} & Current Slave's clock phase shift value\\

  \code{Skew:} & The difference between current and previous estimated
    one-way link delay\\

  \code{Update counter:} & The value of a counter incremented every time
    the WR servo is updated\\

\end{longtable}
\renewcommand\arraystretch{1}


% ##########################################################################
\clearpage
\section{Other ways to write SDBFS image to your Flash memory}
\label{appendix:writing_sdbfs}

\subsection{Writing SDBFS image through PCIe bus}
To write SDBFS filesystem image to the Flash memory of a hosted SPEC card, you
can use the \texttt{flash-write} tool available in the \emph{spec-sw} drivers
package.

First, please download the SDBFS image from \textit{ohwr.org}:
\begin{lstlisting}
$ wget http://www.ohwr.org/attachments/download/4060/sdbfs-flash.bin
\end{lstlisting}
It contains all the files required by the core. They are empty, but have to
exist in the SDBFS structure to be filled later from the WR PTP Core shell or
SNMP.\\

Before calling the tool, you need to have SPEC drivers loaded in your system:
\begin{lstlisting}
$ cd <your_location>/spec-sw
$ sudo insmod fmc-bus/kernel/fmc.ko
$ sudo insmod kernel/spec.ko
\end{lstlisting}

To write the filesystem image to flash, please execute the following
command:
\begin{lstlisting}
$ sudo tools/flash-write -c 0x0 0 1507712 < <your_location>/sdbfs-flash.bin
\end{lstlisting}

\noindent\textbf{Note:} If you have more than one SPEC board in your computer,
you can use \code{-b} parameter which takes the PCI bus address of the card you
want to program. This can be found in the list of the PCI devices installed in
the system (\code{lspci}).

\subsection{Writing SDBFS image in standalone configuration}

If you use SPEC board in a host-less environment, or you use custom
hardware and SPEC drivers/tools cannot be used, there is still a
possibility of writing SDBFS through Xilinx JTAG.

\vspace{1em}
In the case when you want to run on the the SPEC a reference bitstream provided with a stable
WRPC release, you can simply program your Flash with \textit{spec\_top.mcs}
provided with the release binaries using for example Xilinx ISE Impact tool.
This \textit{mcs} file already includes both SDBFS image and FPGA bitstream.

In the case when you want to run a custom gateware or you have a custom hardware, you can 
download a standalone SDBFS image:
\begin{lstlisting}
$ wget http://www.ohwr.org/attachments/download/4558/sdbfs-standalone-160812.bin
\end{lstlisting}
and generate a custom \textit{*.mcs} file with your own FPGA bitstream. You should
use the following layout:
\begin{longtable}{  l  l }
\code{0x000000} & your FPGA bitstream \\
\code{0x170000} & SDBFS image\\
\end{longtable}

For example, to generate the \textit{*.mcs} file for M25P32 Flash on SPEC, the
following \textit{promgen} parameters should be used:
\begin{lstlisting}
promgen -w -spi -p mcs -c FF -s 32768 -u 0 <your_bitstream>.bit \
-bd sdbfs-standalone-160812.bin start 0x170000 -o output.mcs
\end{lstlisting}

After that, you can use the Xilinx JTAG cable and Xilinx ISE Impact tool to
write your \textit{output.mcs} file to the Flash memory.
\newpage
% ##########################################################################
\section{wrpc-dump with Older WRPC Binaries}
\label{wrpc-dump with Older WRPC Binaries}

The tool has another working mode, that you can use with older
\textit{wrpc} builds, where the special table is missing (but please be
aware that the data structures may have changed slightly).  In this
mode, you provide the address of the data structure and its name. It
supports the names \textit{pll}, \textit{fifo} (the circular log described
above), \textit{ppg}, \textit{ppi}, \textit{servo\_state}, \textit{ds} and
\textit{stats}. The \textit{ds} name
refers to the PTP data sets, and for this the pointer needed is
\texttt{ppg} (ppsi global data).
This is an example:

\begin{lstlisting}
   $ ./tools/wrpc-dump dump-file 00015798 servo_state

   servo_state at 0x15798
           if_name:                   ""
           flags:                     1
           state:                     4
           delta_tx_m:                10
           delta_rx_m:                174410
           delta_tx_s:                0
           delta_rx_s:                3200
           fiber_fix_alpha:           73622176
           clock_period_ps:           8000
           t1:                        correct 1: 1448628309.390213200:0000
           t2:                        correct 1: 1448628309.390213520:2921
           t3:                        correct 1: 1448628310.419072616:0000
           t4:                        correct 1: 1448628310.419073104:6041
   [...]
\end{lstlisting}

The address is specified as a hex number, and can be retrieved running
``\texttt{lm32-elf-nm  wrc.elf}''.

Please note that the LM32 soft-core processor inside the WRPC has a different
endianness than your host machine, thus a special endian conversion is needed.
With current \textit{wrpc} it is autodetected, but if you dump an older binary
you'll need to set the
\texttt{WRPC\_SPEC} environment variable (to any value you like) to properly
access the PCI memory or a dump taken from PCI:
\begin{lstlisting}
   $ export WRPC_SPEC=y
\end{lstlisting}
or
\begin{lstlisting}
   $ WRPC_SPEC=y ./tools/wrpc-dump dump-file 00015798 servo_state
\end{lstlisting}
This is not needed if the dump is retrieved using Etherbone.

% ##########################################################################
\newpage
\section{Adding new objects to the SNMP}
\label{Adding new objects to the SNMP}

The \textit{Mini SNMP responder} can be easily expanded to export new objects.
Values of new objects can come from WRPC's variables or other HDL modules
as long as there is a proper interface to the WRPC to read these values.

This section contains the instruction on how to export new objects with
the given variables' content.

The \textit{Mini SNMP responder} internally divides all OIDs into two parts.
The first part is called \textit{limb}. The \textit{limb} part of the incoming OID is
matched by a function \texttt{snmp\_respond}, with the defined \textit{limb} parts of OIDs
in the structure \texttt{oid\_limb\_array}.
When the \textit{limb} part is matched then the corresponding function from
the structure \texttt{oid\_limb\_array} is called to try to match the second part of
OID (the \textit{twig} part).

\begin{sloppypar} % to prevent \texttt{} from going to the margine
The example below adds to the \textit{Mini SNMP responder} an \texttt{int32\_t} variable
(\texttt{example\_i32var}) with OID \texttt{1.3.6.1.4.1.96.102.1.1.0} and a string
(\texttt{example\_string}) with OID \texttt{1.3.6.1.4.1.96.102.1.2.0}.
Before assigning new OIDs in your projects please contact the maintainer of
\texttt{wrpc-sw} repo to avoid conflicts.
\end{sloppypar}

\begin{itemize*}
\item First declare \texttt{example\_i32var} and \texttt{example\_string}:
\begin{lstlisting}
 static int32_t example_i32var;
 static char example_string[] = "test string";
\end{lstlisting}

\item Define the \textit{limb} part of the OID:
\begin{lstlisting}
 static uint8_t oid_wrpcExampleGroup[] = {0x2B,6,1,4,1,96,101,99};
\end{lstlisting}

\item Define the \textit{twig} part of the OID:
\begin{lstlisting}
 static uint8_t oid_wrpcExampleV1[] = {1,0};
 static uint8_t oid_wrpcExampleV2[] = {2,0};
\end{lstlisting}

\item Add a group definition to the \texttt{struct snmp\_oid\_limb oid\_limb\_array}.
      Please note that this structure has to be sorted by ascending OIDs.
\begin{lstlisting}
OID_LIMB_FIELD(oid_wrpcExampleGroup, func_group, oid_array_wrpcExampleGroup),
\end{lstlisting}
The macro \texttt{OID\_LIMB\_FIELD} takes the following arguments:
\begin{itemize*}
   \item \texttt{oid\_wrpcExampleGroup} -- an array with the \textit{limb} part of the
         OID
   \item \texttt{func\_group} -- a function to be called when the \textit{limb} part of
         the OID is matched; this function will try to match the \textit{twig} part
         of the OID within a table or a group.
   \item \texttt{oid\_array\_wrpcExampleGroup} -- an array of \textit{twig} parts of OIDs
\end{itemize*}
\item Declare a previously used \texttt{oid\_wrpcExampleGroup}. Please note that
      this structure has to be sorted by ascending \textit{twig} part of OIDs.
\begin{lstlisting}
 static struct snmp_oid oid_array_wrpcExampleGroup[] = {
   OID_FIELD_VAR(oid_wrpcExampleV1, get_p, set_p, ASN_INTEGER,   &example_i32var),
   OID_FIELD_VAR(oid_wrpcExampleV2, get_p, set_p, ASN_OCTET_STR, &example_string),
   { 0, }
 };
\end{lstlisting}
The macro \texttt{OID\_FIELD\_VAR} takes the following arguments:
\begin{itemize*}
   \item \texttt{oid\_wrpcExampleV1} -- an array with \textit{twig} part of the OID
   \item \texttt{get\_p} (or \texttt{get\_pp)} -- a function to be called when \textit{twig}
         part of the OID is matched for SNMP GET requests;
   \item \texttt{set\_p} (or \texttt{set\_pp)} -- a function to be called when a \textit{twig}
         part of the OID is matched for SNMP SET requests; if no SET
         functionality is planned, please use NULL
   \item \texttt{ASN\_INTEGER, ASN\_OCTET\_STR} -- type of the OID; please
         check the source for other possible types
   \item \texttt{\&example\_i32var, \&example\_string} -- addresses to the data in
         memory
\end{itemize*}
In the case when the address of variable is not known at boot, it is possible to define
a pointer to the variable which will be initialized (e.g. in the \texttt{snmp\_init}
the function) at the boot time. In that case function \texttt{get\_pp} (\texttt{set\_pp}) has
to be used instead of \texttt{get\_p} (\texttt{set\_p}). For variables that are a part of
a structure and have to be accessed via a pointer, a macro \texttt{OID\_FIELD\_STRUCT}
is available.

For more complex extraction of variables or run-time value corrections,
it is possible to use a custom \textit{get} function. It is possible to pass
a constant number to the custom function instead of an address. For example:
\begin{lstlisting}
 OID_FIELD_VAR(oid_wrpcPtpServoUpdateTime, get_servo, NO_SET, ASN_COUNTER64, \
               SERVO_UPDATE_TIME),
\end{lstlisting}

\end{itemize*}
Perform a \texttt{snmpwalk} to get new OIDs:
\begin{lstlisting}
 $ snmpwalk -On $SNMP_OPT 1.3.6.1.4.1.96.102.1
 .1.3.6.1.4.1.96.102.1.1.0 = INTEGER: 123432
 .1.3.6.1.4.1.96.102.1.2.0 = STRING: "test string"
 End of MIB
\end{lstlisting}
Trying to set too long string into the \texttt{example\_string} results in an error:
\begin{lstlisting}
 $ snmpset -On $SNMP_OPT 1.3.6.1.4.1.96.102.1.2.0 s "new long string"
 Error in packet.
 Reason: (badValue) The value given has the wrong type or length.
 Failed object: .1.3.6.1.4.1.96.102.1.2.0
\end{lstlisting}
A short enough (not longer than defined \texttt{"test string"}) value succeeds:
\begin{lstlisting}
 $ snmpset -On $SNMP_OPT 1.3.6.1.4.1.96.102.1.2.0 s "new value12"
 .1.3.6.1.4.1.96.102.1.2.0 = STRING: "new value12"
\end{lstlisting}
Set 999 to the \texttt{example\_i32var}:
\begin{lstlisting}
$ snmpset -On $SNMP_OPT 1.3.6.1.4.1.96.102.1.1.0 i 999
.1.3.6.1.4.1.96.102.1.1.0 = INTEGER: 999
\end{lstlisting}
Perform \texttt{snmpwalk} to verify changes:
\begin{lstlisting}
$ snmpwalk -On $SNMP_OPT 1.3.6.1.4.1.96.102.1
.1.3.6.1.4.1.96.102.1.1.0 = INTEGER: 999
.1.3.6.1.4.1.96.102.1.2.0 = STRING: "new value12"
End of MIB
\end{lstlisting}

% --------------------------------------------------------------------------
\subsection{Mini SNMP responder's tests}
\label{Mini SNMP responder's tests}

In the \texttt{wrpc-sw} repo, automatic tests are available to validate the \textit{Mini
SNMP responder} implementation. These tests are placed in the \texttt{test/snmp}
directory.
They use the \texttt{bats} framework (\url{https://github.com/sstephenson/bats}).

To run these tests, please go to \texttt{test/snmp} directory, then set
\texttt{TARGET\_IP} environment varaible to the IP of your target board, then type
\texttt{make}. For example:
\begin{lstlisting}
 $ TARGET_IP=192.168.1.20 make
 bats/libexec/bats snmp_tests_*.bats
   Host up 192.168.1.20
   Check the presence of snmpget
   Check the presence of snmpgetnext
   Check the presence of snmpwalk
   Check the presence of snmpset
   snmpget existing oid 1.3.6.1.4.1.96.101.1.1.1.0
 [...]
   snmpwalk 1.3.6.1.4.1.95
   snmpwalk 1.3.6 count all OIDs
 100 tests, 0 failures
\end{lstlisting}
On the left of each test there will be a tick symbol shown or an \texttt{x}
depending of the test's result (not included in the example above).

Be aware that it might be necessary to clone the \texttt{bats} repo first.
\texttt{make} command will inform whether this is needed.

In the case when the number of OIDs changes please correct variable \texttt{TOTAL\_NUM\_OIDS}
in file \texttt{snmp\_test\_config.bash}.

These tests have to be executed after any changes are made to the \textit{Mini SNMP
responder}.

% --------------------------------------------------------------------------
\newpage
\section{Wishbone Memory Maps}
\label{sec:wb_mem_maps}

This appendix provides a register map of Wishbone modules that are exposed to
external applications/HDL core. In the table below you can see the list of
modules available on the External Wishbone interface of the WR PTP Core. Please
bare in mind that WR Streamers Diagnostics module is available only when
\tts{STREAMERS} external fabric mode is selected.
\begin{center}
\begin{tabular}{|l|l|}
  \hline
  module name & base address\\
  \hline \hline
  WR Core Diagnostics & 0x20800\\
  WR Streamers Diagnostics & 0x20700\\
  \hline
\end{tabular}
\end{center}

\subsection{WR Core Diagnostics}
\label{subsec:wbgen:wrc_diags}
[version 0x00000001]\\
Diagnostics information accessible via WR
\subsubsection{Memory map summary}
\rowcolors{2}{gray!25}{white}
\resizebox{\textwidth}{!}{
\begin{tabular}{|l|l|l|l|l|}
\rowcolor{RoyalPurple}
\color{white} SW Offset & \color{white} Type & \color{white} Name &
\color{white} HW prefix & \color{white} C prefix\\
0x0& REG & Version register & wrc\_diags\_ver & VER\\
0x4& REG & Ctrl & wrc\_diags\_ctrl & CTRL\\
0x8& REG & WRPC Diag: servo status & wrc\_diags\_wdiag\_sstat & WDIAG\_SSTAT\\
0xc& REG & WRPC Diag: Port status & wrc\_diags\_wdiag\_pstat & WDIAG\_PSTAT\\
0x10& REG & WRPC Diag: PTP state & wrc\_diags\_wdiag\_ptpstat & WDIAG\_PTPSTAT\\
0x14& REG & WRPC Diag: AUX state & wrc\_diags\_wdiag\_astat & WDIAG\_ASTAT\\
0x18& REG & WRPC Diag: Tx PTP Frame cnts & wrc\_diags\_wdiag\_txfcnt & WDIAG\_TXFCNT\\
0x1c& REG & WRPC Diag: Rx PTP Frame cnts & wrc\_diags\_wdiag\_rxfcnt & WDIAG\_RXFCNT\\
0x20& REG & WRPC Diag:local time [msb of s] & wrc\_diags\_wdiag\_sec\_msb & WDIAG\_SEC\_MSB\\
0x24& REG & WRPC Diag: local time [lsb of s] & wrc\_diags\_wdiag\_sec\_lsb & WDIAG\_SEC\_LSB\\
0x28& REG & WRPC Diag: local time [ns] & wrc\_diags\_wdiag\_ns & WDIAG\_NS\\
0x2c& REG & WRPC Diag: Round trip (mu) [msb of ps] & wrc\_diags\_wdiag\_mu\_msb & WDIAG\_MU\_MSB\\
0x30& REG & WRPC Diag: Round trip (mu) [lsb of ps] & wrc\_diags\_wdiag\_mu\_lsb & WDIAG\_MU\_LSB\\
0x34& REG & WRPC Diag: Master-slave delay (dms) [msb of ps] & wrc\_diags\_wdiag\_dms\_msb & WDIAG\_DMS\_MSB\\
0x38& REG & WRPC Diag: Master-slave delay (dms) [lsb of ps] & wrc\_diags\_wdiag\_dms\_lsb & WDIAG\_DMS\_LSB\\
0x3c& REG & WRPC Diag: Total link asymmetry [ps] & wrc\_diags\_wdiag\_asym & WDIAG\_ASYM\\
0x40& REG & WRPC Diag: Clock offset (cko) [ps] & wrc\_diags\_wdiag\_cko & WDIAG\_CKO\\
0x44& REG & WRPC Diag: Phase setpoint (setp) [ps] & wrc\_diags\_wdiag\_setp & WDIAG\_SETP\\
0x48& REG & WRPC Diag: Update counter (ucnt) & wrc\_diags\_wdiag\_ucnt & WDIAG\_UCNT\\
0x4c& REG & WRPC Diag: Board temperature [C degree] & wrc\_diags\_wdiag\_temp & WDIAG\_TEMP\\
\hline
\end{tabular}
}

\subsubsection{Register description}
\paragraph*{Version register}\mbox{}\\\vskip 6pt
\rowcolors{1}{white}{white}
\begin{tabular}{l l }
{\bf HW prefix:}  & wrc\_diags\_ver\\
{\bf HW address:}  & 0x0\\
{\bf SW prefix:}  & VER\\
{\bf SW offset:}  & 0x0\\
\end{tabular}


\vspace{12pt}
\noindent
\resizebox{\textwidth}{!}{
\begin{tabular}{>{\centering\arraybackslash}p{1.5cm} >{\centering\arraybackslash}p{1.5cm} >{\centering\arraybackslash}p{1.5cm} >{\centering\arraybackslash}p{1.5cm} >{\centering\arraybackslash}p{1.5cm} >{\centering\arraybackslash}p{1.5cm} >{\centering\arraybackslash}p{1.5cm} >{\centering\arraybackslash}p{1.5cm} }
31 & 30 & 29 & 28 & 27 & 26 & 25 & 24\\
\hline
\multicolumn{8}{|c|}{\cellcolor{RoyalPurple!25}ID[31:24]}\\
\hline
23 & 22 & 21 & 20 & 19 & 18 & 17 & 16\\
\hline
\multicolumn{8}{|c|}{\cellcolor{RoyalPurple!25}ID[23:16]}\\
\hline
15 & 14 & 13 & 12 & 11 & 10 & 9 & 8\\
\hline
\multicolumn{8}{|c|}{\cellcolor{RoyalPurple!25}ID[15:8]}\\
\hline
7 & 6 & 5 & 4 & 3 & 2 & 1 & 0\\
\hline
\multicolumn{8}{|c|}{\cellcolor{RoyalPurple!25}ID[7:0]}\\
\hline
\end{tabular}
}

\begin{itemize}
\item \begin{small}
{\bf 
ID
} [\emph{read/write}]: Version identifier
\\
Version identifier for the peripheral
\end{small}
\end{itemize}
\paragraph*{Ctrl}\mbox{}\\\vskip 6pt
\rowcolors{1}{white}{white}
\begin{tabular}{l l }
{\bf HW prefix:}  & wrc\_diags\_ctrl\\
{\bf HW address:}  & 0x1\\
{\bf SW prefix:}  & CTRL\\
{\bf SW offset:}  & 0x4\\
\end{tabular}


\vspace{12pt}
\noindent
\resizebox{\textwidth}{!}{
\begin{tabular}{>{\centering\arraybackslash}p{1.5cm} >{\centering\arraybackslash}p{1.5cm} >{\centering\arraybackslash}p{1.5cm} >{\centering\arraybackslash}p{1.5cm} >{\centering\arraybackslash}p{1.5cm} >{\centering\arraybackslash}p{1.5cm} >{\centering\arraybackslash}p{1.5cm} >{\centering\arraybackslash}p{1.5cm} }
31 & 30 & 29 & 28 & 27 & 26 & 25 & 24\\
\hline
\multicolumn{1}{|c}{-} & - & - & - & - & - & - & \multicolumn{1}{c|}{-}\\
\hline
23 & 22 & 21 & 20 & 19 & 18 & 17 & 16\\
\hline
\multicolumn{1}{|c}{-} & - & - & - & - & - & - & \multicolumn{1}{c|}{-}\\
\hline
15 & 14 & 13 & 12 & 11 & 10 & 9 & 8\\
\hline
\multicolumn{1}{|c}{-} & - & - & - & - & - & - & \multicolumn{1}{|c|}{\cellcolor{RoyalPurple!25}DATA\_SNAPSHOT}\\
\hline
7 & 6 & 5 & 4 & 3 & 2 & 1 & 0\\
\hline
\multicolumn{1}{|c}{-} & - & - & - & - & - & - & \multicolumn{1}{|c|}{\cellcolor{RoyalPurple!25}DATA\_VALID}\\
\hline
\end{tabular}
}

\begin{itemize}
\item \begin{small}
{\bf 
DATA\_VALID
} [\emph{read-only}]: WR DIAG data valid
\\
0: valid\\                     1:transcient
\end{small}
\item \begin{small}
{\bf 
DATA\_SNAPSHOT
} [\emph{read/write}]: WR DIAG data snapshot
\\
1: snapshot data (data in registers will not change aveter VALID becomes true)
\end{small}
\end{itemize}
\paragraph*{WRPC Diag: servo status}\mbox{}\\\vskip 6pt
\rowcolors{1}{white}{white}
\begin{tabular}{l l }
{\bf HW prefix:}  & wrc\_diags\_wdiag\_sstat\\
{\bf HW address:}  & 0x2\\
{\bf SW prefix:}  & WDIAG\_SSTAT\\
{\bf SW offset:}  & 0x8\\
\end{tabular}


\vspace{12pt}
\noindent
\resizebox{\textwidth}{!}{
\begin{tabular}{>{\centering\arraybackslash}p{1.5cm} >{\centering\arraybackslash}p{1.5cm} >{\centering\arraybackslash}p{1.5cm} >{\centering\arraybackslash}p{1.5cm} >{\centering\arraybackslash}p{1.5cm} >{\centering\arraybackslash}p{1.5cm} >{\centering\arraybackslash}p{1.5cm} >{\centering\arraybackslash}p{1.5cm} }
31 & 30 & 29 & 28 & 27 & 26 & 25 & 24\\
\hline
\multicolumn{1}{|c}{-} & - & - & - & - & - & - & \multicolumn{1}{c|}{-}\\
\hline
23 & 22 & 21 & 20 & 19 & 18 & 17 & 16\\
\hline
\multicolumn{1}{|c}{-} & - & - & - & - & - & - & \multicolumn{1}{c|}{-}\\
\hline
15 & 14 & 13 & 12 & 11 & 10 & 9 & 8\\
\hline
\multicolumn{1}{|c}{-} & - & - & - & \multicolumn{4}{|c|}{\cellcolor{RoyalPurple!25}SERVOSTATE[3:0]}\\
\hline
7 & 6 & 5 & 4 & 3 & 2 & 1 & 0\\
\hline
\multicolumn{1}{|c}{-} & - & - & - & - & - & - & \multicolumn{1}{|c|}{\cellcolor{RoyalPurple!25}WR\_MODE}\\
\hline
\end{tabular}
}

\begin{itemize}
\item \begin{small}
{\bf 
WR\_MODE
} [\emph{read-only}]: WR valid
\\
0: not valid\\                     1:valid
\end{small}
\item \begin{small}
{\bf 
SERVOSTATE
} [\emph{read-only}]: Servo State
\\
0: Uninitialized\\                     1: SYNC\_NSEC\\                     2: SYNC\_TAI\\                     3: SYNC\_PHASE\\                     4: TRACK\_PHASE\\                     5: WAIT\_OFFSET\_STABLE
\end{small}
\end{itemize}
\paragraph*{WRPC Diag: Port status}\mbox{}\\\vskip 6pt
\rowcolors{1}{white}{white}
\begin{tabular}{l l }
{\bf HW prefix:}  & wrc\_diags\_wdiag\_pstat\\
{\bf HW address:}  & 0x3\\
{\bf SW prefix:}  & WDIAG\_PSTAT\\
{\bf SW offset:}  & 0xc\\
\end{tabular}


\vspace{12pt}
\noindent
\resizebox{\textwidth}{!}{
\begin{tabular}{>{\centering\arraybackslash}p{1.5cm} >{\centering\arraybackslash}p{1.5cm} >{\centering\arraybackslash}p{1.5cm} >{\centering\arraybackslash}p{1.5cm} >{\centering\arraybackslash}p{1.5cm} >{\centering\arraybackslash}p{1.5cm} >{\centering\arraybackslash}p{1.5cm} >{\centering\arraybackslash}p{1.5cm} }
31 & 30 & 29 & 28 & 27 & 26 & 25 & 24\\
\hline
\multicolumn{1}{|c}{-} & - & - & - & - & - & - & \multicolumn{1}{c|}{-}\\
\hline
23 & 22 & 21 & 20 & 19 & 18 & 17 & 16\\
\hline
\multicolumn{1}{|c}{-} & - & - & - & - & - & - & \multicolumn{1}{c|}{-}\\
\hline
15 & 14 & 13 & 12 & 11 & 10 & 9 & 8\\
\hline
\multicolumn{1}{|c}{-} & - & - & - & - & - & - & \multicolumn{1}{c|}{-}\\
\hline
7 & 6 & 5 & 4 & 3 & 2 & 1 & 0\\
\hline
\multicolumn{1}{|c}{-} & - & - & - & - & - & \multicolumn{1}{|c|}{\cellcolor{RoyalPurple!25}LOCKED} & \multicolumn{1}{|c|}{\cellcolor{RoyalPurple!25}LINK}\\
\hline
\end{tabular}
}

\begin{itemize}
\item \begin{small}
{\bf 
LINK
} [\emph{read-only}]: Link Status
\\
0: link down\\                     1: link up
\end{small}
\item \begin{small}
{\bf 
LOCKED
} [\emph{read-only}]: PLL Locked
\\
0: not locked\\                     1: locked
\end{small}
\end{itemize}
\paragraph*{WRPC Diag: PTP state}\mbox{}\\\vskip 6pt
\rowcolors{1}{white}{white}
\begin{tabular}{l l }
{\bf HW prefix:}  & wrc\_diags\_wdiag\_ptpstat\\
{\bf HW address:}  & 0x4\\
{\bf SW prefix:}  & WDIAG\_PTPSTAT\\
{\bf SW offset:}  & 0x10\\
\end{tabular}


\vspace{12pt}
\noindent
\resizebox{\textwidth}{!}{
\begin{tabular}{>{\centering\arraybackslash}p{1.5cm} >{\centering\arraybackslash}p{1.5cm} >{\centering\arraybackslash}p{1.5cm} >{\centering\arraybackslash}p{1.5cm} >{\centering\arraybackslash}p{1.5cm} >{\centering\arraybackslash}p{1.5cm} >{\centering\arraybackslash}p{1.5cm} >{\centering\arraybackslash}p{1.5cm} }
31 & 30 & 29 & 28 & 27 & 26 & 25 & 24\\
\hline
\multicolumn{1}{|c}{-} & - & - & - & - & - & - & \multicolumn{1}{c|}{-}\\
\hline
23 & 22 & 21 & 20 & 19 & 18 & 17 & 16\\
\hline
\multicolumn{1}{|c}{-} & - & - & - & - & - & - & \multicolumn{1}{c|}{-}\\
\hline
15 & 14 & 13 & 12 & 11 & 10 & 9 & 8\\
\hline
\multicolumn{1}{|c}{-} & - & - & - & - & - & - & \multicolumn{1}{c|}{-}\\
\hline
7 & 6 & 5 & 4 & 3 & 2 & 1 & 0\\
\hline
\multicolumn{8}{|c|}{\cellcolor{RoyalPurple!25}PTPSTATE[7:0]}\\
\hline
\end{tabular}
}

\begin{itemize}
\item \begin{small}
{\bf 
PTPSTATE
} [\emph{read-only}]: PTP State
\\
0: NONE\\                     1: PPS\_INITIALIZING\\                     2: PPS\_FAULTY\\                     3: disabled\\                     4: PPS\_LISTENING\\                     5: PPS\_PRE\_MASTER\\                     6: PPS\_MASTER\\                     7: PPS\_PASSIVE\\                     8: PPS\_UNCALIBRATED\\                     9: PPS\_SLAVE\\                     100-116: WR STATES\\                     see: ppsi/proto-ext-whiterabbit/wr-constants.h\\                          ppsi/include/ppsi/ieee1588\_types.h
\end{small}
\end{itemize}
\paragraph*{WRPC Diag: AUX state}\mbox{}\\\vskip 6pt
\rowcolors{1}{white}{white}
\begin{tabular}{l l }
{\bf HW prefix:}  & wrc\_diags\_wdiag\_astat\\
{\bf HW address:}  & 0x5\\
{\bf SW prefix:}  & WDIAG\_ASTAT\\
{\bf SW offset:}  & 0x14\\
\end{tabular}


\vspace{12pt}
\noindent
\resizebox{\textwidth}{!}{
\begin{tabular}{>{\centering\arraybackslash}p{1.5cm} >{\centering\arraybackslash}p{1.5cm} >{\centering\arraybackslash}p{1.5cm} >{\centering\arraybackslash}p{1.5cm} >{\centering\arraybackslash}p{1.5cm} >{\centering\arraybackslash}p{1.5cm} >{\centering\arraybackslash}p{1.5cm} >{\centering\arraybackslash}p{1.5cm} }
31 & 30 & 29 & 28 & 27 & 26 & 25 & 24\\
\hline
\multicolumn{1}{|c}{-} & - & - & - & - & - & - & \multicolumn{1}{c|}{-}\\
\hline
23 & 22 & 21 & 20 & 19 & 18 & 17 & 16\\
\hline
\multicolumn{1}{|c}{-} & - & - & - & - & - & - & \multicolumn{1}{c|}{-}\\
\hline
15 & 14 & 13 & 12 & 11 & 10 & 9 & 8\\
\hline
\multicolumn{1}{|c}{-} & - & - & - & - & - & - & \multicolumn{1}{c|}{-}\\
\hline
7 & 6 & 5 & 4 & 3 & 2 & 1 & 0\\
\hline
\multicolumn{8}{|c|}{\cellcolor{RoyalPurple!25}AUX[7:0]}\\
\hline
\end{tabular}
}

\begin{itemize}
\item \begin{small}
{\bf 
AUX
} [\emph{read-only}]: AUX channel
\\
A vector of bits, one bit per channel\\                     0: not valid\\                     1:valid
\end{small}
\end{itemize}
\paragraph*{WRPC Diag: Tx PTP Frame cnts}\mbox{}\\\vskip 6pt
\rowcolors{1}{white}{white}
\begin{tabular}{l l }
{\bf HW prefix:}  & wrc\_diags\_wdiag\_txfcnt\\
{\bf HW address:}  & 0x6\\
{\bf SW prefix:}  & WDIAG\_TXFCNT\\
{\bf SW offset:}  & 0x18\\
\end{tabular}

\vspace{12pt}
Number of transmitted PTP Frames

\vspace{12pt}
\noindent
\resizebox{\textwidth}{!}{
\begin{tabular}{>{\centering\arraybackslash}p{1.5cm} >{\centering\arraybackslash}p{1.5cm} >{\centering\arraybackslash}p{1.5cm} >{\centering\arraybackslash}p{1.5cm} >{\centering\arraybackslash}p{1.5cm} >{\centering\arraybackslash}p{1.5cm} >{\centering\arraybackslash}p{1.5cm} >{\centering\arraybackslash}p{1.5cm} }
31 & 30 & 29 & 28 & 27 & 26 & 25 & 24\\
\hline
\multicolumn{8}{|c|}{\cellcolor{RoyalPurple!25}WDIAG\_TXFCNT[31:24]}\\
\hline
23 & 22 & 21 & 20 & 19 & 18 & 17 & 16\\
\hline
\multicolumn{8}{|c|}{\cellcolor{RoyalPurple!25}WDIAG\_TXFCNT[23:16]}\\
\hline
15 & 14 & 13 & 12 & 11 & 10 & 9 & 8\\
\hline
\multicolumn{8}{|c|}{\cellcolor{RoyalPurple!25}WDIAG\_TXFCNT[15:8]}\\
\hline
7 & 6 & 5 & 4 & 3 & 2 & 1 & 0\\
\hline
\multicolumn{8}{|c|}{\cellcolor{RoyalPurple!25}WDIAG\_TXFCNT[7:0]}\\
\hline
\end{tabular}
}

\begin{itemize}
\item \begin{small}
{\bf 
WDIAG\_TXFCNT
} [\emph{read-only}]: Data
\end{small}
\end{itemize}
\paragraph*{WRPC Diag: Rx PTP Frame cnts}\mbox{}\\\vskip 6pt
\rowcolors{1}{white}{white}
\begin{tabular}{l l }
{\bf HW prefix:}  & wrc\_diags\_wdiag\_rxfcnt\\
{\bf HW address:}  & 0x7\\
{\bf SW prefix:}  & WDIAG\_RXFCNT\\
{\bf SW offset:}  & 0x1c\\
\end{tabular}

\vspace{12pt}
Number of received PTP Frames

\vspace{12pt}
\noindent
\resizebox{\textwidth}{!}{
\begin{tabular}{>{\centering\arraybackslash}p{1.5cm} >{\centering\arraybackslash}p{1.5cm} >{\centering\arraybackslash}p{1.5cm} >{\centering\arraybackslash}p{1.5cm} >{\centering\arraybackslash}p{1.5cm} >{\centering\arraybackslash}p{1.5cm} >{\centering\arraybackslash}p{1.5cm} >{\centering\arraybackslash}p{1.5cm} }
31 & 30 & 29 & 28 & 27 & 26 & 25 & 24\\
\hline
\multicolumn{8}{|c|}{\cellcolor{RoyalPurple!25}WDIAG\_RXFCNT[31:24]}\\
\hline
23 & 22 & 21 & 20 & 19 & 18 & 17 & 16\\
\hline
\multicolumn{8}{|c|}{\cellcolor{RoyalPurple!25}WDIAG\_RXFCNT[23:16]}\\
\hline
15 & 14 & 13 & 12 & 11 & 10 & 9 & 8\\
\hline
\multicolumn{8}{|c|}{\cellcolor{RoyalPurple!25}WDIAG\_RXFCNT[15:8]}\\
\hline
7 & 6 & 5 & 4 & 3 & 2 & 1 & 0\\
\hline
\multicolumn{8}{|c|}{\cellcolor{RoyalPurple!25}WDIAG\_RXFCNT[7:0]}\\
\hline
\end{tabular}
}

\begin{itemize}
\item \begin{small}
{\bf 
WDIAG\_RXFCNT
} [\emph{read-only}]: Data
\end{small}
\end{itemize}
\paragraph*{WRPC Diag:local time [msb of s]}\mbox{}\\\vskip 6pt
\rowcolors{1}{white}{white}
\begin{tabular}{l l }
{\bf HW prefix:}  & wrc\_diags\_wdiag\_sec\_msb\\
{\bf HW address:}  & 0x8\\
{\bf SW prefix:}  & WDIAG\_SEC\_MSB\\
{\bf SW offset:}  & 0x20\\
\end{tabular}

\vspace{12pt}
Local Time expressed in seconds since epoch (TAI)

\vspace{12pt}
\noindent
\resizebox{\textwidth}{!}{
\begin{tabular}{>{\centering\arraybackslash}p{1.5cm} >{\centering\arraybackslash}p{1.5cm} >{\centering\arraybackslash}p{1.5cm} >{\centering\arraybackslash}p{1.5cm} >{\centering\arraybackslash}p{1.5cm} >{\centering\arraybackslash}p{1.5cm} >{\centering\arraybackslash}p{1.5cm} >{\centering\arraybackslash}p{1.5cm} }
31 & 30 & 29 & 28 & 27 & 26 & 25 & 24\\
\hline
\multicolumn{8}{|c|}{\cellcolor{RoyalPurple!25}WDIAG\_SEC\_MSB[31:24]}\\
\hline
23 & 22 & 21 & 20 & 19 & 18 & 17 & 16\\
\hline
\multicolumn{8}{|c|}{\cellcolor{RoyalPurple!25}WDIAG\_SEC\_MSB[23:16]}\\
\hline
15 & 14 & 13 & 12 & 11 & 10 & 9 & 8\\
\hline
\multicolumn{8}{|c|}{\cellcolor{RoyalPurple!25}WDIAG\_SEC\_MSB[15:8]}\\
\hline
7 & 6 & 5 & 4 & 3 & 2 & 1 & 0\\
\hline
\multicolumn{8}{|c|}{\cellcolor{RoyalPurple!25}WDIAG\_SEC\_MSB[7:0]}\\
\hline
\end{tabular}
}

\begin{itemize}
\item \begin{small}
{\bf 
WDIAG\_SEC\_MSB
} [\emph{read-only}]: Data
\end{small}
\end{itemize}
\paragraph*{WRPC Diag: local time [lsb of s]}\mbox{}\\\vskip 6pt
\rowcolors{1}{white}{white}
\begin{tabular}{l l }
{\bf HW prefix:}  & wrc\_diags\_wdiag\_sec\_lsb\\
{\bf HW address:}  & 0x9\\
{\bf SW prefix:}  & WDIAG\_SEC\_LSB\\
{\bf SW offset:}  & 0x24\\
\end{tabular}

\vspace{12pt}
Local Time expressed in seconds since epoch (TAI)

\vspace{12pt}
\noindent
\resizebox{\textwidth}{!}{
\begin{tabular}{>{\centering\arraybackslash}p{1.5cm} >{\centering\arraybackslash}p{1.5cm} >{\centering\arraybackslash}p{1.5cm} >{\centering\arraybackslash}p{1.5cm} >{\centering\arraybackslash}p{1.5cm} >{\centering\arraybackslash}p{1.5cm} >{\centering\arraybackslash}p{1.5cm} >{\centering\arraybackslash}p{1.5cm} }
31 & 30 & 29 & 28 & 27 & 26 & 25 & 24\\
\hline
\multicolumn{8}{|c|}{\cellcolor{RoyalPurple!25}WDIAG\_SEC\_LSB[31:24]}\\
\hline
23 & 22 & 21 & 20 & 19 & 18 & 17 & 16\\
\hline
\multicolumn{8}{|c|}{\cellcolor{RoyalPurple!25}WDIAG\_SEC\_LSB[23:16]}\\
\hline
15 & 14 & 13 & 12 & 11 & 10 & 9 & 8\\
\hline
\multicolumn{8}{|c|}{\cellcolor{RoyalPurple!25}WDIAG\_SEC\_LSB[15:8]}\\
\hline
7 & 6 & 5 & 4 & 3 & 2 & 1 & 0\\
\hline
\multicolumn{8}{|c|}{\cellcolor{RoyalPurple!25}WDIAG\_SEC\_LSB[7:0]}\\
\hline
\end{tabular}
}

\begin{itemize}
\item \begin{small}
{\bf 
WDIAG\_SEC\_LSB
} [\emph{read-only}]: Data
\end{small}
\end{itemize}
\paragraph*{WRPC Diag: local time [ns]}\mbox{}\\\vskip 6pt
\rowcolors{1}{white}{white}
\begin{tabular}{l l }
{\bf HW prefix:}  & wrc\_diags\_wdiag\_ns\\
{\bf HW address:}  & 0xa\\
{\bf SW prefix:}  & WDIAG\_NS\\
{\bf SW offset:}  & 0x28\\
\end{tabular}

\vspace{12pt}
Nanoseconds part of the Local Time expressed in seconds since epoch (TAI)

\vspace{12pt}
\noindent
\resizebox{\textwidth}{!}{
\begin{tabular}{>{\centering\arraybackslash}p{1.5cm} >{\centering\arraybackslash}p{1.5cm} >{\centering\arraybackslash}p{1.5cm} >{\centering\arraybackslash}p{1.5cm} >{\centering\arraybackslash}p{1.5cm} >{\centering\arraybackslash}p{1.5cm} >{\centering\arraybackslash}p{1.5cm} >{\centering\arraybackslash}p{1.5cm} }
31 & 30 & 29 & 28 & 27 & 26 & 25 & 24\\
\hline
\multicolumn{8}{|c|}{\cellcolor{RoyalPurple!25}WDIAG\_NS[31:24]}\\
\hline
23 & 22 & 21 & 20 & 19 & 18 & 17 & 16\\
\hline
\multicolumn{8}{|c|}{\cellcolor{RoyalPurple!25}WDIAG\_NS[23:16]}\\
\hline
15 & 14 & 13 & 12 & 11 & 10 & 9 & 8\\
\hline
\multicolumn{8}{|c|}{\cellcolor{RoyalPurple!25}WDIAG\_NS[15:8]}\\
\hline
7 & 6 & 5 & 4 & 3 & 2 & 1 & 0\\
\hline
\multicolumn{8}{|c|}{\cellcolor{RoyalPurple!25}WDIAG\_NS[7:0]}\\
\hline
\end{tabular}
}

\begin{itemize}
\item \begin{small}
{\bf 
WDIAG\_NS
} [\emph{read-only}]: Data
\end{small}
\end{itemize}
\paragraph*{WRPC Diag: Round trip (mu) [msb of ps]}\mbox{}\\\vskip 6pt
\rowcolors{1}{white}{white}
\begin{tabular}{l l }
{\bf HW prefix:}  & wrc\_diags\_wdiag\_mu\_msb\\
{\bf HW address:}  & 0xb\\
{\bf SW prefix:}  & WDIAG\_MU\_MSB\\
{\bf SW offset:}  & 0x2c\\
\end{tabular}


\vspace{12pt}
\noindent
\resizebox{\textwidth}{!}{
\begin{tabular}{>{\centering\arraybackslash}p{1.5cm} >{\centering\arraybackslash}p{1.5cm} >{\centering\arraybackslash}p{1.5cm} >{\centering\arraybackslash}p{1.5cm} >{\centering\arraybackslash}p{1.5cm} >{\centering\arraybackslash}p{1.5cm} >{\centering\arraybackslash}p{1.5cm} >{\centering\arraybackslash}p{1.5cm} }
31 & 30 & 29 & 28 & 27 & 26 & 25 & 24\\
\hline
\multicolumn{8}{|c|}{\cellcolor{RoyalPurple!25}WDIAG\_MU\_MSB[31:24]}\\
\hline
23 & 22 & 21 & 20 & 19 & 18 & 17 & 16\\
\hline
\multicolumn{8}{|c|}{\cellcolor{RoyalPurple!25}WDIAG\_MU\_MSB[23:16]}\\
\hline
15 & 14 & 13 & 12 & 11 & 10 & 9 & 8\\
\hline
\multicolumn{8}{|c|}{\cellcolor{RoyalPurple!25}WDIAG\_MU\_MSB[15:8]}\\
\hline
7 & 6 & 5 & 4 & 3 & 2 & 1 & 0\\
\hline
\multicolumn{8}{|c|}{\cellcolor{RoyalPurple!25}WDIAG\_MU\_MSB[7:0]}\\
\hline
\end{tabular}
}

\begin{itemize}
\item \begin{small}
{\bf 
WDIAG\_MU\_MSB
} [\emph{read-only}]: Data
\end{small}
\end{itemize}
\paragraph*{WRPC Diag: Round trip (mu) [lsb of ps]}\mbox{}\\\vskip 6pt
\rowcolors{1}{white}{white}
\begin{tabular}{l l }
{\bf HW prefix:}  & wrc\_diags\_wdiag\_mu\_lsb\\
{\bf HW address:}  & 0xc\\
{\bf SW prefix:}  & WDIAG\_MU\_LSB\\
{\bf SW offset:}  & 0x30\\
\end{tabular}


\vspace{12pt}
\noindent
\resizebox{\textwidth}{!}{
\begin{tabular}{>{\centering\arraybackslash}p{1.5cm} >{\centering\arraybackslash}p{1.5cm} >{\centering\arraybackslash}p{1.5cm} >{\centering\arraybackslash}p{1.5cm} >{\centering\arraybackslash}p{1.5cm} >{\centering\arraybackslash}p{1.5cm} >{\centering\arraybackslash}p{1.5cm} >{\centering\arraybackslash}p{1.5cm} }
31 & 30 & 29 & 28 & 27 & 26 & 25 & 24\\
\hline
\multicolumn{8}{|c|}{\cellcolor{RoyalPurple!25}WDIAG\_MU\_LSB[31:24]}\\
\hline
23 & 22 & 21 & 20 & 19 & 18 & 17 & 16\\
\hline
\multicolumn{8}{|c|}{\cellcolor{RoyalPurple!25}WDIAG\_MU\_LSB[23:16]}\\
\hline
15 & 14 & 13 & 12 & 11 & 10 & 9 & 8\\
\hline
\multicolumn{8}{|c|}{\cellcolor{RoyalPurple!25}WDIAG\_MU\_LSB[15:8]}\\
\hline
7 & 6 & 5 & 4 & 3 & 2 & 1 & 0\\
\hline
\multicolumn{8}{|c|}{\cellcolor{RoyalPurple!25}WDIAG\_MU\_LSB[7:0]}\\
\hline
\end{tabular}
}

\begin{itemize}
\item \begin{small}
{\bf 
WDIAG\_MU\_LSB
} [\emph{read-only}]: Data
\end{small}
\end{itemize}
\paragraph*{WRPC Diag: Master-slave delay (dms) [msb of ps]}\mbox{}\\\vskip 6pt
\rowcolors{1}{white}{white}
\begin{tabular}{l l }
{\bf HW prefix:}  & wrc\_diags\_wdiag\_dms\_msb\\
{\bf HW address:}  & 0xd\\
{\bf SW prefix:}  & WDIAG\_DMS\_MSB\\
{\bf SW offset:}  & 0x34\\
\end{tabular}


\vspace{12pt}
\noindent
\resizebox{\textwidth}{!}{
\begin{tabular}{>{\centering\arraybackslash}p{1.5cm} >{\centering\arraybackslash}p{1.5cm} >{\centering\arraybackslash}p{1.5cm} >{\centering\arraybackslash}p{1.5cm} >{\centering\arraybackslash}p{1.5cm} >{\centering\arraybackslash}p{1.5cm} >{\centering\arraybackslash}p{1.5cm} >{\centering\arraybackslash}p{1.5cm} }
31 & 30 & 29 & 28 & 27 & 26 & 25 & 24\\
\hline
\multicolumn{8}{|c|}{\cellcolor{RoyalPurple!25}WDIAG\_DMS\_MSB[31:24]}\\
\hline
23 & 22 & 21 & 20 & 19 & 18 & 17 & 16\\
\hline
\multicolumn{8}{|c|}{\cellcolor{RoyalPurple!25}WDIAG\_DMS\_MSB[23:16]}\\
\hline
15 & 14 & 13 & 12 & 11 & 10 & 9 & 8\\
\hline
\multicolumn{8}{|c|}{\cellcolor{RoyalPurple!25}WDIAG\_DMS\_MSB[15:8]}\\
\hline
7 & 6 & 5 & 4 & 3 & 2 & 1 & 0\\
\hline
\multicolumn{8}{|c|}{\cellcolor{RoyalPurple!25}WDIAG\_DMS\_MSB[7:0]}\\
\hline
\end{tabular}
}

\begin{itemize}
\item \begin{small}
{\bf 
WDIAG\_DMS\_MSB
} [\emph{read-only}]: Data
\end{small}
\end{itemize}
\paragraph*{WRPC Diag: Master-slave delay (dms) [lsb of ps]}\mbox{}\\\vskip 6pt
\rowcolors{1}{white}{white}
\begin{tabular}{l l }
{\bf HW prefix:}  & wrc\_diags\_wdiag\_dms\_lsb\\
{\bf HW address:}  & 0xe\\
{\bf SW prefix:}  & WDIAG\_DMS\_LSB\\
{\bf SW offset:}  & 0x38\\
\end{tabular}


\vspace{12pt}
\noindent
\resizebox{\textwidth}{!}{
\begin{tabular}{>{\centering\arraybackslash}p{1.5cm} >{\centering\arraybackslash}p{1.5cm} >{\centering\arraybackslash}p{1.5cm} >{\centering\arraybackslash}p{1.5cm} >{\centering\arraybackslash}p{1.5cm} >{\centering\arraybackslash}p{1.5cm} >{\centering\arraybackslash}p{1.5cm} >{\centering\arraybackslash}p{1.5cm} }
31 & 30 & 29 & 28 & 27 & 26 & 25 & 24\\
\hline
\multicolumn{8}{|c|}{\cellcolor{RoyalPurple!25}WDIAG\_DMS\_LSB[31:24]}\\
\hline
23 & 22 & 21 & 20 & 19 & 18 & 17 & 16\\
\hline
\multicolumn{8}{|c|}{\cellcolor{RoyalPurple!25}WDIAG\_DMS\_LSB[23:16]}\\
\hline
15 & 14 & 13 & 12 & 11 & 10 & 9 & 8\\
\hline
\multicolumn{8}{|c|}{\cellcolor{RoyalPurple!25}WDIAG\_DMS\_LSB[15:8]}\\
\hline
7 & 6 & 5 & 4 & 3 & 2 & 1 & 0\\
\hline
\multicolumn{8}{|c|}{\cellcolor{RoyalPurple!25}WDIAG\_DMS\_LSB[7:0]}\\
\hline
\end{tabular}
}

\begin{itemize}
\item \begin{small}
{\bf 
WDIAG\_DMS\_LSB
} [\emph{read-only}]: Data
\end{small}
\end{itemize}
\paragraph*{WRPC Diag: Total link asymmetry [ps]}\mbox{}\\\vskip 6pt
\rowcolors{1}{white}{white}
\begin{tabular}{l l }
{\bf HW prefix:}  & wrc\_diags\_wdiag\_asym\\
{\bf HW address:}  & 0xf\\
{\bf SW prefix:}  & WDIAG\_ASYM\\
{\bf SW offset:}  & 0x3c\\
\end{tabular}


\vspace{12pt}
\noindent
\resizebox{\textwidth}{!}{
\begin{tabular}{>{\centering\arraybackslash}p{1.5cm} >{\centering\arraybackslash}p{1.5cm} >{\centering\arraybackslash}p{1.5cm} >{\centering\arraybackslash}p{1.5cm} >{\centering\arraybackslash}p{1.5cm} >{\centering\arraybackslash}p{1.5cm} >{\centering\arraybackslash}p{1.5cm} >{\centering\arraybackslash}p{1.5cm} }
31 & 30 & 29 & 28 & 27 & 26 & 25 & 24\\
\hline
\multicolumn{8}{|c|}{\cellcolor{RoyalPurple!25}WDIAG\_ASYM[31:24]}\\
\hline
23 & 22 & 21 & 20 & 19 & 18 & 17 & 16\\
\hline
\multicolumn{8}{|c|}{\cellcolor{RoyalPurple!25}WDIAG\_ASYM[23:16]}\\
\hline
15 & 14 & 13 & 12 & 11 & 10 & 9 & 8\\
\hline
\multicolumn{8}{|c|}{\cellcolor{RoyalPurple!25}WDIAG\_ASYM[15:8]}\\
\hline
7 & 6 & 5 & 4 & 3 & 2 & 1 & 0\\
\hline
\multicolumn{8}{|c|}{\cellcolor{RoyalPurple!25}WDIAG\_ASYM[7:0]}\\
\hline
\end{tabular}
}

\begin{itemize}
\item \begin{small}
{\bf 
WDIAG\_ASYM
} [\emph{read-only}]: Data
\end{small}
\end{itemize}
\paragraph*{WRPC Diag: Clock offset (cko) [ps]}\mbox{}\\\vskip 6pt
\rowcolors{1}{white}{white}
\begin{tabular}{l l }
{\bf HW prefix:}  & wrc\_diags\_wdiag\_cko\\
{\bf HW address:}  & 0x10\\
{\bf SW prefix:}  & WDIAG\_CKO\\
{\bf SW offset:}  & 0x40\\
\end{tabular}


\vspace{12pt}
\noindent
\resizebox{\textwidth}{!}{
\begin{tabular}{>{\centering\arraybackslash}p{1.5cm} >{\centering\arraybackslash}p{1.5cm} >{\centering\arraybackslash}p{1.5cm} >{\centering\arraybackslash}p{1.5cm} >{\centering\arraybackslash}p{1.5cm} >{\centering\arraybackslash}p{1.5cm} >{\centering\arraybackslash}p{1.5cm} >{\centering\arraybackslash}p{1.5cm} }
31 & 30 & 29 & 28 & 27 & 26 & 25 & 24\\
\hline
\multicolumn{8}{|c|}{\cellcolor{RoyalPurple!25}WDIAG\_CKO[31:24]}\\
\hline
23 & 22 & 21 & 20 & 19 & 18 & 17 & 16\\
\hline
\multicolumn{8}{|c|}{\cellcolor{RoyalPurple!25}WDIAG\_CKO[23:16]}\\
\hline
15 & 14 & 13 & 12 & 11 & 10 & 9 & 8\\
\hline
\multicolumn{8}{|c|}{\cellcolor{RoyalPurple!25}WDIAG\_CKO[15:8]}\\
\hline
7 & 6 & 5 & 4 & 3 & 2 & 1 & 0\\
\hline
\multicolumn{8}{|c|}{\cellcolor{RoyalPurple!25}WDIAG\_CKO[7:0]}\\
\hline
\end{tabular}
}

\begin{itemize}
\item \begin{small}
{\bf 
WDIAG\_CKO
} [\emph{read-only}]: Data
\end{small}
\end{itemize}
\paragraph*{WRPC Diag: Phase setpoint (setp) [ps]}\mbox{}\\\vskip 6pt
\rowcolors{1}{white}{white}
\begin{tabular}{l l }
{\bf HW prefix:}  & wrc\_diags\_wdiag\_setp\\
{\bf HW address:}  & 0x11\\
{\bf SW prefix:}  & WDIAG\_SETP\\
{\bf SW offset:}  & 0x44\\
\end{tabular}


\vspace{12pt}
\noindent
\resizebox{\textwidth}{!}{
\begin{tabular}{>{\centering\arraybackslash}p{1.5cm} >{\centering\arraybackslash}p{1.5cm} >{\centering\arraybackslash}p{1.5cm} >{\centering\arraybackslash}p{1.5cm} >{\centering\arraybackslash}p{1.5cm} >{\centering\arraybackslash}p{1.5cm} >{\centering\arraybackslash}p{1.5cm} >{\centering\arraybackslash}p{1.5cm} }
31 & 30 & 29 & 28 & 27 & 26 & 25 & 24\\
\hline
\multicolumn{8}{|c|}{\cellcolor{RoyalPurple!25}WDIAG\_SETP[31:24]}\\
\hline
23 & 22 & 21 & 20 & 19 & 18 & 17 & 16\\
\hline
\multicolumn{8}{|c|}{\cellcolor{RoyalPurple!25}WDIAG\_SETP[23:16]}\\
\hline
15 & 14 & 13 & 12 & 11 & 10 & 9 & 8\\
\hline
\multicolumn{8}{|c|}{\cellcolor{RoyalPurple!25}WDIAG\_SETP[15:8]}\\
\hline
7 & 6 & 5 & 4 & 3 & 2 & 1 & 0\\
\hline
\multicolumn{8}{|c|}{\cellcolor{RoyalPurple!25}WDIAG\_SETP[7:0]}\\
\hline
\end{tabular}
}

\begin{itemize}
\item \begin{small}
{\bf 
WDIAG\_SETP
} [\emph{read-only}]: Data
\end{small}
\end{itemize}
\paragraph*{WRPC Diag: Update counter (ucnt)}\mbox{}\\\vskip 6pt
\rowcolors{1}{white}{white}
\begin{tabular}{l l }
{\bf HW prefix:}  & wrc\_diags\_wdiag\_ucnt\\
{\bf HW address:}  & 0x12\\
{\bf SW prefix:}  & WDIAG\_UCNT\\
{\bf SW offset:}  & 0x48\\
\end{tabular}


\vspace{12pt}
\noindent
\resizebox{\textwidth}{!}{
\begin{tabular}{>{\centering\arraybackslash}p{1.5cm} >{\centering\arraybackslash}p{1.5cm} >{\centering\arraybackslash}p{1.5cm} >{\centering\arraybackslash}p{1.5cm} >{\centering\arraybackslash}p{1.5cm} >{\centering\arraybackslash}p{1.5cm} >{\centering\arraybackslash}p{1.5cm} >{\centering\arraybackslash}p{1.5cm} }
31 & 30 & 29 & 28 & 27 & 26 & 25 & 24\\
\hline
\multicolumn{8}{|c|}{\cellcolor{RoyalPurple!25}WDIAG\_UCNT[31:24]}\\
\hline
23 & 22 & 21 & 20 & 19 & 18 & 17 & 16\\
\hline
\multicolumn{8}{|c|}{\cellcolor{RoyalPurple!25}WDIAG\_UCNT[23:16]}\\
\hline
15 & 14 & 13 & 12 & 11 & 10 & 9 & 8\\
\hline
\multicolumn{8}{|c|}{\cellcolor{RoyalPurple!25}WDIAG\_UCNT[15:8]}\\
\hline
7 & 6 & 5 & 4 & 3 & 2 & 1 & 0\\
\hline
\multicolumn{8}{|c|}{\cellcolor{RoyalPurple!25}WDIAG\_UCNT[7:0]}\\
\hline
\end{tabular}
}

\begin{itemize}
\item \begin{small}
{\bf 
WDIAG\_UCNT
} [\emph{read-only}]: Data
\end{small}
\end{itemize}
\paragraph*{WRPC Diag: Board temperature [C degree]}\mbox{}\\\vskip 6pt
\rowcolors{1}{white}{white}
\begin{tabular}{l l }
{\bf HW prefix:}  & wrc\_diags\_wdiag\_temp\\
{\bf HW address:}  & 0x13\\
{\bf SW prefix:}  & WDIAG\_TEMP\\
{\bf SW offset:}  & 0x4c\\
\end{tabular}


\vspace{12pt}
\noindent
\resizebox{\textwidth}{!}{
\begin{tabular}{>{\centering\arraybackslash}p{1.5cm} >{\centering\arraybackslash}p{1.5cm} >{\centering\arraybackslash}p{1.5cm} >{\centering\arraybackslash}p{1.5cm} >{\centering\arraybackslash}p{1.5cm} >{\centering\arraybackslash}p{1.5cm} >{\centering\arraybackslash}p{1.5cm} >{\centering\arraybackslash}p{1.5cm} }
31 & 30 & 29 & 28 & 27 & 26 & 25 & 24\\
\hline
\multicolumn{8}{|c|}{\cellcolor{RoyalPurple!25}WDIAG\_TEMP[31:24]}\\
\hline
23 & 22 & 21 & 20 & 19 & 18 & 17 & 16\\
\hline
\multicolumn{8}{|c|}{\cellcolor{RoyalPurple!25}WDIAG\_TEMP[23:16]}\\
\hline
15 & 14 & 13 & 12 & 11 & 10 & 9 & 8\\
\hline
\multicolumn{8}{|c|}{\cellcolor{RoyalPurple!25}WDIAG\_TEMP[15:8]}\\
\hline
7 & 6 & 5 & 4 & 3 & 2 & 1 & 0\\
\hline
\multicolumn{8}{|c|}{\cellcolor{RoyalPurple!25}WDIAG\_TEMP[7:0]}\\
\hline
\end{tabular}
}

\begin{itemize}
\item \begin{small}
{\bf 
WDIAG\_TEMP
} [\emph{read-only}]: Data
\end{small}
\end{itemize}




\newpage
\subsection{WR Streamers, status and debug}
\label{subsec:wbgen:wr_streamers}
[version 0x00000001]\\
-----------------------------------------------------------------\\  This WB registers allow to diagnose transmission and reception of\\  data using WR streamers.                                         \\  In particular, these registers provide access to streamer's      \\  statistics that can be also access from SNMP, if supported.      \\  -----------------------------------------------------------------\\  Copyright (c) 2016 CERN/BE-CO-HT and CERN/TE-MS-MM               \\                                                                   \\  This source file is free software; you can redistribute it       \\  and/or modify it under the terms of the GNU Lesser General       \\  Public License as published by the Free Software Foundation;     \\  either version 2.1 of the License, or (at your option) any       \\  later version.                                                   \\                                                                   \\  This source is distributed in the hope that it will be           \\  useful, but WITHOUT ANY WARRANTY; without even the implied       \\  warranty of MERCHANTABILITY or FITNESS FOR A PARTICULAR          \\  PURPOSE.  See the GNU Lesser General Public License for more     \\  details                                                          \\                                                                   \\  You should have received a copy of the GNU Lesser General        \\  Public License along with this source; if not, download it       \\  from http://www.gnu.org/licenses/lgpl-2.1.html                   \\  -----------------------------------------------------------------
\subsubsection{Memory map summary}
\rowcolors{2}{gray!25}{white}
\resizebox{\textwidth}{!}{
\begin{tabular}{|l|l|l|l|l|}
\rowcolor{RoyalPurple}
\color{white} SW Offset & \color{white} Type & \color{white} Name &
\color{white} HW prefix & \color{white} C prefix\\
0x0& REG & Version register & wr\_streamers\_ver & VER\\
0x4& REG & Statistics status and ctrl register & wr\_streamers\_sscr1 & SSCR1\\
0x8& REG & Statistics status and ctrl register & wr\_streamers\_sscr2 & SSCR2\\
0xc& REG & Statistics status and ctrl register & wr\_streamers\_sscr3 & SSCR3\\
0x10& REG & Rx statistics & wr\_streamers\_rx\_stat0 & RX\_STAT0\\
0x14& REG & Rx statistics & wr\_streamers\_rx\_stat1 & RX\_STAT1\\
0x18& REG & Tx statistics & wr\_streamers\_tx\_stat2 & TX\_STAT2\\
0x1c& REG & Tx statistics & wr\_streamers\_tx\_stat3 & TX\_STAT3\\
0x20& REG & Rx statistics & wr\_streamers\_rx\_stat4 & RX\_STAT4\\
0x24& REG & Rx statistics & wr\_streamers\_rx\_stat5 & RX\_STAT5\\
0x28& REG & Rx statistics & wr\_streamers\_rx\_stat6 & RX\_STAT6\\
0x2c& REG & Rx statistics & wr\_streamers\_rx\_stat7 & RX\_STAT7\\
0x30& REG & Rx statistics & wr\_streamers\_rx\_stat8 & RX\_STAT8\\
0x34& REG & Rx statistics & wr\_streamers\_rx\_stat9 & RX\_STAT9\\
0x38& REG & Rx statistics & wr\_streamers\_rx\_stat10 & RX\_STAT10\\
0x3c& REG & Rx statistics & wr\_streamers\_rx\_stat11 & RX\_STAT11\\
0x40& REG & Rx statistics & wr\_streamers\_rx\_stat12 & RX\_STAT12\\
0x44& REG & Rx statistics & wr\_streamers\_rx\_stat13 & RX\_STAT13\\
0x48& REG & Tx Config Reg 0 & wr\_streamers\_tx\_cfg0 & TX\_CFG0\\
0x4c& REG & Tx Config Reg 1 & wr\_streamers\_tx\_cfg1 & TX\_CFG1\\
0x50& REG & Tx Config Reg 2 & wr\_streamers\_tx\_cfg2 & TX\_CFG2\\
0x54& REG & Tx Config Reg 3 & wr\_streamers\_tx\_cfg3 & TX\_CFG3\\
0x58& REG & Tx Config Reg 4 & wr\_streamers\_tx\_cfg4 & TX\_CFG4\\
0x5c& REG & Tx Config Reg 4 & wr\_streamers\_tx\_cfg5 & TX\_CFG5\\
0x60& REG & Rx Config Reg 0 & wr\_streamers\_rx\_cfg0 & RX\_CFG0\\
0x64& REG & Rx Config Reg 1 & wr\_streamers\_rx\_cfg1 & RX\_CFG1\\
0x68& REG & Rx Config Reg 2 & wr\_streamers\_rx\_cfg2 & RX\_CFG2\\
0x6c& REG & Rx Config Reg 3 & wr\_streamers\_rx\_cfg3 & RX\_CFG3\\
0x70& REG & Rx Config Reg 4 & wr\_streamers\_rx\_cfg4 & RX\_CFG4\\
0x74& REG & Rx Config Reg 5 & wr\_streamers\_rx\_cfg5 & RX\_CFG5\\
0x78& REG & TxRx Config Override & wr\_streamers\_cfg & CFG\\
0x7c& REG & DBG Control register & wr\_streamers\_dbg\_ctrl & DBG\_CTRL\\
0x80& REG & DBG Data & wr\_streamers\_dbg\_data & DBG\_DATA\\
0x84& REG & Test value & wr\_streamers\_dummy & DUMMY\\
\hline
\end{tabular}
}

\subsubsection{Register description}
\paragraph*{Version register}\mbox{}\\\vskip 6pt
\rowcolors{1}{white}{white}
\begin{tabular}{l l }
{\bf HW prefix:}  & wr\_streamers\_ver\\
{\bf HW address:}  & 0x0\\
{\bf SW prefix:}  & VER\\
{\bf SW offset:}  & 0x0\\
\end{tabular}


\vspace{12pt}
\noindent
\resizebox{\textwidth}{!}{
\begin{tabular}{>{\centering\arraybackslash}p{1.5cm} >{\centering\arraybackslash}p{1.5cm} >{\centering\arraybackslash}p{1.5cm} >{\centering\arraybackslash}p{1.5cm} >{\centering\arraybackslash}p{1.5cm} >{\centering\arraybackslash}p{1.5cm} >{\centering\arraybackslash}p{1.5cm} >{\centering\arraybackslash}p{1.5cm} }
31 & 30 & 29 & 28 & 27 & 26 & 25 & 24\\
\hline
\multicolumn{8}{|c|}{\cellcolor{RoyalPurple!25}ID[31:24]}\\
\hline
23 & 22 & 21 & 20 & 19 & 18 & 17 & 16\\
\hline
\multicolumn{8}{|c|}{\cellcolor{RoyalPurple!25}ID[23:16]}\\
\hline
15 & 14 & 13 & 12 & 11 & 10 & 9 & 8\\
\hline
\multicolumn{8}{|c|}{\cellcolor{RoyalPurple!25}ID[15:8]}\\
\hline
7 & 6 & 5 & 4 & 3 & 2 & 1 & 0\\
\hline
\multicolumn{8}{|c|}{\cellcolor{RoyalPurple!25}ID[7:0]}\\
\hline
\end{tabular}
}

\begin{itemize}
\item \begin{small}
{\bf 
ID
} [\emph{read/write}]: Version identifier
\\
Version identifier for the peripheral
\end{small}
\end{itemize}
\paragraph*{Statistics status and ctrl register}\mbox{}\\\vskip 6pt
\rowcolors{1}{white}{white}
\begin{tabular}{l l }
{\bf HW prefix:}  & wr\_streamers\_sscr1\\
{\bf HW address:}  & 0x1\\
{\bf SW prefix:}  & SSCR1\\
{\bf SW offset:}  & 0x4\\
\end{tabular}


\vspace{12pt}
\noindent
\resizebox{\textwidth}{!}{
\begin{tabular}{>{\centering\arraybackslash}p{1.5cm} >{\centering\arraybackslash}p{1.5cm} >{\centering\arraybackslash}p{1.5cm} >{\centering\arraybackslash}p{1.5cm} >{\centering\arraybackslash}p{1.5cm} >{\centering\arraybackslash}p{1.5cm} >{\centering\arraybackslash}p{1.5cm} >{\centering\arraybackslash}p{1.5cm} }
31 & 30 & 29 & 28 & 27 & 26 & 25 & 24\\
\hline
\multicolumn{8}{|c|}{\cellcolor{RoyalPurple!25}RST\_TS\_CYC[27:20]}\\
\hline
23 & 22 & 21 & 20 & 19 & 18 & 17 & 16\\
\hline
\multicolumn{8}{|c|}{\cellcolor{RoyalPurple!25}RST\_TS\_CYC[19:12]}\\
\hline
15 & 14 & 13 & 12 & 11 & 10 & 9 & 8\\
\hline
\multicolumn{8}{|c|}{\cellcolor{RoyalPurple!25}RST\_TS\_CYC[11:4]}\\
\hline
7 & 6 & 5 & 4 & 3 & 2 & 1 & 0\\
\hline
\multicolumn{4}{|c|}{\cellcolor{RoyalPurple!25}RST\_TS\_CYC[3:0]} & \multicolumn{1}{|c|}{\cellcolor{RoyalPurple!25}RX\_LATENCY\_ACC\_OVERFLOW} & \multicolumn{1}{|c|}{\cellcolor{RoyalPurple!25}SNAPSHOT\_STATS} & \multicolumn{1}{|c|}{\cellcolor{RoyalPurple!25}RST\_SEQ\_ID} & \multicolumn{1}{|c|}{\cellcolor{RoyalPurple!25}RST\_STATS}\\
\hline
\end{tabular}
}

\begin{itemize}
\item \begin{small}
{\bf 
RST\_STATS
} [\emph{write-only}]: Reset statistics
\\
Writing 1 reset counters, latency acc/max/min. This reset is timestamped
\end{small}
\item \begin{small}
{\bf 
RST\_SEQ\_ID
} [\emph{write-only}]: Reset tx seq id
\\
Writing 1 reset sequence ID of transmitted frames
\end{small}
\item \begin{small}
{\bf 
SNAPSHOT\_STATS
} [\emph{read/write}]: Snapshot statistics
\\
Writing 1 snapshots statistics for reading, it means that all the counters \\                    are copied at the same instant to registers and this registers can be read\\                    via wishbone/snmp while the counters are still running in the background. \\                    this allows to read coherent data
\end{small}
\item \begin{small}
{\bf 
RX\_LATENCY\_ACC\_OVERFLOW
} [\emph{read-only}]: Latency accumulator overflow
\\
Latency accumulator overflow - the lateny accumulator value is invalid
\end{small}
\item \begin{small}
{\bf 
RST\_TS\_CYC
} [\emph{read-only}]: Reset timestamp cycles
\\
Timestamp of the last reset of stats (RST\_STAT) -- count of clock cycles
\end{small}
\end{itemize}
\paragraph*{Statistics status and ctrl register}\mbox{}\\\vskip 6pt
\rowcolors{1}{white}{white}
\begin{tabular}{l l }
{\bf HW prefix:}  & wr\_streamers\_sscr2\\
{\bf HW address:}  & 0x2\\
{\bf SW prefix:}  & SSCR2\\
{\bf SW offset:}  & 0x8\\
\end{tabular}


\vspace{12pt}
\noindent
\resizebox{\textwidth}{!}{
\begin{tabular}{>{\centering\arraybackslash}p{1.5cm} >{\centering\arraybackslash}p{1.5cm} >{\centering\arraybackslash}p{1.5cm} >{\centering\arraybackslash}p{1.5cm} >{\centering\arraybackslash}p{1.5cm} >{\centering\arraybackslash}p{1.5cm} >{\centering\arraybackslash}p{1.5cm} >{\centering\arraybackslash}p{1.5cm} }
31 & 30 & 29 & 28 & 27 & 26 & 25 & 24\\
\hline
\multicolumn{8}{|c|}{\cellcolor{RoyalPurple!25}RST\_TS\_TAI\_LSB[31:24]}\\
\hline
23 & 22 & 21 & 20 & 19 & 18 & 17 & 16\\
\hline
\multicolumn{8}{|c|}{\cellcolor{RoyalPurple!25}RST\_TS\_TAI\_LSB[23:16]}\\
\hline
15 & 14 & 13 & 12 & 11 & 10 & 9 & 8\\
\hline
\multicolumn{8}{|c|}{\cellcolor{RoyalPurple!25}RST\_TS\_TAI\_LSB[15:8]}\\
\hline
7 & 6 & 5 & 4 & 3 & 2 & 1 & 0\\
\hline
\multicolumn{8}{|c|}{\cellcolor{RoyalPurple!25}RST\_TS\_TAI\_LSB[7:0]}\\
\hline
\end{tabular}
}

\begin{itemize}
\item \begin{small}
{\bf 
RST\_TS\_TAI\_LSB
} [\emph{read-only}]: Reset timestamp 32 LSB of TAI
\\
Timestamp of the last reset of stats (RST\_STAT)  -- LSB 32 bits of TAI
\end{small}
\end{itemize}
\paragraph*{Statistics status and ctrl register}\mbox{}\\\vskip 6pt
\rowcolors{1}{white}{white}
\begin{tabular}{l l }
{\bf HW prefix:}  & wr\_streamers\_sscr3\\
{\bf HW address:}  & 0x3\\
{\bf SW prefix:}  & SSCR3\\
{\bf SW offset:}  & 0xc\\
\end{tabular}


\vspace{12pt}
\noindent
\resizebox{\textwidth}{!}{
\begin{tabular}{>{\centering\arraybackslash}p{1.5cm} >{\centering\arraybackslash}p{1.5cm} >{\centering\arraybackslash}p{1.5cm} >{\centering\arraybackslash}p{1.5cm} >{\centering\arraybackslash}p{1.5cm} >{\centering\arraybackslash}p{1.5cm} >{\centering\arraybackslash}p{1.5cm} >{\centering\arraybackslash}p{1.5cm} }
31 & 30 & 29 & 28 & 27 & 26 & 25 & 24\\
\hline
\multicolumn{1}{|c}{-} & - & - & - & - & - & - & \multicolumn{1}{c|}{-}\\
\hline
23 & 22 & 21 & 20 & 19 & 18 & 17 & 16\\
\hline
\multicolumn{1}{|c}{-} & - & - & - & - & - & - & \multicolumn{1}{c|}{-}\\
\hline
15 & 14 & 13 & 12 & 11 & 10 & 9 & 8\\
\hline
\multicolumn{1}{|c}{-} & - & - & - & - & - & - & \multicolumn{1}{c|}{-}\\
\hline
7 & 6 & 5 & 4 & 3 & 2 & 1 & 0\\
\hline
\multicolumn{8}{|c|}{\cellcolor{RoyalPurple!25}RST\_TS\_TAI\_MSB[7:0]}\\
\hline
\end{tabular}
}

\begin{itemize}
\item \begin{small}
{\bf 
RST\_TS\_TAI\_MSB
} [\emph{read-only}]: Reset timestamp 8 MSB of TAI
\\
Timestamp of the last reset of stats (RST\_STAT)  -- MSB 8 bits of TAI
\end{small}
\end{itemize}
\paragraph*{Rx statistics}\mbox{}\\\vskip 6pt
\rowcolors{1}{white}{white}
\begin{tabular}{l l }
{\bf HW prefix:}  & wr\_streamers\_rx\_stat0\\
{\bf HW address:}  & 0x4\\
{\bf SW prefix:}  & RX\_STAT0\\
{\bf SW offset:}  & 0x10\\
\end{tabular}


\vspace{12pt}
\noindent
\resizebox{\textwidth}{!}{
\begin{tabular}{>{\centering\arraybackslash}p{1.5cm} >{\centering\arraybackslash}p{1.5cm} >{\centering\arraybackslash}p{1.5cm} >{\centering\arraybackslash}p{1.5cm} >{\centering\arraybackslash}p{1.5cm} >{\centering\arraybackslash}p{1.5cm} >{\centering\arraybackslash}p{1.5cm} >{\centering\arraybackslash}p{1.5cm} }
31 & 30 & 29 & 28 & 27 & 26 & 25 & 24\\
\hline
\multicolumn{1}{|c}{-} & - & - & - & \multicolumn{4}{|c|}{\cellcolor{RoyalPurple!25}RX\_LATENCY\_MAX[27:24]}\\
\hline
23 & 22 & 21 & 20 & 19 & 18 & 17 & 16\\
\hline
\multicolumn{8}{|c|}{\cellcolor{RoyalPurple!25}RX\_LATENCY\_MAX[23:16]}\\
\hline
15 & 14 & 13 & 12 & 11 & 10 & 9 & 8\\
\hline
\multicolumn{8}{|c|}{\cellcolor{RoyalPurple!25}RX\_LATENCY\_MAX[15:8]}\\
\hline
7 & 6 & 5 & 4 & 3 & 2 & 1 & 0\\
\hline
\multicolumn{8}{|c|}{\cellcolor{RoyalPurple!25}RX\_LATENCY\_MAX[7:0]}\\
\hline
\end{tabular}
}

\begin{itemize}
\item \begin{small}
{\bf 
RX\_LATENCY\_MAX
} [\emph{read-only}]: WR Streamer frame latency
\\
Maximum latency of received frames since reset
\end{small}
\end{itemize}
\paragraph*{Rx statistics}\mbox{}\\\vskip 6pt
\rowcolors{1}{white}{white}
\begin{tabular}{l l }
{\bf HW prefix:}  & wr\_streamers\_rx\_stat1\\
{\bf HW address:}  & 0x5\\
{\bf SW prefix:}  & RX\_STAT1\\
{\bf SW offset:}  & 0x14\\
\end{tabular}


\vspace{12pt}
\noindent
\resizebox{\textwidth}{!}{
\begin{tabular}{>{\centering\arraybackslash}p{1.5cm} >{\centering\arraybackslash}p{1.5cm} >{\centering\arraybackslash}p{1.5cm} >{\centering\arraybackslash}p{1.5cm} >{\centering\arraybackslash}p{1.5cm} >{\centering\arraybackslash}p{1.5cm} >{\centering\arraybackslash}p{1.5cm} >{\centering\arraybackslash}p{1.5cm} }
31 & 30 & 29 & 28 & 27 & 26 & 25 & 24\\
\hline
\multicolumn{1}{|c}{-} & - & - & - & \multicolumn{4}{|c|}{\cellcolor{RoyalPurple!25}RX\_LATENCY\_MIN[27:24]}\\
\hline
23 & 22 & 21 & 20 & 19 & 18 & 17 & 16\\
\hline
\multicolumn{8}{|c|}{\cellcolor{RoyalPurple!25}RX\_LATENCY\_MIN[23:16]}\\
\hline
15 & 14 & 13 & 12 & 11 & 10 & 9 & 8\\
\hline
\multicolumn{8}{|c|}{\cellcolor{RoyalPurple!25}RX\_LATENCY\_MIN[15:8]}\\
\hline
7 & 6 & 5 & 4 & 3 & 2 & 1 & 0\\
\hline
\multicolumn{8}{|c|}{\cellcolor{RoyalPurple!25}RX\_LATENCY\_MIN[7:0]}\\
\hline
\end{tabular}
}

\begin{itemize}
\item \begin{small}
{\bf 
RX\_LATENCY\_MIN
} [\emph{read-only}]: WR Streamer frame latency
\\
Minimum latency of received frames since reset
\end{small}
\end{itemize}
\paragraph*{Tx statistics}\mbox{}\\\vskip 6pt
\rowcolors{1}{white}{white}
\begin{tabular}{l l }
{\bf HW prefix:}  & wr\_streamers\_tx\_stat2\\
{\bf HW address:}  & 0x6\\
{\bf SW prefix:}  & TX\_STAT2\\
{\bf SW offset:}  & 0x18\\
\end{tabular}


\vspace{12pt}
\noindent
\resizebox{\textwidth}{!}{
\begin{tabular}{>{\centering\arraybackslash}p{1.5cm} >{\centering\arraybackslash}p{1.5cm} >{\centering\arraybackslash}p{1.5cm} >{\centering\arraybackslash}p{1.5cm} >{\centering\arraybackslash}p{1.5cm} >{\centering\arraybackslash}p{1.5cm} >{\centering\arraybackslash}p{1.5cm} >{\centering\arraybackslash}p{1.5cm} }
31 & 30 & 29 & 28 & 27 & 26 & 25 & 24\\
\hline
\multicolumn{8}{|c|}{\cellcolor{RoyalPurple!25}TX\_SENT\_CNT\_LSB[31:24]}\\
\hline
23 & 22 & 21 & 20 & 19 & 18 & 17 & 16\\
\hline
\multicolumn{8}{|c|}{\cellcolor{RoyalPurple!25}TX\_SENT\_CNT\_LSB[23:16]}\\
\hline
15 & 14 & 13 & 12 & 11 & 10 & 9 & 8\\
\hline
\multicolumn{8}{|c|}{\cellcolor{RoyalPurple!25}TX\_SENT\_CNT\_LSB[15:8]}\\
\hline
7 & 6 & 5 & 4 & 3 & 2 & 1 & 0\\
\hline
\multicolumn{8}{|c|}{\cellcolor{RoyalPurple!25}TX\_SENT\_CNT\_LSB[7:0]}\\
\hline
\end{tabular}
}

\begin{itemize}
\item \begin{small}
{\bf 
TX\_SENT\_CNT\_LSB
} [\emph{read-only}]: WR Streamer frame sent count (LSB)
\\
Number of sent wr streamer frames since reset
\end{small}
\end{itemize}
\paragraph*{Tx statistics}\mbox{}\\\vskip 6pt
\rowcolors{1}{white}{white}
\begin{tabular}{l l }
{\bf HW prefix:}  & wr\_streamers\_tx\_stat3\\
{\bf HW address:}  & 0x7\\
{\bf SW prefix:}  & TX\_STAT3\\
{\bf SW offset:}  & 0x1c\\
\end{tabular}


\vspace{12pt}
\noindent
\resizebox{\textwidth}{!}{
\begin{tabular}{>{\centering\arraybackslash}p{1.5cm} >{\centering\arraybackslash}p{1.5cm} >{\centering\arraybackslash}p{1.5cm} >{\centering\arraybackslash}p{1.5cm} >{\centering\arraybackslash}p{1.5cm} >{\centering\arraybackslash}p{1.5cm} >{\centering\arraybackslash}p{1.5cm} >{\centering\arraybackslash}p{1.5cm} }
31 & 30 & 29 & 28 & 27 & 26 & 25 & 24\\
\hline
\multicolumn{8}{|c|}{\cellcolor{RoyalPurple!25}TX\_SENT\_CNT\_MSB[31:24]}\\
\hline
23 & 22 & 21 & 20 & 19 & 18 & 17 & 16\\
\hline
\multicolumn{8}{|c|}{\cellcolor{RoyalPurple!25}TX\_SENT\_CNT\_MSB[23:16]}\\
\hline
15 & 14 & 13 & 12 & 11 & 10 & 9 & 8\\
\hline
\multicolumn{8}{|c|}{\cellcolor{RoyalPurple!25}TX\_SENT\_CNT\_MSB[15:8]}\\
\hline
7 & 6 & 5 & 4 & 3 & 2 & 1 & 0\\
\hline
\multicolumn{8}{|c|}{\cellcolor{RoyalPurple!25}TX\_SENT\_CNT\_MSB[7:0]}\\
\hline
\end{tabular}
}

\begin{itemize}
\item \begin{small}
{\bf 
TX\_SENT\_CNT\_MSB
} [\emph{read-only}]: WR Streamer frame sent count (MSB)
\\
Number of sent wr streamer frames since reset
\end{small}
\end{itemize}
\paragraph*{Rx statistics}\mbox{}\\\vskip 6pt
\rowcolors{1}{white}{white}
\begin{tabular}{l l }
{\bf HW prefix:}  & wr\_streamers\_rx\_stat4\\
{\bf HW address:}  & 0x8\\
{\bf SW prefix:}  & RX\_STAT4\\
{\bf SW offset:}  & 0x20\\
\end{tabular}


\vspace{12pt}
\noindent
\resizebox{\textwidth}{!}{
\begin{tabular}{>{\centering\arraybackslash}p{1.5cm} >{\centering\arraybackslash}p{1.5cm} >{\centering\arraybackslash}p{1.5cm} >{\centering\arraybackslash}p{1.5cm} >{\centering\arraybackslash}p{1.5cm} >{\centering\arraybackslash}p{1.5cm} >{\centering\arraybackslash}p{1.5cm} >{\centering\arraybackslash}p{1.5cm} }
31 & 30 & 29 & 28 & 27 & 26 & 25 & 24\\
\hline
\multicolumn{8}{|c|}{\cellcolor{RoyalPurple!25}RX\_RCVD\_CNT\_LSB[31:24]}\\
\hline
23 & 22 & 21 & 20 & 19 & 18 & 17 & 16\\
\hline
\multicolumn{8}{|c|}{\cellcolor{RoyalPurple!25}RX\_RCVD\_CNT\_LSB[23:16]}\\
\hline
15 & 14 & 13 & 12 & 11 & 10 & 9 & 8\\
\hline
\multicolumn{8}{|c|}{\cellcolor{RoyalPurple!25}RX\_RCVD\_CNT\_LSB[15:8]}\\
\hline
7 & 6 & 5 & 4 & 3 & 2 & 1 & 0\\
\hline
\multicolumn{8}{|c|}{\cellcolor{RoyalPurple!25}RX\_RCVD\_CNT\_LSB[7:0]}\\
\hline
\end{tabular}
}

\begin{itemize}
\item \begin{small}
{\bf 
RX\_RCVD\_CNT\_LSB
} [\emph{read-only}]: WR Streamer frame received count (LSB)
\\
Number of received wr streamer frames since reset
\end{small}
\end{itemize}
\paragraph*{Rx statistics}\mbox{}\\\vskip 6pt
\rowcolors{1}{white}{white}
\begin{tabular}{l l }
{\bf HW prefix:}  & wr\_streamers\_rx\_stat5\\
{\bf HW address:}  & 0x9\\
{\bf SW prefix:}  & RX\_STAT5\\
{\bf SW offset:}  & 0x24\\
\end{tabular}


\vspace{12pt}
\noindent
\resizebox{\textwidth}{!}{
\begin{tabular}{>{\centering\arraybackslash}p{1.5cm} >{\centering\arraybackslash}p{1.5cm} >{\centering\arraybackslash}p{1.5cm} >{\centering\arraybackslash}p{1.5cm} >{\centering\arraybackslash}p{1.5cm} >{\centering\arraybackslash}p{1.5cm} >{\centering\arraybackslash}p{1.5cm} >{\centering\arraybackslash}p{1.5cm} }
31 & 30 & 29 & 28 & 27 & 26 & 25 & 24\\
\hline
\multicolumn{8}{|c|}{\cellcolor{RoyalPurple!25}RX\_RCVD\_CNT\_MSB[31:24]}\\
\hline
23 & 22 & 21 & 20 & 19 & 18 & 17 & 16\\
\hline
\multicolumn{8}{|c|}{\cellcolor{RoyalPurple!25}RX\_RCVD\_CNT\_MSB[23:16]}\\
\hline
15 & 14 & 13 & 12 & 11 & 10 & 9 & 8\\
\hline
\multicolumn{8}{|c|}{\cellcolor{RoyalPurple!25}RX\_RCVD\_CNT\_MSB[15:8]}\\
\hline
7 & 6 & 5 & 4 & 3 & 2 & 1 & 0\\
\hline
\multicolumn{8}{|c|}{\cellcolor{RoyalPurple!25}RX\_RCVD\_CNT\_MSB[7:0]}\\
\hline
\end{tabular}
}

\begin{itemize}
\item \begin{small}
{\bf 
RX\_RCVD\_CNT\_MSB
} [\emph{read-only}]: WR Streamer frame received count (MSB)
\\
Number of received wr streamer frames since reset
\end{small}
\end{itemize}
\paragraph*{Rx statistics}\mbox{}\\\vskip 6pt
\rowcolors{1}{white}{white}
\begin{tabular}{l l }
{\bf HW prefix:}  & wr\_streamers\_rx\_stat6\\
{\bf HW address:}  & 0xa\\
{\bf SW prefix:}  & RX\_STAT6\\
{\bf SW offset:}  & 0x28\\
\end{tabular}


\vspace{12pt}
\noindent
\resizebox{\textwidth}{!}{
\begin{tabular}{>{\centering\arraybackslash}p{1.5cm} >{\centering\arraybackslash}p{1.5cm} >{\centering\arraybackslash}p{1.5cm} >{\centering\arraybackslash}p{1.5cm} >{\centering\arraybackslash}p{1.5cm} >{\centering\arraybackslash}p{1.5cm} >{\centering\arraybackslash}p{1.5cm} >{\centering\arraybackslash}p{1.5cm} }
31 & 30 & 29 & 28 & 27 & 26 & 25 & 24\\
\hline
\multicolumn{8}{|c|}{\cellcolor{RoyalPurple!25}RX\_LOSS\_CNT\_LSB[31:24]}\\
\hline
23 & 22 & 21 & 20 & 19 & 18 & 17 & 16\\
\hline
\multicolumn{8}{|c|}{\cellcolor{RoyalPurple!25}RX\_LOSS\_CNT\_LSB[23:16]}\\
\hline
15 & 14 & 13 & 12 & 11 & 10 & 9 & 8\\
\hline
\multicolumn{8}{|c|}{\cellcolor{RoyalPurple!25}RX\_LOSS\_CNT\_LSB[15:8]}\\
\hline
7 & 6 & 5 & 4 & 3 & 2 & 1 & 0\\
\hline
\multicolumn{8}{|c|}{\cellcolor{RoyalPurple!25}RX\_LOSS\_CNT\_LSB[7:0]}\\
\hline
\end{tabular}
}

\begin{itemize}
\item \begin{small}
{\bf 
RX\_LOSS\_CNT\_LSB
} [\emph{read-only}]: WR Streamer frame loss count (LSB)
\\
Number of lost wr streamer frames since reset
\end{small}
\end{itemize}
\paragraph*{Rx statistics}\mbox{}\\\vskip 6pt
\rowcolors{1}{white}{white}
\begin{tabular}{l l }
{\bf HW prefix:}  & wr\_streamers\_rx\_stat7\\
{\bf HW address:}  & 0xb\\
{\bf SW prefix:}  & RX\_STAT7\\
{\bf SW offset:}  & 0x2c\\
\end{tabular}


\vspace{12pt}
\noindent
\resizebox{\textwidth}{!}{
\begin{tabular}{>{\centering\arraybackslash}p{1.5cm} >{\centering\arraybackslash}p{1.5cm} >{\centering\arraybackslash}p{1.5cm} >{\centering\arraybackslash}p{1.5cm} >{\centering\arraybackslash}p{1.5cm} >{\centering\arraybackslash}p{1.5cm} >{\centering\arraybackslash}p{1.5cm} >{\centering\arraybackslash}p{1.5cm} }
31 & 30 & 29 & 28 & 27 & 26 & 25 & 24\\
\hline
\multicolumn{8}{|c|}{\cellcolor{RoyalPurple!25}RX\_LOSS\_CNT\_MSB[31:24]}\\
\hline
23 & 22 & 21 & 20 & 19 & 18 & 17 & 16\\
\hline
\multicolumn{8}{|c|}{\cellcolor{RoyalPurple!25}RX\_LOSS\_CNT\_MSB[23:16]}\\
\hline
15 & 14 & 13 & 12 & 11 & 10 & 9 & 8\\
\hline
\multicolumn{8}{|c|}{\cellcolor{RoyalPurple!25}RX\_LOSS\_CNT\_MSB[15:8]}\\
\hline
7 & 6 & 5 & 4 & 3 & 2 & 1 & 0\\
\hline
\multicolumn{8}{|c|}{\cellcolor{RoyalPurple!25}RX\_LOSS\_CNT\_MSB[7:0]}\\
\hline
\end{tabular}
}

\begin{itemize}
\item \begin{small}
{\bf 
RX\_LOSS\_CNT\_MSB
} [\emph{read-only}]: WR Streamer frame loss count (MSB)
\\
Number of lost wr streamer frames since reset
\end{small}
\end{itemize}
\paragraph*{Rx statistics}\mbox{}\\\vskip 6pt
\rowcolors{1}{white}{white}
\begin{tabular}{l l }
{\bf HW prefix:}  & wr\_streamers\_rx\_stat8\\
{\bf HW address:}  & 0xc\\
{\bf SW prefix:}  & RX\_STAT8\\
{\bf SW offset:}  & 0x30\\
\end{tabular}


\vspace{12pt}
\noindent
\resizebox{\textwidth}{!}{
\begin{tabular}{>{\centering\arraybackslash}p{1.5cm} >{\centering\arraybackslash}p{1.5cm} >{\centering\arraybackslash}p{1.5cm} >{\centering\arraybackslash}p{1.5cm} >{\centering\arraybackslash}p{1.5cm} >{\centering\arraybackslash}p{1.5cm} >{\centering\arraybackslash}p{1.5cm} >{\centering\arraybackslash}p{1.5cm} }
31 & 30 & 29 & 28 & 27 & 26 & 25 & 24\\
\hline
\multicolumn{8}{|c|}{\cellcolor{RoyalPurple!25}RX\_LOST\_BLOCK\_CNT\_LSB[31:24]}\\
\hline
23 & 22 & 21 & 20 & 19 & 18 & 17 & 16\\
\hline
\multicolumn{8}{|c|}{\cellcolor{RoyalPurple!25}RX\_LOST\_BLOCK\_CNT\_LSB[23:16]}\\
\hline
15 & 14 & 13 & 12 & 11 & 10 & 9 & 8\\
\hline
\multicolumn{8}{|c|}{\cellcolor{RoyalPurple!25}RX\_LOST\_BLOCK\_CNT\_LSB[15:8]}\\
\hline
7 & 6 & 5 & 4 & 3 & 2 & 1 & 0\\
\hline
\multicolumn{8}{|c|}{\cellcolor{RoyalPurple!25}RX\_LOST\_BLOCK\_CNT\_LSB[7:0]}\\
\hline
\end{tabular}
}

\begin{itemize}
\item \begin{small}
{\bf 
RX\_LOST\_BLOCK\_CNT\_LSB
} [\emph{read-only}]: WR Streamer block loss count (LSB)
\\
Number of indications that one or more blocks in a frame were lost (probably CRC\\                     error) since reset
\end{small}
\end{itemize}
\paragraph*{Rx statistics}\mbox{}\\\vskip 6pt
\rowcolors{1}{white}{white}
\begin{tabular}{l l }
{\bf HW prefix:}  & wr\_streamers\_rx\_stat9\\
{\bf HW address:}  & 0xd\\
{\bf SW prefix:}  & RX\_STAT9\\
{\bf SW offset:}  & 0x34\\
\end{tabular}


\vspace{12pt}
\noindent
\resizebox{\textwidth}{!}{
\begin{tabular}{>{\centering\arraybackslash}p{1.5cm} >{\centering\arraybackslash}p{1.5cm} >{\centering\arraybackslash}p{1.5cm} >{\centering\arraybackslash}p{1.5cm} >{\centering\arraybackslash}p{1.5cm} >{\centering\arraybackslash}p{1.5cm} >{\centering\arraybackslash}p{1.5cm} >{\centering\arraybackslash}p{1.5cm} }
31 & 30 & 29 & 28 & 27 & 26 & 25 & 24\\
\hline
\multicolumn{8}{|c|}{\cellcolor{RoyalPurple!25}RX\_LOST\_BLOCK\_CNT\_MSB[31:24]}\\
\hline
23 & 22 & 21 & 20 & 19 & 18 & 17 & 16\\
\hline
\multicolumn{8}{|c|}{\cellcolor{RoyalPurple!25}RX\_LOST\_BLOCK\_CNT\_MSB[23:16]}\\
\hline
15 & 14 & 13 & 12 & 11 & 10 & 9 & 8\\
\hline
\multicolumn{8}{|c|}{\cellcolor{RoyalPurple!25}RX\_LOST\_BLOCK\_CNT\_MSB[15:8]}\\
\hline
7 & 6 & 5 & 4 & 3 & 2 & 1 & 0\\
\hline
\multicolumn{8}{|c|}{\cellcolor{RoyalPurple!25}RX\_LOST\_BLOCK\_CNT\_MSB[7:0]}\\
\hline
\end{tabular}
}

\begin{itemize}
\item \begin{small}
{\bf 
RX\_LOST\_BLOCK\_CNT\_MSB
} [\emph{read-only}]: WR Streamer block loss count (MSB)
\\
Number of indications that one or more blocks in a frame were lost (probably CRC\\                     error) since reset
\end{small}
\end{itemize}
\paragraph*{Rx statistics}\mbox{}\\\vskip 6pt
\rowcolors{1}{white}{white}
\begin{tabular}{l l }
{\bf HW prefix:}  & wr\_streamers\_rx\_stat10\\
{\bf HW address:}  & 0xe\\
{\bf SW prefix:}  & RX\_STAT10\\
{\bf SW offset:}  & 0x38\\
\end{tabular}


\vspace{12pt}
\noindent
\resizebox{\textwidth}{!}{
\begin{tabular}{>{\centering\arraybackslash}p{1.5cm} >{\centering\arraybackslash}p{1.5cm} >{\centering\arraybackslash}p{1.5cm} >{\centering\arraybackslash}p{1.5cm} >{\centering\arraybackslash}p{1.5cm} >{\centering\arraybackslash}p{1.5cm} >{\centering\arraybackslash}p{1.5cm} >{\centering\arraybackslash}p{1.5cm} }
31 & 30 & 29 & 28 & 27 & 26 & 25 & 24\\
\hline
\multicolumn{8}{|c|}{\cellcolor{RoyalPurple!25}RX\_LATENCY\_ACC\_LSB[31:24]}\\
\hline
23 & 22 & 21 & 20 & 19 & 18 & 17 & 16\\
\hline
\multicolumn{8}{|c|}{\cellcolor{RoyalPurple!25}RX\_LATENCY\_ACC\_LSB[23:16]}\\
\hline
15 & 14 & 13 & 12 & 11 & 10 & 9 & 8\\
\hline
\multicolumn{8}{|c|}{\cellcolor{RoyalPurple!25}RX\_LATENCY\_ACC\_LSB[15:8]}\\
\hline
7 & 6 & 5 & 4 & 3 & 2 & 1 & 0\\
\hline
\multicolumn{8}{|c|}{\cellcolor{RoyalPurple!25}RX\_LATENCY\_ACC\_LSB[7:0]}\\
\hline
\end{tabular}
}

\begin{itemize}
\item \begin{small}
{\bf 
RX\_LATENCY\_ACC\_LSB
} [\emph{read-only}]: WR Streamer frame latency (LSB)
\\
Accumulated latency of received frames since reset
\end{small}
\end{itemize}
\paragraph*{Rx statistics}\mbox{}\\\vskip 6pt
\rowcolors{1}{white}{white}
\begin{tabular}{l l }
{\bf HW prefix:}  & wr\_streamers\_rx\_stat11\\
{\bf HW address:}  & 0xf\\
{\bf SW prefix:}  & RX\_STAT11\\
{\bf SW offset:}  & 0x3c\\
\end{tabular}


\vspace{12pt}
\noindent
\resizebox{\textwidth}{!}{
\begin{tabular}{>{\centering\arraybackslash}p{1.5cm} >{\centering\arraybackslash}p{1.5cm} >{\centering\arraybackslash}p{1.5cm} >{\centering\arraybackslash}p{1.5cm} >{\centering\arraybackslash}p{1.5cm} >{\centering\arraybackslash}p{1.5cm} >{\centering\arraybackslash}p{1.5cm} >{\centering\arraybackslash}p{1.5cm} }
31 & 30 & 29 & 28 & 27 & 26 & 25 & 24\\
\hline
\multicolumn{8}{|c|}{\cellcolor{RoyalPurple!25}RX\_LATENCY\_ACC\_MSB[31:24]}\\
\hline
23 & 22 & 21 & 20 & 19 & 18 & 17 & 16\\
\hline
\multicolumn{8}{|c|}{\cellcolor{RoyalPurple!25}RX\_LATENCY\_ACC\_MSB[23:16]}\\
\hline
15 & 14 & 13 & 12 & 11 & 10 & 9 & 8\\
\hline
\multicolumn{8}{|c|}{\cellcolor{RoyalPurple!25}RX\_LATENCY\_ACC\_MSB[15:8]}\\
\hline
7 & 6 & 5 & 4 & 3 & 2 & 1 & 0\\
\hline
\multicolumn{8}{|c|}{\cellcolor{RoyalPurple!25}RX\_LATENCY\_ACC\_MSB[7:0]}\\
\hline
\end{tabular}
}

\begin{itemize}
\item \begin{small}
{\bf 
RX\_LATENCY\_ACC\_MSB
} [\emph{read-only}]: WR Streamer frame latency (MSB)
\\
Accumulated latency of received frames since reset
\end{small}
\end{itemize}
\paragraph*{Rx statistics}\mbox{}\\\vskip 6pt
\rowcolors{1}{white}{white}
\begin{tabular}{l l }
{\bf HW prefix:}  & wr\_streamers\_rx\_stat12\\
{\bf HW address:}  & 0x10\\
{\bf SW prefix:}  & RX\_STAT12\\
{\bf SW offset:}  & 0x40\\
\end{tabular}


\vspace{12pt}
\noindent
\resizebox{\textwidth}{!}{
\begin{tabular}{>{\centering\arraybackslash}p{1.5cm} >{\centering\arraybackslash}p{1.5cm} >{\centering\arraybackslash}p{1.5cm} >{\centering\arraybackslash}p{1.5cm} >{\centering\arraybackslash}p{1.5cm} >{\centering\arraybackslash}p{1.5cm} >{\centering\arraybackslash}p{1.5cm} >{\centering\arraybackslash}p{1.5cm} }
31 & 30 & 29 & 28 & 27 & 26 & 25 & 24\\
\hline
\multicolumn{8}{|c|}{\cellcolor{RoyalPurple!25}RX\_LATENCY\_ACC\_CNT\_LSB[31:24]}\\
\hline
23 & 22 & 21 & 20 & 19 & 18 & 17 & 16\\
\hline
\multicolumn{8}{|c|}{\cellcolor{RoyalPurple!25}RX\_LATENCY\_ACC\_CNT\_LSB[23:16]}\\
\hline
15 & 14 & 13 & 12 & 11 & 10 & 9 & 8\\
\hline
\multicolumn{8}{|c|}{\cellcolor{RoyalPurple!25}RX\_LATENCY\_ACC\_CNT\_LSB[15:8]}\\
\hline
7 & 6 & 5 & 4 & 3 & 2 & 1 & 0\\
\hline
\multicolumn{8}{|c|}{\cellcolor{RoyalPurple!25}RX\_LATENCY\_ACC\_CNT\_LSB[7:0]}\\
\hline
\end{tabular}
}

\begin{itemize}
\item \begin{small}
{\bf 
RX\_LATENCY\_ACC\_CNT\_LSB
} [\emph{read-only}]: WR Streamer frame latency counter (LSB)
\\
Counter of the accumulated frequency (so avg can be calculated in SW) since reset
\end{small}
\end{itemize}
\paragraph*{Rx statistics}\mbox{}\\\vskip 6pt
\rowcolors{1}{white}{white}
\begin{tabular}{l l }
{\bf HW prefix:}  & wr\_streamers\_rx\_stat13\\
{\bf HW address:}  & 0x11\\
{\bf SW prefix:}  & RX\_STAT13\\
{\bf SW offset:}  & 0x44\\
\end{tabular}


\vspace{12pt}
\noindent
\resizebox{\textwidth}{!}{
\begin{tabular}{>{\centering\arraybackslash}p{1.5cm} >{\centering\arraybackslash}p{1.5cm} >{\centering\arraybackslash}p{1.5cm} >{\centering\arraybackslash}p{1.5cm} >{\centering\arraybackslash}p{1.5cm} >{\centering\arraybackslash}p{1.5cm} >{\centering\arraybackslash}p{1.5cm} >{\centering\arraybackslash}p{1.5cm} }
31 & 30 & 29 & 28 & 27 & 26 & 25 & 24\\
\hline
\multicolumn{8}{|c|}{\cellcolor{RoyalPurple!25}RX\_LATENCY\_ACC\_CNT\_MSB[31:24]}\\
\hline
23 & 22 & 21 & 20 & 19 & 18 & 17 & 16\\
\hline
\multicolumn{8}{|c|}{\cellcolor{RoyalPurple!25}RX\_LATENCY\_ACC\_CNT\_MSB[23:16]}\\
\hline
15 & 14 & 13 & 12 & 11 & 10 & 9 & 8\\
\hline
\multicolumn{8}{|c|}{\cellcolor{RoyalPurple!25}RX\_LATENCY\_ACC\_CNT\_MSB[15:8]}\\
\hline
7 & 6 & 5 & 4 & 3 & 2 & 1 & 0\\
\hline
\multicolumn{8}{|c|}{\cellcolor{RoyalPurple!25}RX\_LATENCY\_ACC\_CNT\_MSB[7:0]}\\
\hline
\end{tabular}
}

\begin{itemize}
\item \begin{small}
{\bf 
RX\_LATENCY\_ACC\_CNT\_MSB
} [\emph{read-only}]: WR Streamer frame latency counter (MSB)
\\
Counter of the accumulated frequency (so avg can be calculated in SW) since reset
\end{small}
\end{itemize}
\paragraph*{Tx Config Reg 0}\mbox{}\\\vskip 6pt
\rowcolors{1}{white}{white}
\begin{tabular}{l l }
{\bf HW prefix:}  & wr\_streamers\_tx\_cfg0\\
{\bf HW address:}  & 0x12\\
{\bf SW prefix:}  & TX\_CFG0\\
{\bf SW offset:}  & 0x48\\
\end{tabular}


\vspace{12pt}
\noindent
\resizebox{\textwidth}{!}{
\begin{tabular}{>{\centering\arraybackslash}p{1.5cm} >{\centering\arraybackslash}p{1.5cm} >{\centering\arraybackslash}p{1.5cm} >{\centering\arraybackslash}p{1.5cm} >{\centering\arraybackslash}p{1.5cm} >{\centering\arraybackslash}p{1.5cm} >{\centering\arraybackslash}p{1.5cm} >{\centering\arraybackslash}p{1.5cm} }
31 & 30 & 29 & 28 & 27 & 26 & 25 & 24\\
\hline
\multicolumn{1}{|c}{-} & - & - & - & - & - & - & \multicolumn{1}{c|}{-}\\
\hline
23 & 22 & 21 & 20 & 19 & 18 & 17 & 16\\
\hline
\multicolumn{1}{|c}{-} & - & - & - & - & - & - & \multicolumn{1}{c|}{-}\\
\hline
15 & 14 & 13 & 12 & 11 & 10 & 9 & 8\\
\hline
\multicolumn{8}{|c|}{\cellcolor{RoyalPurple!25}ETHERTYPE[15:8]}\\
\hline
7 & 6 & 5 & 4 & 3 & 2 & 1 & 0\\
\hline
\multicolumn{8}{|c|}{\cellcolor{RoyalPurple!25}ETHERTYPE[7:0]}\\
\hline
\end{tabular}
}

\begin{itemize}
\item \begin{small}
{\bf 
ETHERTYPE
} [\emph{read/write}]: Ethertype
\end{small}
\end{itemize}
\paragraph*{Tx Config Reg 1}\mbox{}\\\vskip 6pt
\rowcolors{1}{white}{white}
\begin{tabular}{l l }
{\bf HW prefix:}  & wr\_streamers\_tx\_cfg1\\
{\bf HW address:}  & 0x13\\
{\bf SW prefix:}  & TX\_CFG1\\
{\bf SW offset:}  & 0x4c\\
\end{tabular}


\vspace{12pt}
\noindent
\resizebox{\textwidth}{!}{
\begin{tabular}{>{\centering\arraybackslash}p{1.5cm} >{\centering\arraybackslash}p{1.5cm} >{\centering\arraybackslash}p{1.5cm} >{\centering\arraybackslash}p{1.5cm} >{\centering\arraybackslash}p{1.5cm} >{\centering\arraybackslash}p{1.5cm} >{\centering\arraybackslash}p{1.5cm} >{\centering\arraybackslash}p{1.5cm} }
31 & 30 & 29 & 28 & 27 & 26 & 25 & 24\\
\hline
\multicolumn{8}{|c|}{\cellcolor{RoyalPurple!25}MAC\_LOCAL\_LSB[31:24]}\\
\hline
23 & 22 & 21 & 20 & 19 & 18 & 17 & 16\\
\hline
\multicolumn{8}{|c|}{\cellcolor{RoyalPurple!25}MAC\_LOCAL\_LSB[23:16]}\\
\hline
15 & 14 & 13 & 12 & 11 & 10 & 9 & 8\\
\hline
\multicolumn{8}{|c|}{\cellcolor{RoyalPurple!25}MAC\_LOCAL\_LSB[15:8]}\\
\hline
7 & 6 & 5 & 4 & 3 & 2 & 1 & 0\\
\hline
\multicolumn{8}{|c|}{\cellcolor{RoyalPurple!25}MAC\_LOCAL\_LSB[7:0]}\\
\hline
\end{tabular}
}

\begin{itemize}
\item \begin{small}
{\bf 
MAC\_LOCAL\_LSB
} [\emph{read/write}]: MAC Local LSB
\end{small}
\end{itemize}
\paragraph*{Tx Config Reg 2}\mbox{}\\\vskip 6pt
\rowcolors{1}{white}{white}
\begin{tabular}{l l }
{\bf HW prefix:}  & wr\_streamers\_tx\_cfg2\\
{\bf HW address:}  & 0x14\\
{\bf SW prefix:}  & TX\_CFG2\\
{\bf SW offset:}  & 0x50\\
\end{tabular}


\vspace{12pt}
\noindent
\resizebox{\textwidth}{!}{
\begin{tabular}{>{\centering\arraybackslash}p{1.5cm} >{\centering\arraybackslash}p{1.5cm} >{\centering\arraybackslash}p{1.5cm} >{\centering\arraybackslash}p{1.5cm} >{\centering\arraybackslash}p{1.5cm} >{\centering\arraybackslash}p{1.5cm} >{\centering\arraybackslash}p{1.5cm} >{\centering\arraybackslash}p{1.5cm} }
31 & 30 & 29 & 28 & 27 & 26 & 25 & 24\\
\hline
\multicolumn{1}{|c}{-} & - & - & - & - & - & - & \multicolumn{1}{c|}{-}\\
\hline
23 & 22 & 21 & 20 & 19 & 18 & 17 & 16\\
\hline
\multicolumn{1}{|c}{-} & - & - & - & - & - & - & \multicolumn{1}{c|}{-}\\
\hline
15 & 14 & 13 & 12 & 11 & 10 & 9 & 8\\
\hline
\multicolumn{8}{|c|}{\cellcolor{RoyalPurple!25}MAC\_LOCAL\_MSB[15:8]}\\
\hline
7 & 6 & 5 & 4 & 3 & 2 & 1 & 0\\
\hline
\multicolumn{8}{|c|}{\cellcolor{RoyalPurple!25}MAC\_LOCAL\_MSB[7:0]}\\
\hline
\end{tabular}
}

\begin{itemize}
\item \begin{small}
{\bf 
MAC\_LOCAL\_MSB
} [\emph{read/write}]: MAC Local MSB
\end{small}
\end{itemize}
\paragraph*{Tx Config Reg 3}\mbox{}\\\vskip 6pt
\rowcolors{1}{white}{white}
\begin{tabular}{l l }
{\bf HW prefix:}  & wr\_streamers\_tx\_cfg3\\
{\bf HW address:}  & 0x15\\
{\bf SW prefix:}  & TX\_CFG3\\
{\bf SW offset:}  & 0x54\\
\end{tabular}


\vspace{12pt}
\noindent
\resizebox{\textwidth}{!}{
\begin{tabular}{>{\centering\arraybackslash}p{1.5cm} >{\centering\arraybackslash}p{1.5cm} >{\centering\arraybackslash}p{1.5cm} >{\centering\arraybackslash}p{1.5cm} >{\centering\arraybackslash}p{1.5cm} >{\centering\arraybackslash}p{1.5cm} >{\centering\arraybackslash}p{1.5cm} >{\centering\arraybackslash}p{1.5cm} }
31 & 30 & 29 & 28 & 27 & 26 & 25 & 24\\
\hline
\multicolumn{8}{|c|}{\cellcolor{RoyalPurple!25}MAC\_TARGET\_LSB[31:24]}\\
\hline
23 & 22 & 21 & 20 & 19 & 18 & 17 & 16\\
\hline
\multicolumn{8}{|c|}{\cellcolor{RoyalPurple!25}MAC\_TARGET\_LSB[23:16]}\\
\hline
15 & 14 & 13 & 12 & 11 & 10 & 9 & 8\\
\hline
\multicolumn{8}{|c|}{\cellcolor{RoyalPurple!25}MAC\_TARGET\_LSB[15:8]}\\
\hline
7 & 6 & 5 & 4 & 3 & 2 & 1 & 0\\
\hline
\multicolumn{8}{|c|}{\cellcolor{RoyalPurple!25}MAC\_TARGET\_LSB[7:0]}\\
\hline
\end{tabular}
}

\begin{itemize}
\item \begin{small}
{\bf 
MAC\_TARGET\_LSB
} [\emph{read/write}]: MAC Target LSB
\end{small}
\end{itemize}
\paragraph*{Tx Config Reg 4}\mbox{}\\\vskip 6pt
\rowcolors{1}{white}{white}
\begin{tabular}{l l }
{\bf HW prefix:}  & wr\_streamers\_tx\_cfg4\\
{\bf HW address:}  & 0x16\\
{\bf SW prefix:}  & TX\_CFG4\\
{\bf SW offset:}  & 0x58\\
\end{tabular}


\vspace{12pt}
\noindent
\resizebox{\textwidth}{!}{
\begin{tabular}{>{\centering\arraybackslash}p{1.5cm} >{\centering\arraybackslash}p{1.5cm} >{\centering\arraybackslash}p{1.5cm} >{\centering\arraybackslash}p{1.5cm} >{\centering\arraybackslash}p{1.5cm} >{\centering\arraybackslash}p{1.5cm} >{\centering\arraybackslash}p{1.5cm} >{\centering\arraybackslash}p{1.5cm} }
31 & 30 & 29 & 28 & 27 & 26 & 25 & 24\\
\hline
\multicolumn{1}{|c}{-} & - & - & - & - & - & - & \multicolumn{1}{c|}{-}\\
\hline
23 & 22 & 21 & 20 & 19 & 18 & 17 & 16\\
\hline
\multicolumn{1}{|c}{-} & - & - & - & - & - & - & \multicolumn{1}{c|}{-}\\
\hline
15 & 14 & 13 & 12 & 11 & 10 & 9 & 8\\
\hline
\multicolumn{8}{|c|}{\cellcolor{RoyalPurple!25}MAC\_TARGET\_MSB[15:8]}\\
\hline
7 & 6 & 5 & 4 & 3 & 2 & 1 & 0\\
\hline
\multicolumn{8}{|c|}{\cellcolor{RoyalPurple!25}MAC\_TARGET\_MSB[7:0]}\\
\hline
\end{tabular}
}

\begin{itemize}
\item \begin{small}
{\bf 
MAC\_TARGET\_MSB
} [\emph{read/write}]: MAC Target MSB
\end{small}
\end{itemize}
\paragraph*{Tx Config Reg 4}\mbox{}\\\vskip 6pt
\rowcolors{1}{white}{white}
\begin{tabular}{l l }
{\bf HW prefix:}  & wr\_streamers\_tx\_cfg5\\
{\bf HW address:}  & 0x17\\
{\bf SW prefix:}  & TX\_CFG5\\
{\bf SW offset:}  & 0x5c\\
\end{tabular}


\vspace{12pt}
\noindent
\resizebox{\textwidth}{!}{
\begin{tabular}{>{\centering\arraybackslash}p{1.5cm} >{\centering\arraybackslash}p{1.5cm} >{\centering\arraybackslash}p{1.5cm} >{\centering\arraybackslash}p{1.5cm} >{\centering\arraybackslash}p{1.5cm} >{\centering\arraybackslash}p{1.5cm} >{\centering\arraybackslash}p{1.5cm} >{\centering\arraybackslash}p{1.5cm} }
31 & 30 & 29 & 28 & 27 & 26 & 25 & 24\\
\hline
\multicolumn{1}{|c}{-} & - & - & - & - & \multicolumn{3}{|c|}{\cellcolor{RoyalPurple!25}QTAG\_PRIO[2:0]}\\
\hline
23 & 22 & 21 & 20 & 19 & 18 & 17 & 16\\
\hline
\multicolumn{1}{|c}{-} & - & - & - & \multicolumn{4}{|c|}{\cellcolor{RoyalPurple!25}QTAG\_VID[11:8]}\\
\hline
15 & 14 & 13 & 12 & 11 & 10 & 9 & 8\\
\hline
\multicolumn{8}{|c|}{\cellcolor{RoyalPurple!25}QTAG\_VID[7:0]}\\
\hline
7 & 6 & 5 & 4 & 3 & 2 & 1 & 0\\
\hline
\multicolumn{1}{|c}{-} & - & - & - & - & - & - & \multicolumn{1}{|c|}{\cellcolor{RoyalPurple!25}QTAG\_ENA}\\
\hline
\end{tabular}
}

\begin{itemize}
\item \begin{small}
{\bf 
QTAG\_ENA
} [\emph{read/write}]: Enable tagging with Qtags
\end{small}
\item \begin{small}
{\bf 
QTAG\_VID
} [\emph{read/write}]: VLAN ID
\end{small}
\item \begin{small}
{\bf 
QTAG\_PRIO
} [\emph{read/write}]: Priority
\end{small}
\end{itemize}
\paragraph*{Rx Config Reg 0}\mbox{}\\\vskip 6pt
\rowcolors{1}{white}{white}
\begin{tabular}{l l }
{\bf HW prefix:}  & wr\_streamers\_rx\_cfg0\\
{\bf HW address:}  & 0x18\\
{\bf SW prefix:}  & RX\_CFG0\\
{\bf SW offset:}  & 0x60\\
\end{tabular}


\vspace{12pt}
\noindent
\resizebox{\textwidth}{!}{
\begin{tabular}{>{\centering\arraybackslash}p{1.5cm} >{\centering\arraybackslash}p{1.5cm} >{\centering\arraybackslash}p{1.5cm} >{\centering\arraybackslash}p{1.5cm} >{\centering\arraybackslash}p{1.5cm} >{\centering\arraybackslash}p{1.5cm} >{\centering\arraybackslash}p{1.5cm} >{\centering\arraybackslash}p{1.5cm} }
31 & 30 & 29 & 28 & 27 & 26 & 25 & 24\\
\hline
\multicolumn{1}{|c}{-} & - & - & - & - & - & - & \multicolumn{1}{c|}{-}\\
\hline
23 & 22 & 21 & 20 & 19 & 18 & 17 & 16\\
\hline
\multicolumn{1}{|c}{-} & - & - & - & - & - & \multicolumn{1}{|c|}{\cellcolor{RoyalPurple!25}FILTER\_REMOTE} & \multicolumn{1}{|c|}{\cellcolor{RoyalPurple!25}ACCEPT\_BROADCAST}\\
\hline
15 & 14 & 13 & 12 & 11 & 10 & 9 & 8\\
\hline
\multicolumn{8}{|c|}{\cellcolor{RoyalPurple!25}ETHERTYPE[15:8]}\\
\hline
7 & 6 & 5 & 4 & 3 & 2 & 1 & 0\\
\hline
\multicolumn{8}{|c|}{\cellcolor{RoyalPurple!25}ETHERTYPE[7:0]}\\
\hline
\end{tabular}
}

\begin{itemize}
\item \begin{small}
{\bf 
ETHERTYPE
} [\emph{read/write}]: Ethertype
\end{small}
\item \begin{small}
{\bf 
ACCEPT\_BROADCAST
} [\emph{read/write}]: Accept Broadcast
\\
0: accept only unicasts; \\      1: accept all broadcast packets
\end{small}
\item \begin{small}
{\bf 
FILTER\_REMOTE
} [\emph{read/write}]: Filter Remote
\\
0: accept streamer frames with any source MAC address; \\      1: accept streamer frames only with the source MAC address defined in mac\_remote
\end{small}
\end{itemize}
\paragraph*{Rx Config Reg 1}\mbox{}\\\vskip 6pt
\rowcolors{1}{white}{white}
\begin{tabular}{l l }
{\bf HW prefix:}  & wr\_streamers\_rx\_cfg1\\
{\bf HW address:}  & 0x19\\
{\bf SW prefix:}  & RX\_CFG1\\
{\bf SW offset:}  & 0x64\\
\end{tabular}


\vspace{12pt}
\noindent
\resizebox{\textwidth}{!}{
\begin{tabular}{>{\centering\arraybackslash}p{1.5cm} >{\centering\arraybackslash}p{1.5cm} >{\centering\arraybackslash}p{1.5cm} >{\centering\arraybackslash}p{1.5cm} >{\centering\arraybackslash}p{1.5cm} >{\centering\arraybackslash}p{1.5cm} >{\centering\arraybackslash}p{1.5cm} >{\centering\arraybackslash}p{1.5cm} }
31 & 30 & 29 & 28 & 27 & 26 & 25 & 24\\
\hline
\multicolumn{8}{|c|}{\cellcolor{RoyalPurple!25}MAC\_LOCAL\_LSB[31:24]}\\
\hline
23 & 22 & 21 & 20 & 19 & 18 & 17 & 16\\
\hline
\multicolumn{8}{|c|}{\cellcolor{RoyalPurple!25}MAC\_LOCAL\_LSB[23:16]}\\
\hline
15 & 14 & 13 & 12 & 11 & 10 & 9 & 8\\
\hline
\multicolumn{8}{|c|}{\cellcolor{RoyalPurple!25}MAC\_LOCAL\_LSB[15:8]}\\
\hline
7 & 6 & 5 & 4 & 3 & 2 & 1 & 0\\
\hline
\multicolumn{8}{|c|}{\cellcolor{RoyalPurple!25}MAC\_LOCAL\_LSB[7:0]}\\
\hline
\end{tabular}
}

\begin{itemize}
\item \begin{small}
{\bf 
MAC\_LOCAL\_LSB
} [\emph{read/write}]: MAC Local LSB
\end{small}
\end{itemize}
\paragraph*{Rx Config Reg 2}\mbox{}\\\vskip 6pt
\rowcolors{1}{white}{white}
\begin{tabular}{l l }
{\bf HW prefix:}  & wr\_streamers\_rx\_cfg2\\
{\bf HW address:}  & 0x1a\\
{\bf SW prefix:}  & RX\_CFG2\\
{\bf SW offset:}  & 0x68\\
\end{tabular}


\vspace{12pt}
\noindent
\resizebox{\textwidth}{!}{
\begin{tabular}{>{\centering\arraybackslash}p{1.5cm} >{\centering\arraybackslash}p{1.5cm} >{\centering\arraybackslash}p{1.5cm} >{\centering\arraybackslash}p{1.5cm} >{\centering\arraybackslash}p{1.5cm} >{\centering\arraybackslash}p{1.5cm} >{\centering\arraybackslash}p{1.5cm} >{\centering\arraybackslash}p{1.5cm} }
31 & 30 & 29 & 28 & 27 & 26 & 25 & 24\\
\hline
\multicolumn{1}{|c}{-} & - & - & - & - & - & - & \multicolumn{1}{c|}{-}\\
\hline
23 & 22 & 21 & 20 & 19 & 18 & 17 & 16\\
\hline
\multicolumn{1}{|c}{-} & - & - & - & - & - & - & \multicolumn{1}{c|}{-}\\
\hline
15 & 14 & 13 & 12 & 11 & 10 & 9 & 8\\
\hline
\multicolumn{8}{|c|}{\cellcolor{RoyalPurple!25}MAC\_LOCAL\_MSB[15:8]}\\
\hline
7 & 6 & 5 & 4 & 3 & 2 & 1 & 0\\
\hline
\multicolumn{8}{|c|}{\cellcolor{RoyalPurple!25}MAC\_LOCAL\_MSB[7:0]}\\
\hline
\end{tabular}
}

\begin{itemize}
\item \begin{small}
{\bf 
MAC\_LOCAL\_MSB
} [\emph{read/write}]: MAC Local MSB
\end{small}
\end{itemize}
\paragraph*{Rx Config Reg 3}\mbox{}\\\vskip 6pt
\rowcolors{1}{white}{white}
\begin{tabular}{l l }
{\bf HW prefix:}  & wr\_streamers\_rx\_cfg3\\
{\bf HW address:}  & 0x1b\\
{\bf SW prefix:}  & RX\_CFG3\\
{\bf SW offset:}  & 0x6c\\
\end{tabular}


\vspace{12pt}
\noindent
\resizebox{\textwidth}{!}{
\begin{tabular}{>{\centering\arraybackslash}p{1.5cm} >{\centering\arraybackslash}p{1.5cm} >{\centering\arraybackslash}p{1.5cm} >{\centering\arraybackslash}p{1.5cm} >{\centering\arraybackslash}p{1.5cm} >{\centering\arraybackslash}p{1.5cm} >{\centering\arraybackslash}p{1.5cm} >{\centering\arraybackslash}p{1.5cm} }
31 & 30 & 29 & 28 & 27 & 26 & 25 & 24\\
\hline
\multicolumn{8}{|c|}{\cellcolor{RoyalPurple!25}MAC\_REMOTE\_LSB[31:24]}\\
\hline
23 & 22 & 21 & 20 & 19 & 18 & 17 & 16\\
\hline
\multicolumn{8}{|c|}{\cellcolor{RoyalPurple!25}MAC\_REMOTE\_LSB[23:16]}\\
\hline
15 & 14 & 13 & 12 & 11 & 10 & 9 & 8\\
\hline
\multicolumn{8}{|c|}{\cellcolor{RoyalPurple!25}MAC\_REMOTE\_LSB[15:8]}\\
\hline
7 & 6 & 5 & 4 & 3 & 2 & 1 & 0\\
\hline
\multicolumn{8}{|c|}{\cellcolor{RoyalPurple!25}MAC\_REMOTE\_LSB[7:0]}\\
\hline
\end{tabular}
}

\begin{itemize}
\item \begin{small}
{\bf 
MAC\_REMOTE\_LSB
} [\emph{read/write}]: MAC Remote LSB
\end{small}
\end{itemize}
\paragraph*{Rx Config Reg 4}\mbox{}\\\vskip 6pt
\rowcolors{1}{white}{white}
\begin{tabular}{l l }
{\bf HW prefix:}  & wr\_streamers\_rx\_cfg4\\
{\bf HW address:}  & 0x1c\\
{\bf SW prefix:}  & RX\_CFG4\\
{\bf SW offset:}  & 0x70\\
\end{tabular}


\vspace{12pt}
\noindent
\resizebox{\textwidth}{!}{
\begin{tabular}{>{\centering\arraybackslash}p{1.5cm} >{\centering\arraybackslash}p{1.5cm} >{\centering\arraybackslash}p{1.5cm} >{\centering\arraybackslash}p{1.5cm} >{\centering\arraybackslash}p{1.5cm} >{\centering\arraybackslash}p{1.5cm} >{\centering\arraybackslash}p{1.5cm} >{\centering\arraybackslash}p{1.5cm} }
31 & 30 & 29 & 28 & 27 & 26 & 25 & 24\\
\hline
\multicolumn{1}{|c}{-} & - & - & - & - & - & - & \multicolumn{1}{c|}{-}\\
\hline
23 & 22 & 21 & 20 & 19 & 18 & 17 & 16\\
\hline
\multicolumn{1}{|c}{-} & - & - & - & - & - & - & \multicolumn{1}{c|}{-}\\
\hline
15 & 14 & 13 & 12 & 11 & 10 & 9 & 8\\
\hline
\multicolumn{8}{|c|}{\cellcolor{RoyalPurple!25}MAC\_REMOTE\_MSB[15:8]}\\
\hline
7 & 6 & 5 & 4 & 3 & 2 & 1 & 0\\
\hline
\multicolumn{8}{|c|}{\cellcolor{RoyalPurple!25}MAC\_REMOTE\_MSB[7:0]}\\
\hline
\end{tabular}
}

\begin{itemize}
\item \begin{small}
{\bf 
MAC\_REMOTE\_MSB
} [\emph{read/write}]: MAC Remote MSB
\end{small}
\end{itemize}
\paragraph*{Rx Config Reg 5}\mbox{}\\\vskip 6pt
\rowcolors{1}{white}{white}
\begin{tabular}{l l }
{\bf HW prefix:}  & wr\_streamers\_rx\_cfg5\\
{\bf HW address:}  & 0x1d\\
{\bf SW prefix:}  & RX\_CFG5\\
{\bf SW offset:}  & 0x74\\
\end{tabular}


\vspace{12pt}
\noindent
\resizebox{\textwidth}{!}{
\begin{tabular}{>{\centering\arraybackslash}p{1.5cm} >{\centering\arraybackslash}p{1.5cm} >{\centering\arraybackslash}p{1.5cm} >{\centering\arraybackslash}p{1.5cm} >{\centering\arraybackslash}p{1.5cm} >{\centering\arraybackslash}p{1.5cm} >{\centering\arraybackslash}p{1.5cm} >{\centering\arraybackslash}p{1.5cm} }
31 & 30 & 29 & 28 & 27 & 26 & 25 & 24\\
\hline
\multicolumn{1}{|c}{-} & - & - & - & \multicolumn{4}{|c|}{\cellcolor{RoyalPurple!25}FIXED\_LATENCY[27:24]}\\
\hline
23 & 22 & 21 & 20 & 19 & 18 & 17 & 16\\
\hline
\multicolumn{8}{|c|}{\cellcolor{RoyalPurple!25}FIXED\_LATENCY[23:16]}\\
\hline
15 & 14 & 13 & 12 & 11 & 10 & 9 & 8\\
\hline
\multicolumn{8}{|c|}{\cellcolor{RoyalPurple!25}FIXED\_LATENCY[15:8]}\\
\hline
7 & 6 & 5 & 4 & 3 & 2 & 1 & 0\\
\hline
\multicolumn{8}{|c|}{\cellcolor{RoyalPurple!25}FIXED\_LATENCY[7:0]}\\
\hline
\end{tabular}
}

\begin{itemize}
\item \begin{small}
{\bf 
FIXED\_LATENCY
} [\emph{read/write}]: Fixed Latency
\\
This register allows to configure fixed-latency. If the value is other than zero, the instant of outputing the received data from the rx streamer to the user application is delayed, so that the time-difference between the transmission fo the data and the output to the user matches the provided value. If the configured latency value is smaller than the network latency, the data is provided to the user instantly. The configuration value is expressed in clock cycles (16ns) 
\end{small}
\end{itemize}
\paragraph*{TxRx Config Override}\mbox{}\\\vskip 6pt
\rowcolors{1}{white}{white}
\begin{tabular}{l l }
{\bf HW prefix:}  & wr\_streamers\_cfg\\
{\bf HW address:}  & 0x1e\\
{\bf SW prefix:}  & CFG\\
{\bf SW offset:}  & 0x78\\
\end{tabular}


\vspace{12pt}
\noindent
\resizebox{\textwidth}{!}{
\begin{tabular}{>{\centering\arraybackslash}p{1.5cm} >{\centering\arraybackslash}p{1.5cm} >{\centering\arraybackslash}p{1.5cm} >{\centering\arraybackslash}p{1.5cm} >{\centering\arraybackslash}p{1.5cm} >{\centering\arraybackslash}p{1.5cm} >{\centering\arraybackslash}p{1.5cm} >{\centering\arraybackslash}p{1.5cm} }
31 & 30 & 29 & 28 & 27 & 26 & 25 & 24\\
\hline
\multicolumn{1}{|c}{-} & - & - & - & - & - & - & \multicolumn{1}{c|}{-}\\
\hline
23 & 22 & 21 & 20 & 19 & 18 & 17 & 16\\
\hline
\multicolumn{1}{|c}{-} & - & \multicolumn{1}{|c|}{\cellcolor{RoyalPurple!25}OR\_RX\_FIX\_LAT} & \multicolumn{1}{|c|}{\cellcolor{RoyalPurple!25}OR\_RX\_FTR\_REMOTE} & \multicolumn{1}{|c|}{\cellcolor{RoyalPurple!25}OR\_RX\_ACC\_BROADCAST} & \multicolumn{1}{|c|}{\cellcolor{RoyalPurple!25}OR\_RX\_MAC\_REM} & \multicolumn{1}{|c|}{\cellcolor{RoyalPurple!25}OR\_RX\_MAC\_LOC} & \multicolumn{1}{|c|}{\cellcolor{RoyalPurple!25}OR\_RX\_ETHERTYPE}\\
\hline
15 & 14 & 13 & 12 & 11 & 10 & 9 & 8\\
\hline
\multicolumn{1}{|c}{-} & - & - & - & - & - & - & \multicolumn{1}{c|}{-}\\
\hline
7 & 6 & 5 & 4 & 3 & 2 & 1 & 0\\
\hline
\multicolumn{1}{|c}{-} & - & - & - & \multicolumn{1}{|c|}{\cellcolor{RoyalPurple!25}OR\_TX\_QTAG} & \multicolumn{1}{|c|}{\cellcolor{RoyalPurple!25}OR\_TX\_MAC\_TAR} & \multicolumn{1}{|c|}{\cellcolor{RoyalPurple!25}OR\_TX\_MAC\_LOC} & \multicolumn{1}{|c|}{\cellcolor{RoyalPurple!25}OR\_TX\_ETHTYPE}\\
\hline
\end{tabular}
}

\begin{itemize}
\item \begin{small}
{\bf 
OR\_TX\_ETHTYPE
} [\emph{read/write}]: Tx Ethertype
\\
Overrides default/application Tx Ethertype configuration with configuration in the proper register:\\      0: Default/set by application; \\      1: Value from WB register
\end{small}
\item \begin{small}
{\bf 
OR\_TX\_MAC\_LOC
} [\emph{read/write}]: Tx MAC Local
\\
Overrides default/application Tx local MAC configuration with configuration in the proper register:\\      0: Default/set by application; \\      1: Value from WB register
\end{small}
\item \begin{small}
{\bf 
OR\_TX\_MAC\_TAR
} [\emph{read/write}]: Tx MAC Target
\\
Overrides default/application Tx target MAC configuration with configuration in the proper register:\\      0: Default/set by application; \\      1: Value from WB register
\end{small}
\item \begin{small}
{\bf 
OR\_TX\_QTAG
} [\emph{read/write}]: QTAG
\\
Overrides default/application QTAG values with configuration in the proper register:\\      0: Default/set by application; \\      1: Value from WB register
\end{small}
\item \begin{small}
{\bf 
OR\_RX\_ETHERTYPE
} [\emph{read/write}]: Rx Ethertype
\\
Overrides default/application Rx Ethertype configuration with configuration in the proper register:\\      0: Default/set by application; \\      1: Value from WB register
\end{small}
\item \begin{small}
{\bf 
OR\_RX\_MAC\_LOC
} [\emph{read/write}]: Rx MAC Local
\\
Overrides default/application Rx MAC Local configuration with configuration in the proper register:\\      0: Default/set by application; \\      1: Value from WB register
\end{small}
\item \begin{small}
{\bf 
OR\_RX\_MAC\_REM
} [\emph{read/write}]: Rx MAC Remote
\\
Overrides default/application Rx MAC Remote configuration with configuration in the proper register:\\      0: Default/set by application; \\      1: Value from WB register
\end{small}
\item \begin{small}
{\bf 
OR\_RX\_ACC\_BROADCAST
} [\emph{read/write}]: Rx Accept Broadcast
\\
Overrides default/application Rx Accept Broardcast configuration with configuration in the proper register:\\      0: Default/set by application; \\      1: Value from WB register
\end{small}
\item \begin{small}
{\bf 
OR\_RX\_FTR\_REMOTE
} [\emph{read/write}]: Rx Filter Remote
\\
Overrides default/application Rx Filter Remote configuration with configuration in the proper register:\\      0: Default/set by application; \\      1: Value from WB register
\end{small}
\item \begin{small}
{\bf 
OR\_RX\_FIX\_LAT
} [\emph{read/write}]: Rx Fixed Latency 
\\
Overrides default/application Rx fixed latency configuration with configuration in the proper register:\\      0: Default/set by application; \\      1: Value from WB register
\end{small}
\end{itemize}
\paragraph*{DBG Control register}\mbox{}\\\vskip 6pt
\rowcolors{1}{white}{white}
\begin{tabular}{l l }
{\bf HW prefix:}  & wr\_streamers\_dbg\_ctrl\\
{\bf HW address:}  & 0x1f\\
{\bf SW prefix:}  & DBG\_CTRL\\
{\bf SW offset:}  & 0x7c\\
\end{tabular}

\vspace{12pt}
This register is meant to control simple debugging of transmitted or received data.\\    It allows to sniff a 32-bit word at a configurable offset from received or transmitted data.

\vspace{12pt}
\noindent
\resizebox{\textwidth}{!}{
\begin{tabular}{>{\centering\arraybackslash}p{1.5cm} >{\centering\arraybackslash}p{1.5cm} >{\centering\arraybackslash}p{1.5cm} >{\centering\arraybackslash}p{1.5cm} >{\centering\arraybackslash}p{1.5cm} >{\centering\arraybackslash}p{1.5cm} >{\centering\arraybackslash}p{1.5cm} >{\centering\arraybackslash}p{1.5cm} }
31 & 30 & 29 & 28 & 27 & 26 & 25 & 24\\
\hline
\multicolumn{1}{|c}{-} & - & - & - & - & - & - & \multicolumn{1}{c|}{-}\\
\hline
23 & 22 & 21 & 20 & 19 & 18 & 17 & 16\\
\hline
\multicolumn{1}{|c}{-} & - & - & - & - & - & - & \multicolumn{1}{c|}{-}\\
\hline
15 & 14 & 13 & 12 & 11 & 10 & 9 & 8\\
\hline
\multicolumn{8}{|c|}{\cellcolor{RoyalPurple!25}START\_BYTE[7:0]}\\
\hline
7 & 6 & 5 & 4 & 3 & 2 & 1 & 0\\
\hline
\multicolumn{1}{|c}{-} & - & - & - & - & - & - & \multicolumn{1}{|c|}{\cellcolor{RoyalPurple!25}MUX}\\
\hline
\end{tabular}
}

\begin{itemize}
\item \begin{small}
{\bf 
MUX
} [\emph{read/write}]: Debug Tx (0) or Rx (1)
\end{small}
\item \begin{small}
{\bf 
START\_BYTE
} [\emph{read/write}]: Debug Start byte
\\
The offset, in bytes, from which the 32-bit word is read.
\end{small}
\end{itemize}
\paragraph*{DBG Data}\mbox{}\\\vskip 6pt
\rowcolors{1}{white}{white}
\begin{tabular}{l l }
{\bf HW prefix:}  & wr\_streamers\_dbg\_data\\
{\bf HW address:}  & 0x20\\
{\bf SW prefix:}  & DBG\_DATA\\
{\bf SW offset:}  & 0x80\\
\end{tabular}


\vspace{12pt}
\noindent
\resizebox{\textwidth}{!}{
\begin{tabular}{>{\centering\arraybackslash}p{1.5cm} >{\centering\arraybackslash}p{1.5cm} >{\centering\arraybackslash}p{1.5cm} >{\centering\arraybackslash}p{1.5cm} >{\centering\arraybackslash}p{1.5cm} >{\centering\arraybackslash}p{1.5cm} >{\centering\arraybackslash}p{1.5cm} >{\centering\arraybackslash}p{1.5cm} }
31 & 30 & 29 & 28 & 27 & 26 & 25 & 24\\
\hline
\multicolumn{8}{|c|}{\cellcolor{RoyalPurple!25}DBG\_DATA[31:24]}\\
\hline
23 & 22 & 21 & 20 & 19 & 18 & 17 & 16\\
\hline
\multicolumn{8}{|c|}{\cellcolor{RoyalPurple!25}DBG\_DATA[23:16]}\\
\hline
15 & 14 & 13 & 12 & 11 & 10 & 9 & 8\\
\hline
\multicolumn{8}{|c|}{\cellcolor{RoyalPurple!25}DBG\_DATA[15:8]}\\
\hline
7 & 6 & 5 & 4 & 3 & 2 & 1 & 0\\
\hline
\multicolumn{8}{|c|}{\cellcolor{RoyalPurple!25}DBG\_DATA[7:0]}\\
\hline
\end{tabular}
}

\begin{itemize}
\item \begin{small}
{\bf 
DBG\_DATA
} [\emph{read-only}]: Debug content
\end{small}
\end{itemize}
\paragraph*{Test value}\mbox{}\\\vskip 6pt
\rowcolors{1}{white}{white}
\begin{tabular}{l l }
{\bf HW prefix:}  & wr\_streamers\_dummy\\
{\bf HW address:}  & 0x21\\
{\bf SW prefix:}  & DUMMY\\
{\bf SW offset:}  & 0x84\\
\end{tabular}


\vspace{12pt}
\noindent
\resizebox{\textwidth}{!}{
\begin{tabular}{>{\centering\arraybackslash}p{1.5cm} >{\centering\arraybackslash}p{1.5cm} >{\centering\arraybackslash}p{1.5cm} >{\centering\arraybackslash}p{1.5cm} >{\centering\arraybackslash}p{1.5cm} >{\centering\arraybackslash}p{1.5cm} >{\centering\arraybackslash}p{1.5cm} >{\centering\arraybackslash}p{1.5cm} }
31 & 30 & 29 & 28 & 27 & 26 & 25 & 24\\
\hline
\multicolumn{8}{|c|}{\cellcolor{RoyalPurple!25}DUMMY[31:24]}\\
\hline
23 & 22 & 21 & 20 & 19 & 18 & 17 & 16\\
\hline
\multicolumn{8}{|c|}{\cellcolor{RoyalPurple!25}DUMMY[23:16]}\\
\hline
15 & 14 & 13 & 12 & 11 & 10 & 9 & 8\\
\hline
\multicolumn{8}{|c|}{\cellcolor{RoyalPurple!25}DUMMY[15:8]}\\
\hline
7 & 6 & 5 & 4 & 3 & 2 & 1 & 0\\
\hline
\multicolumn{8}{|c|}{\cellcolor{RoyalPurple!25}DUMMY[7:0]}\\
\hline
\end{tabular}
}

\begin{itemize}
\item \begin{small}
{\bf 
DUMMY
} [\emph{read-only}]: DUMMY value to read
\end{small}
\end{itemize}





\end{document}
